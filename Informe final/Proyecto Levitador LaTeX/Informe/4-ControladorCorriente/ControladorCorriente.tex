\chapter{Controlador de corriente} \chapterlabel{Informe/4-ControladorCorriente} \label{cap:ControladorCorriente}

En este capítulo se diseña y modela el circuito encargado de controlar la corriente que circula por el electroimán. Como se vio en el capítulo anterior, el sistema trabaja con corrientes elevadas por lo que se implementan estrategias de conmutación para reducir las pérdidas de energía. Para ello se utiliza una topología de puente H con cuatro MOSFET y un \textsl{driver} que los controla. Además, se detallan los criterios tenidos en cuenta al momento de  elegir  y dimensionar todos los componentes que intervienen para lograr el correcto funcionamiento del controlador de corriente. Por último, se obtiene su función transferencia  para ser utilizada en el diseño del compensador.

\section{Descripción general}\label{sec_descripcion-general}

Para controlar la posición es necesario modificar la fuerza que ejerce el electroimán sobre la pieza móvil. Como se analizó en el capítulo \ref{cap:CaracterizacionElectroiman}, dicha fuerza esta determinada por la ecuación \ref{eq_fuerza_magnetica}, que se repite a continuación. 

\begin{equation*}
	\abs{F_{m}}=\frac{i^{2}*N^{2}*\mu_{o}*A}{4*Y_{g}^{2}}
\end{equation*}

En ella se ve que la fuerza magnética depende de la corriente del bobinado. Por lo tanto, se propone implementar un controlador de corriente que permita regular esta fuerza a partir de una tensión de entrada de referencia correspondiente al valor deseado. De esta forma, al variar dicha tensión se logra ajustar la fuerza ejercida.

\subsection{Comportamiento eléctrico del electroimán}\label{sec_comportamiento-electrico-electroiman}

Como se analizó en el capítulo \ref{cap:CaracterizacionElectroiman}, el electroimán puede ser modelado como una inductancia que varía con la distancia de entrehierro ($Y_g$) y una resistencia serie ($R_L$). Es decir, como un circuito RL serie cuya relación entre corriente de salida y tensión de entrada (admitancia) es:

\begin{equation} \label{eq_corriente}
	\frac{I_L}{V_L}(s)=\frac{1}{s*L_{(Y_g)}+R_L}
\end{equation}

Al aplicar la transformada inversa de Laplace a la expresión  \ref{eq_corriente}, se obtiene la respuesta temporal de la corriente ante un escalón de tensión en la entrada con amplitud $v_L$, considerando corriente inicial $I_o$ y constante de tiempo $\tau=\frac{L_{(Y_g)}}{R_L}$.

\begin{equation} \label{eq_corriente_temporal_cond_iniciales}
	i_L(t)=\frac{v_L}{R_L} + (I_o-\frac{v_L}{R_L})*e^{-\frac{t}{\tau}}
\end{equation}

En la expresión \ref{eq_corriente_temporal_cond_iniciales} se puede observar que la respuesta al escalón está compuesta por dos partes: un término con una exponencial negativa correspondiente al transitorio, y un término constante correspondiente al valor en régimen permanente $\frac{v_L}{R_L}$. El primero es el responsable de que la corriente en el inductor crezca de manera amortiguada, hasta alcanzar el valor de régimen permanente luego de cierto tiempo. Este comportamiento se puede observar en la simulación realizada en la figura \ref{fig:img_respuesta_escalon}. En la parte superior se observa la tensión de entrada y, en la inferior, la corriente del electroimán. Este análisis resulta de utilidad para conocer el comportamiento del electroimán y diseñar un controlador de corriente adecuado.


\begin{figure}[H]
	\centering
	\includegraphics[scale=0.5]{corriente_escalon.png}
	\caption{Respuesta ante una entrada en escalón.}
	\label{fig:img_respuesta_escalon}
\end{figure}


\section{Diseño del controlador}


Se desea controlar el valor medio de corriente que circula por el electroimán a partir de un sistema realimentado. Para ello se propone utilizar un controlador que trabaja en conmutación, alternando la alimentación del electroimán entre un valor superior positivo $\ V_{sup}$, y un valor inferior negativo $V_{inf}$. De esta manera, al controlar los tiempos de conmutación, se puede lograr una forma de onda como la que se muestra en la figura  \ref{fig:img_corriente_exponencial}. El resultado que se obtiene es una corriente con un valor medio correspondiente al deseado y un ripple superpuesto. Este último debe ser lo suficientemente pequeño comparado con el valor medio para que no contribuya en la fuerza magnética generada. 

\begin{figure}[H]
	\centering
	\includegraphics[scale=0.5]{Forma-de-onda-corriente-exponencial.png}
	\caption{Forma de onda de corriente y tensión en el electroimán.}
	\label{fig:img_corriente_exponencial}
\end{figure}

\colorbox{red}{modificar valores de imagen Vsup y Vinf}

Si se elige un período de conmutación lo suficientemente chico con respecto a la constante de tiempo de la planta, la forma de onda de la corriente en estado estacionario puede ser aproximada a una onda triangular como se muestra en la figura \ref{fig:img_corriente_lineal}. Por el mismo motivo, la ecuación \ref{eq_corriente_temporal_cond_iniciales} puede ser aproximada a la \ref{eq_corriente_taylor}.

\begin{equation} \label{eq_corriente_taylor}
	i_L(t)=I_o -  (I_o-\frac{V_L}{R_L})*\frac{t}{\tau}
\end{equation}


\begin{figure}[H]
	\centering
	\includegraphics[scale=0.5]{Forma-de-onda-corriente-lineal.png}
	\caption{Forma de onda de corriente al disminuir el período de conmutación.}
	\label{fig:img_corriente_lineal}
\end{figure}

Si bien la pendiente cambia según las condiciones iniciales, para este análisis se considera $I_o=0$ y luego se analiza cómo afecta en el sistema real. De esta forma, considerando que  $\tau=\frac{L_{(Y_g)}}{R_L}$ resulta:

\begin{equation} \label{eq_corriente_taylor_2}
	i_L(t)= \frac{V_L}{L_{(Y_g)}}*t
\end{equation}


Es posible observar en la expresión \ref{eq_corriente_taylor_2} que la pendiente de la onda triangular depende de la distancia de entrehierro. Esto es interesante ya que se podría conocer o estimar el valor de la distancia de separación a partir de la medición de dicha pendiente.

Debido a que para poder determinar la magnitud de fuerza magnética que debe generar el electroimán es necesario conocer la distancia de entrehierro, se propone medirla indirectamente a través de la pendiente de la onda triangular.

Por lo tanto, a continuación se analiza que variables del sistema afectan el valor de la pendiente y si es posible implementar el estimador de posición.

\subsection{Análisis de estimación de distancia de entrehierro}

La pendiente de la onda triangular de la corriente contiene información de la distancia de entrehierro. Por lo tanto, se calcula dicha pendiente haciendo la derivada de la expresión \ref{eq_corriente_taylor_2} con respecto al tiempo para medir indirectamente dicha separación.

\begin{equation} 
	\frac{di_L(t)}{dt}= \frac{V_L}{L_{(Y_g)}}
\end{equation}


Como se vio en el capítulo \ref{cap:CaracterizacionElectroiman}, la expresión \ref{eq_inductancia_vs_y} indica que la inductancia del electroimán es inversamente proporcional a la distancia de entrehierro. Por lo tanto se llega a:

\begin{equation}\label{eq_pendiente_vs_Yg}
	\frac{di_L(t)}{dt}= Y_g*\frac{2}{N^2*A*\mu_o}*V_L
\end{equation}

En la expresión \ref{eq_pendiente_vs_Yg}, la tensión con la que se alimenta al electroimán está representada por $V_L$. A partir de la figura \ref{fig:img_corriente_lineal} se pueden plantear dos casos para la pendiente: cuando crece (con $V_L=V_{sup}$) y cuando decrece (con $V_L=V_{inf}$). Por lo tanto, se obtienen dos expresiones:

\begin{equation} 
	\frac{di_L(t)}{dt}_{sup}= Y_g*\frac{2}{N^2*A*\mu_o}*V_{sup}
\end{equation}


\begin{equation}
	\frac{di_L(t)}{dt}_{inf}= Y_g*\frac{2}{N^2*A*\mu_o}*V_{inf}
\end{equation}

La tensión $V_L$ es un parámetro de diseño en el sistema, por lo tanto resulta conveniente elegir una fuente de alimentación cuyo valor sea $V_{cc}$ y que $|V_{sup}|=|V_{inf}|=|V_{cc}|$. De esta forma, se obtiene que el módulo de la pendiente es igual en ambos casos y resulta:

\begin{equation} \label{eq_corriente_taylor_3}
	\abs{\frac{di_L(t)}{dt}}= Y_g*\frac{2}{N^2*A*\mu_o}*\abs{V_{cc}}
\end{equation}

Como se muestra en la ecuación de \ref{eq_corriente_taylor_3}, la pendiente depende únicamente de la distancia $Y_g$. Por lo tanto, es posible obtener una estimación de la distancia de entrehierrpo a partir de la medición de la pendiente de la onda triangular.

\subsection{Diseño del lazo de control de corriente}

Para controlar la corriente que circula por el electroimán se propone un controlador de lazo cerrado del tipo ON-OFF con el agregado de una zona de histéresis.

En la figura \ref{fig:img_diag-en-bloques} se puede observar el controlador implementado, en el cual la diferencia entre la corriente de referencia y la que circula por el electorimán está representado por Ie. Esta última ingresa al controlador ON-OF con histéresis, que se encarga de alternar la polaridad de la tensión del electroimán.


\begin{figure}[H]
	\centering
	\includegraphics[width=\textwidth]{Diagrama-en-bloques.png}
	\caption{Diagrama en bloques del controlador de corriente con histéresis.}
	\label{fig:img_diag-en-bloques}
\end{figure}

\begin{figure}[ht]
	\centering
	%\begin{tikzpicture}[auto, node distance=2cm,>=latex']
%	\node [coordinate, name=entrada] {};
%	\node [sum, right=of entrada] (resta) {};
%	\node [coordinate, right=of resta] (ie) {};
%	\node [block, right=of ie] (comparador) {COMP};
%	\node [block, right=of comparador] (planta) {PLANTA};
%	\node [coordinate, right=of planta] (salida) {};
%	\node [coordinate, below of=planta] (realimentacion) {};
%	
%	\draw [->] (entrada) -- node {$I_{L_{ref}}$} (resta);
%	\draw [->] (resta) -- (comparador);
%	\draw [->] (comparador) -- (planta);
%%	\draw [->] (fft) -- node[name=a_resta] {} (resta);
%%	\draw [->] (fft) -- node[near end] {+} (resta);
%	\draw [->] (salida) |- (realimentacion);
%	\draw [->] (realimentacion) -| node[pos=0.95] {-} (resta);
%	\draw [->] (planta) -- node {$I_L$} (salida);
%	
%
%	
%%	\draw [->] (fft) |- (promedio);
%%	\draw [->] (promedio) -| node [near end] {-} (resta);
%\end{tikzpicture}

\tikzstyle{block} = [draw, fill=blue!20, rectangle, 
minimum height=3em, minimum width=6em]
\tikzstyle{sum} = [draw, fill=blue!20, circle, node distance=1cm]
\tikzstyle{input} = [coordinate]
\tikzstyle{output} = [coordinate]
\tikzstyle{pinstyle} = [pin edge={to-,thin,black}]

\def\windup{
	\tikz[remember picture,overlay]{
		\draw [stealth-stealth] (-1,0) -- node[pos=1] {$I_e$} (1,0); 
		\draw [stealth-stealth] (0,-0.6)--(0,0.6); %esto son los ejes
		\draw [-stealth] (-0.9,-0.4)-- (0.3,-0.4);
		\draw  [-stealth] (0.3,-0.4) -- (0.3,0.4); 
		\draw [-stealth] (0.9,0.4) --(-0.3,0.4);
		\draw [-stealth] (-0.3,0.4) -- (-0.3, -0.4); %esto sería la formita
}}
% The block diagram code is probably more verbose than necessary
\begin{tikzpicture}[auto, node distance=3cm,>=latex']
	% We start by placing the blocks
	\node [input, name=input] {};
	\node [sum, right of=input] (sum) {};
	\node [block, right of=sum] (controller) {\windup};
	\node [block, right of=controller, 
	node distance=3cm] (system) {System};
	% We draw an edge between the controller and system block to 
	% calculate the coordinate u. We need it to place the measurement block. 
	\draw [->] (controller) -- node[name=u] {$V_L$} (system);
	\node [output, right of=system] (output) {};
	\node [block, below of=u] (measurements) {Measurements};
	
	% Once the nodes are placed, connecting them is easy. 
	\draw [draw,->] (input) -- node {$I_{L_{ref}}$} (sum);
	\draw [->] (sum) -- node {$I_e$} (controller);
	\draw [->] (system) -- node [name=y] {$I_L$}(output);
	\draw [->] (y) |- (measurements);
	\draw [->] (measurements) -| node[pos=0.99] {$-$} 
	node [near end] {$I_{L_{feed}}$} (sum);
\end{tikzpicture}

	\caption{Diagrama en bloques simplificado del controlador de corriente}	\label{fig:diagrama_bloques_histeresis}
\end{figure}

En este tipo de controlador se alterna la polaridad de la tensión de alimentación del electroimán, con el objetivo de que la corriente se mantenga oscilando en torno a un valor medio de referencia. Para ello se define un margen de corriente $\Delta I_L$ de forma tal que, si la corriente que circula por el electroimán supera a la de referencia, no se producirá un cambio de polaridad en la tensión aplicada hasta que la supere por $\frac{\Delta I_L}{2}$. Análogamente, cuando comienza a decrecer, seguirá haciéndolo hasta que sea menor a la corriente de referencia menos $\frac{\Delta I_L}{2}$.

%% BORRADOR DE ALGO
Debido a que lo único que cambia es la polaridad de la tensión con la que se excita al electroimán, la constante de tiempo del circuito no cambia. Esto da como resultado que la corriente oscile sobre un valor medio con igual tiempo de crecimiento que de decrecimiento. Esta oscilación también es conocida como \textsl{ripple}. Su amplitud es fija y está determinada por el ancho de histéresis con el que se diseñe el controlador.

Para lograr una formad de onda mostrada en la  fig 3.3 el controlador actúa de la siguiente manera. 

Al iniciar, la corriente inicial $I_L$ es cero. Esto da como resultado a la entrada del comparador una $I_e = I_{ref}$ por lo tanto , el comparador excita al electroimán con $+V_{CC}$. De esta forma $I_L$ aumenta de forma exponencial hasta que el valor de $I_e$ sea $\frac{\Delta I_L}{2}$.Una vez alcanzado este valor, el bloque con histeresis actúa y se conmuta la tensión $V_h$ a $-V_{CC}$.

Al hacer esto la corriente comienza a decrecer hasta que el error es igual a $-\frac{\Delta I_L}{2}$. De igual forma, en este punto el bloque comparador actúa y la tensión $V_h$ conmuta $+V_{CC}$. Este ciclo se repite indefinidamente siempre y cuando la $I_{ref}$ sea constante.
%%

\subsection{Medición de corriente}


Como se puede ver en el diagrama en bloques \ref{fig:img_diag-en-bloques}, para poder realizar el control es necesario medir la corriente que circula por el electroimán y actuar en consecuencia.


Para hacer esto se debe utilizar un sensor que permita una medición de corriente hasta, por lo menos, $30\:A$  y con un error máximo tolerable tal que permita actuar al comparador y no afecte al sistema.

Los sensores de corriente trabajan en transresistencia ($rm$), por lo tanto, se agrega el bloque H (correspondiente a la ganancia del mismo) en el lazo de realimentación. Ahora $V_{iL}=i_{L}*rm$. Como se está trabajando con tensiones es necesario agregar un bloque $K_{in}$ para obtener el equivalente en tensión de $I_{ref}$ . El diagrama en bloques resultante es el siguiente (figura \ref{fig:img_diag-en-bloques-conH-y-Kin}):

\begin{figure}[H]
	\centering
	\includegraphics[width=\textwidth]{Diagrama-en-bloques-conH-y-Kin.png}
	\caption{Diagrama en bloques del controlador de corriente completo.}
	\label{fig:img_diag-en-bloques-conH-y-Kin}
\end{figure}

%ESTO AL FINAL ES AL PEDO CREO DE ULTIMA SE PUEDE HACER UN ANALISIS CON VALORES PERO MAS ADELANTE 
%Se puede expresar al error que introduce el sensor de efecto hall como $V_{errSensor}$. Entonces la expresión de error $V_{e}$ resulta: 

%\begin{equation}\label{eq_error_ve}
%	V_{e}= K_{in}*I_{ref}-H*iL\pm V_{errSensor}=\triangle I \pm V_{errSensor}
%\end{equation}


%De esta forma, tomando el  caso extremo $K_{in}*I_{ref}=H*i_{L}$ el error obtenido $V_{e}=\pm V_{errSensor}$.
%Es decir que el error es el propio error del sensor. Si se toma un ancho de histéresis mucho mayor al error máximo introducido por el sensor, este error no afectaría al sistema ya que seria despreciable a comparación del valor que deltaI debería alcanzar para que el comparador cambie de estado.

%El error que introduce el sensor de efecto hall $V_{errSensor}$ debe ser mucho menor al ancho de histéresis del comparador.

\section{Elección y calculo de parámetros del controlador}

En esta sección se determinarán los parámetros críticos para el correcto funcionamiento del controlador de corriente.

\subsection{Cálculo de tensión de alimentación}

Como se esta excitando a un sistema RL que posee una constante de tiempo ($\tau$) definida con una conmutación de tensión $+-Vcc$, la elección de este ultimo valor determina la velocidad de respuesta que el sistema tendrá ante perturbaciones o cambios en el punto  de operación.

%Es importante determinar cómo será la alimentación del controlador de corriente. Para hacerlo se debe tener en cuenta la velocidad de respuesta de la planta que está determinada por su constante de tiempo ($\tau$). Es decir, el sistema debe ser lo suficientemente rápido para modificar el valor medio de la corriente ante perturbaciones o cambios en el punto de operación. 

Un caso a analizar es cuando, en régimen permanente, se modifica bruscamente la carga que esta levitando. En esta situación el sistema debe aumentar o disminuir la corriente de forma abrupta para evitar que el objeto se caiga. El caso de mayor exigencia se da cuando a una distancia máxima de referencia $Y_g=5\:mm$ se modifica de carga mínima ($1\:Kg$) a máxima ($30\:Kg$). Utilizando la ecuación \label{eq_corriente_peso} se obtiene que la corriente para ambos casos es de:

\begin{equation}
	I_L(Y=5\:mm)[m=30\:Kg]=20.4\:A 
\end{equation}

\begin{equation}
	I_L(Y=5\:mm)[m=1\:Kg]=3.72\:A
\end{equation}

Dado que el polo dominante ya está definido por el circuito RL del electroimán, la velocidad con que el sistema pueda alcanzar un valor de corriente elevado está determinado por el valor de la fuente de alimentación. La expresión teórica de la corriente es: 

\begin{equation}
	I_L(t)=\frac{V}{R_L} + (I_o-\frac{V}{R_L})*e^{-\frac{t}{\tau}}
\end{equation}

Donde:
\begin{itemize}
	\item V es la tensión de alimentación, que puede tomar valores $+V_{cc}$ y $-V_{cc}$.
	\item $I_o$ es la corriente en el instante inicial.
	\item $\tau$ es la constante de tiempo del electroimán.
\end{itemize}

Para encontrar la expresión del tiempo que tarda la corriente en alcanzar el valor máximo de corriente $i_{max}$ se reemplaza $I_{0}=2.9$ y se obtiene $T_{1}$. Esto da:

\begin{equation}\label{eq_tiempo_de_subida}
	T_1=-\tau*ln(\frac{V_{cc}-R*I_{max}}{V_{cc}-R*I_{0}})
\end{equation}

Cuando se modifique la carga del objeto ,es necesario que el tiempo de subida de corriente $T_1$ sea mucho menor al tiempo en que la carga llegue a la distancia máxima que el sistema soporta $Y=5mm$ aprox (que corresponde a $I_{max}$). Es decir que el objeto cae libremente un delta $\Delta Y= 1mm $. Utilizando la ecuación que calcula el tiempo (T) que tarda en desplazarse un objeto en caída libre una altura $\Delta Y$:

\begin{equation}
	T=\sqrt{\frac{2*\Delta Y}{g}}=\sqrt{\frac{2*1mm}{9.81}}=14,27\:ms
\end{equation}

Finalmente reemplazando en \ref{eq_tiempo_de_subida} se obtiene: $Vcc>=21.6\:V$.

colorbox{Se puede decir cuanto da $T_1$ si tomamos 24V.}

\subsection{Cálculo de ancho de histéresis}

Como se mencionó, se desea controlar la fuerza ejercida a partir del valor medio de la corriente. Por lo tanto, las variaciones en torno a dicho valor medio no deben generar variaciones significativas en la fuerza magnética. Por ello, se debe elegir un ancho de histéresis tal que la frecuencia de conmutación resultante sea filtrada por la dinámica de la planta. Un valor de al menos 100 veces mayor que la frecuencia del polo de la planta obtenida en \ref{eq_transferencia_planta_m} serìa suficiente. Este se ubica en $70\:r/s$, lo que resulta en que se debe conmutar a una frecuencia de $\omega_{sw}>=7000\:r/s$, y expresada en Hz resulta $F_{sw}>=1\:kHz$.


Por lo tanto, como la frecuencia mínima es $F_{sw}$, y considerando que el tiempo en que crece la corriente es igual al que decrece, se obtiene que el tiempo máximo que puede tener la sección creciente de la corriente es igual a $t_{max}=500\:us$. 

A partir de la expresión \ref{eq_corriente_temporal_cond_iniciales} se puede obtener el valor máximo de ripple cuando $t=t_{max}$, considerando que la corriente inicial es $I_{min}$ y que la corriente final es $I_{min}+\Delta I_L$

\begin{equation} \label{eq_delta_i}
	I_{min}+\Delta I_{L_{max}}=\frac{v_L}{R_L}+(I_{min}-\frac{v_L}{R_L})*e^{-\frac{t_{max}}{\tau}}
\end{equation}

De la ecuación \ref{eq_delta_i} se puede despejar el valor máximo que puede tener $\Delta I_L$. 

\begin{equation} \label{eq_delta_i_2}
	\Delta I_{L_{max}}=6.06*10^{-3}*(\frac{v_L}{R_L}-I_{min})
\end{equation}
\colorbox{red}{Cambiar VL por VCC, y ver imagen de altium}

Sabiendo que el controlador de corriente tendrá una corriente media variable entre 0 y 30 A, se debe satisfacer la expresión \ref{eq_delta_i_2} para cualquier $I_{min}$ dentro de ese rango. Por lo tanto se plantean dos casos: $I_{min}=0$ e $I_{min}=30$. Para el primer caso se llega a que $\Delta I_{L_{max}}=727\:mA$ y en el segundo $\Delta I_{L_{max}}=500\:mA$. Por lo tanto se elige un ancho de histéresis de $500\:mA$ ya que cumple las dos condiciones.

\subsection{Elección del sensor de corriente}

Para poder diseñar un controlador de corriente apropiado es necesario realizar una medición sobre la corriente que circula por el bobinado del electroimán para que luego el controlador pueda actuar en consecuencia. Es necesario conocer tanto su valor medio, como su ripple. Es por ello que se debe idear una estrategia de medición que represente correctamente esta forma de onda cuyo valor medio puede alcanzar valores desde 0 hasta 30 A con un ripple de 500 mA.

En esta sección se analizan dos alternativas para lograr este objetivo. La primera mediante una resistencia en serie al electroimán y la segunda utilizando un sensor de efecto Hall.


\subsubsection{Análisis de medición de corriente mediante resistencia shunt}
La forma de medir corriente que resulta, en principio, más simple es la de utilizar una resistencia de valor $R_s$ en serie con el electroimán como se muestra en la figura \ref{fig:img_puente_con_rshunt}, y medir en sus terminales la diferencia de tensión generada por la corriente. A partir de esta tensión ($V_s-V_a$) se puede utilizar la ley de Ohm para conocer el valor de corriente:

\begin{equation}
	I_L=\frac{V_s-V_a}{R_s}
\end{equation}


\begin{figure}[H]
	\centering
	\includegraphics[width=\textwidth]{puente_con_rshunt.png}
	\caption{Puente H con resistencia de sensado de corriente (Rs).}
	\label{fig:img_puente_con_rshunt}
\end{figure}

Por otro lado, la realimentación del controlador de corriente debe ser en forma de tensión, por lo tanto es suficiente con medir la tensión diferencial $V_s-V_a$ y tener en cuenta en el diseño del controlador la ganancia que se tiene al pasar de corriente a tensión.

Aunque este método para medir corriente pareciera directo, presenta algunos inconvenientes en el diseño:

El primero es que al agregar una resistencia en serie al electroimán, se estaría agregando una mayor disipación de potencia en el sistema. Para intentar reducir este problema se podría elegir un valor de resistencia lo suficientemente bajo para que su consumo sea despreciable. Por ejemplo, se podría adoptar una resistencia de $10\:mOhms$. Este valor resulta en pérdidas de potencia de $4\:W$, que es un valor aceptable. 

El segundo inconveniente es que alteraría la dinámica de la planta, ya que la constante de tiempo sería $\tau=\frac{L}{R_L+R_s}$. Sin embargo, el electroimán presenta una resistencia interna de $0.2\:Ohm$, por lo que una resistencia de sensado con valor $10\:mOhm$ no afectaría en gran medida la dinámica.

El tercero es que se debe realizar una medición de tensión flotante. Esto se debe a que la resistencia, al estar en serie con el electroimán, no tiene ningún punto de medición referido a masa. Por lo tanto, se debe utilizar un amplificador que mida tensión en modo diferencial para luego obtener una señal en modo común. El inconveniente que se presenta es que cada uno de los puntos de medición se encuentra a un alto potencial respecto de masa y, además, este cambia en cada conmutación. Esto genera que durante los transitorios de conmutación haya ruido en la medición diferencial.

Debido a que se requiere medir el valor de corriente sin que el ruido de modo común altere la medición, se propone analizar otra alternativa que sea inmune a dicho efecto.


\subsubsection{Análisis de medición de corriente mediante sensor de efecto Hall}

Dado que la medición con una resistencia de sensado introduce ruido ocasionado por la conmutación de las llaves, se plantea la alternativa de utilizar un sensor de efecto Hall. Este dispositivo mide el campo magnético generado por la corriente, entregando a su salida una tensión proporcional a ésta. La principal ventaja que presenta es que el campo magnético medido sólo es sensible a las variaciones de corriente y no a las conmutaciones de tensión.

Existen una gran variedad de estos sensores en el mercado, cada uno con diferentes características. A continuación se mencionan los criterios que se tendrán en cuenta para la elección del sensor:

\begin{itemize}
	\item Debe ser capaz de medir una corriente de hasta $30\:A$.
	\item El ancho de banda debe ser mucho mayor al polo de la frecuencia de conmutación del controlador de corriente para poder conservar la forma de onda de la corriente triangular a medir. Por lo tanto, debe ser al menos de $100\:kHz$.
	\item La transresistencia debe ser lineal entre $0\:A$ y $30\:A$.	
\end{itemize}

A partir de estas características se decidió utilizar el sensor HO 15-NP-0000 \cite{HO15-NP}. Este permite medir una corriente de $\pm 37.5\:A$ con un ancho de banda de $250\:kHz$ y posee una transresistencia de $53.33\:mV/A$ en todo  el rango de corriente. Además, presenta alta inmunidad a interferencias externas. 

Este sensor tiene la capacidad de medir tanto corrientes en sentido positivo, como negativo. Para ello admite una tensión de bias de $2.5\:V$, la cual se corresponde a la salida cuando la corriente es nula. Cuando la circulación de corriente es positiva, la salida del sensor resulta en una tensión mayor a $2.5\:V$, y para negativas, menor.

De esta forma, el bloque H de realimentación queda definido como:

\begin{equation}
	H=\frac{V_iLF}{I_L}=53,33 mV/A
\end{equation}



\subsection{Cálculo de ganancia de entrada}


