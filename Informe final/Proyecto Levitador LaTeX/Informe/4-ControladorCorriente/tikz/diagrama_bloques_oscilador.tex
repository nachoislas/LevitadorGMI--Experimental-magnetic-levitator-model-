%importo el bloque de histeresis que definí en otro archivo
%esto es para poder dibujar flechitas en la histeresis usando ->-
\tikzset{->-/.style={decoration={
			markings,
			mark=at position .5 with {\arrow{>}}},postaction={decorate}}}

%aca se define el dibujito de la histeresis uniendo lineas
\def\hist{
	\tikz[remember picture,overlay]{
		%dibujamos los ejes. stealth es para las flechas
		\draw [stealth-stealth] (-1,0) -- (1,0) node[pos=1, yshift=0.5em]{${\scriptstyle e}$};  %eje x
		\draw [stealth-stealth] (0,-0.8)--(0,0.8) node[pos=0.9, xshift=0.8em]{${\scriptstyle V_L}$}; %eje y
		
		%esto es para dibujar los tramos de la histeresis
		\draw [->-] (-0.7,-0.4)-- (0.3,-0.4);
		\draw  [->-] (0.3,-0.4) -- (0.3,0.4); 
		\draw [->-] (0.7,0.4) --(-0.3,0.4);
		\draw [->-] (-0.3,0.4) -- (-0.3, -0.4);
}}


%Acá se define eñ diagrama en bloques completo
\begin{tikzpicture}[auto, node distance=2.5cm,>=latex']
	% We start by placing the blocks
	\node [input, name=input] {};
	\node [block, right of=input] (Kin) {$K_{in}$};
	\node [sum, right of=Kin, node distance=2cm] (sum) {};
	\node [block, right of=sum] (controller) {\hist};
	\node [draw,circle, right of=controller](multiplier){X};
	\node [coordinate, above of=multiplier, node distance=1cm](osc) {Fsw}; 
	\node [block, right of=multiplier, 
	node distance=3.5cm] (system) {${\displaystyle\frac{1}{s*L(Y_g)+R_L}}$};
	% We draw an edge between the controller and system block to 
	% calculate the coordinate u. We need it to place the measurement block. 
	\draw [->] (controller) -- node[name=u] {$V_L$} (multiplier);
	\draw [->] (multiplier) -- (system);
	\draw [->] (osc) -| (multiplier);
	\node [output, right of=system] (output) {};
	\node [block, below of=u] (measurements) {H(s)};
	
	% Once the nodes are placed, connecting them is easy. 
	\draw [draw,->] (input) -- node[pos=0.1]{$V_{IL_{ref}}$} (Kin);
	\draw [draw,->] (Kin) -- node[pos=0.9]{$+$} (sum);
	\draw [->] (sum) -- node {$e$} (controller);
	\draw [->] (system) -- node [name=y] {$I_L$}(output);
	\draw  (y) |- (measurements);
	\draw [->] (measurements) -| node[pos=0.99] {$-$} node[above, pos=0.9, right] {$V_{IL_{feed}}$} (sum); %node [near end] {$I_{L_{feed}}$} 
\end{tikzpicture}


