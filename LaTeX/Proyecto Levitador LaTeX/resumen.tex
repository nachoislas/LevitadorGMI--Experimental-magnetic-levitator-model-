%Los convertidores multifásicos se han convertido en una solución muy atractiva para el control de corriente en aplicaciones de alta potencia, debido a la reducción de las pérdidas en los elementos activos y pasivos del convertidor, y la mejora en las características del \textit{ripple} total. 
%
%En estas aplicaciones, comúnmente presentes en el campo de física de altas energías, es necesario controlar corrientes del orden del kiloampere, con una precisión del orden de cientos de partes por millón. Además, existen grandes cambios en la referencia de corriente y perturbaciones en las tensiones de entrada y salida, los cuales generan errores transitorios que se deben corregir en algunos cientos de microsegundos. Debido a los niveles de corriente y a las capacidades tecnológicas de los dispositivos semiconductores, los tiempos de establecimiento requeridos se traducen a unos pocos ciclos de conmutación. Bajo estas condiciones, se generan exigencias que no pueden ser completamente satisfechas por los controles de corriente existentes, debido a su limitada respuesta transitoria, sensibilidad a los elementos parásitos del convertidor o al punto de operación.
%
%En virtud de estas problemáticas, en esta tesis se realiza un estudio del control de corriente en convertidores multifásicos, y se proponen soluciones tendientes a satisfacer las especificaciones de respuesta dinámica y precisión antes mencionadas. 
%
%En primer lugar, se propone una caracterización del ripple de la corriente total, en función de diferencias entre los valores de los inductores de fase. Esta caracterización permite hallar expresiones para el ripple de la corriente total que, a diferencia de los enfoques presentes en la literatura, son de carácter general, válidas para cualquier número de fases y ciclo de trabajo. A partir de estas expresiones se derivan magnitudes tales como valor pico, valor RMS y contenido armónico del ripple total. 
%
%La caracterización propuesta permite el desarrollo de un método de mitigación que mejora en forma significativa el ripple de la corriente total en condiciones prácticas, respecto al único método presente en la literatura basado en el mismo principio, sin incrementar en forma significativa el esfuerzo computacional.
%
%Finalmente se propone un control de corriente para convertidores multifásicos, que permite un seguimiento preciso del valor medio de la referencia de corriente, en conjunto con tiempos transitorios del orden de unos pocos ciclos de conmutación ante cambios en la referencia o perturbaciones en las tensiones de entrada o salida del convertidor.
%
%Las propuestas fueron validadas a través de simulaciones y ensayos experimentales en un prototipo a baja escala.