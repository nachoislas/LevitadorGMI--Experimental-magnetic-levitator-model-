\tikzset{%
	buffer/.style={
		draw,
		shape border rotate=270,
		regular polygon,
		regular polygon sides=3,
		fill=blue!20,
		node distance=2cm,
		minimum height=4em
	}
}

\tikzstyle{block} = [draw, fill=blue!20, rectangle, 
minimum height=2.5em, minimum width=3em]

%Acá se define eñ diagrama en bloques completo
\begin{tikzpicture}[auto, node distance=1cm,>=latex']
	% We start by placing the blocks
	\node [input, name=input] {};
	\node [buffer, right=of input](F){F};
	\node [sum, right of=F, node distance=1.5cm] (suma_externa) {+};
	\node [block, right=of suma_externa] (externo) {$G_{ext}(w)$};
	\node [sum, right=of externo] (suma_interna) {+};
	\node[block, right=of suma_interna] (interno) {$G_c(w)$};
	\node [block, right=of interno] (gil) {$G_{IL}(w)$};
	\node [block, right=of gil] (planta) {$G_p(w)$};
	\node [coordinate, below=of interno] (realimentacion_interna) {$H_{estim}(w)$};
	
	\node [coordinate, below=2cm of externo] (realimentacion_externa) {$H_{estim}$};
%	
%	\node [block, right of=suma] (amplificador) {$A(s)$};
	\node [output, right of=planta, node distance=3cm] (output) {};
%	\node [block, below of=amplificador] (realimentacion) {$H(s)$};
%	
%	
%	% Once the nodes are placed, connecting them is easy. 
	\draw [draw,->] (input) -- node[pos=0.2]{$V_{y_{ref}}$} (F);
	\draw [draw,->] (F) -- node[pos=0.9]{$+$}(suma_externa);
	\draw [draw,->] (suma_externa) -- (externo);
	\draw [draw,->] (externo) -- node[pos=0.95]{$+$} (suma_interna);
	\draw [draw,->] (suma_interna) -- (interno);
	\draw [draw,->] (interno) -- (gil);
	\draw [draw,->] (gil) -- (planta);
	\draw [draw,->] (planta) -- node[name=y]{$Y_g$} (output);
%	\draw [draw,->] (amplificador) -- node[name=y]{$V_{deriv}$} (output);
	\draw [-] (y) |- (realimentacion_interna);
	\draw [->] (realimentacion_interna) -|  node[pos=0.99]{$-$} (suma_interna);
	\draw [-] (y) |- (realimentacion_externa);
	\draw [->] (realimentacion_externa) -|  node[pos=0.99]{$-$} (suma_externa);
\end{tikzpicture}