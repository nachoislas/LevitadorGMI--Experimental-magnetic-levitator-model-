%Acá se define eñ diagrama en bloques completo
\begin{tikzpicture}[auto, node distance=3.5cm,>=latex']
	% We start by placing the blocks
	\node [input, name=input] {};
	\node [block, right of=input] (Gin) {$G_{in}(s)$};
	\node [sum, right of=Gin, node distance=3.5cm] (suma) {+};
	
	\node [block, right of=suma] (amplificador) {$A(s)$};
	\node [output, right of=amplificador] (output) {};
	\node [block, below of=amplificador] (realimentacion) {$H(s)$};
	
	
	% Once the nodes are placed, connecting them is easy. 
	\draw [draw,->] (input) -- node[pos=0.1]{$V_{IL_{feed}}$} (Gin);
	\draw [draw,->] (Gin) -- node[pos=0.9]{-}(suma);
	\draw [draw,->] (suma) -- (amplificador);
	\draw [draw,->] (amplificador) -- node[name=y]{$V_{deriv}$} (output);
	\draw [->] (y) |- (realimentacion);
	\draw [->] (realimentacion) -| node[pos=0.99]{-} (suma);
\end{tikzpicture}