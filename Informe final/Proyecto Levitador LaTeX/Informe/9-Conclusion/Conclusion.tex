\chapter{Conclusión} \chapterlabel{Informe/9-Conclusion} \label{cap:Conclusión}

Es importante destacar que el proyecto se realizó prácticamente en su totalidad bajo la situación de emergencia sanitaria debido al COVID-19.  Esto conllevó a grandes demoras en los plazos estimados para cada una de las etapas del proyecto. Los mayores retrasos se dieron en su comienzo, debido a que coincidió con el período de mayor incertidumbre de la pandemia. Sin embargo, a pesar de no poder realizar encuentros presenciales entre los integrantes ni reuniones con los directores en los establecimientos educativos, se pudo avanzar hasta su finalización  mediante el uso de plataformas de comunicación virtuales.

Debido a la extensión en los plazos temporales del proyecto y a los costos de fabricación e implementación de la placa de control fue necesario acortar el alcance del proyecto al diseño teórico de todas las etapas que componen el sistema, junto con el diseño del circuito impreso.

A pesar de lo interesante y “nutritivo” que hubiese resultado para los integrantes del proyecto poder construir el prototipo y realizar mediciones y observaciones sobre el mismo, fué posible adquirir conocimientos en distintos conceptos propios de la electrónica y la ingeniería en general como son: modelado de sistemas físicos, análisis y desarrollo de sistemas de control tanto en el dominio analógico como en el discreto, diseño de circuitos impresos, y electrónica de potencia y digital. Además, fue posible ganar experiencia en la utilización de herramientas de diseño de circuitos impresos, como Altium Designer; software’s para la simulación de circuitos como Matlab y NL5; softeware’s para el control de versiones, como Git; y software’s para la escritura de documentos formales, como LaTex. Por otra parte, fue posible desarrollar habilidades interpersonales, como trabajo en equipo, comunicación, respeto y empatía, gestión del tiempo, compromiso, dedicación, entre otras. Todas estas habilidades son necesarias e indispensables para el desarrollo profesional y personal..

En cuanto a los alumnos de la cátedra de Sistemas de Control, por más de que no podrán disponer de un prototipo funcional para realizar mediciones y observaciones, en este documento se detalla el diseño de cada una de las etapas que componen al sistema y el criterio utilizado para la elección de cada topología y componente. De esta forma, podrán analizar y comprender las problemáticas que surgen en este tipo de sistemas y la forma en que fueron abordadas. Por otra parte, se deja a disposición de la cátedra  los archivos de fabricación de la placa de control para que en un futuro, cuando sea posible, se envíe a prototipar y pueda ser utilizada por los alumnos.
