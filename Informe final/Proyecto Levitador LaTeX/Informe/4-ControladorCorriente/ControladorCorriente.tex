\chapter{Controlador de corriente} \chapterlabel{Informe/4-ControladorCorriente} \label{cap:ControladorCorriente}

En este capítulo se diseña y modela el circuito encargado de controlar la corriente que circula por el electroimán. Como se vio en el capítulo anterior, el sistema trabaja con corrientes elevadas por lo que se implementan estrategias de conmutación para reducir las pérdidas de energía. Para ello se utiliza una topología de puente H con cuatro MOSFET y un \textsl{driver} que los controla. Además, se detallan los criterios tenidos en cuenta al momento de  elegir  y dimensionar todos los componentes que intervienen para lograr el correcto funcionamiento del controlador de corriente. Por último, se obtiene su función transferencia  para ser utilizada en el diseño del compensador.

\section{Descripción general}

Para mantener en suspensión a la pieza móvil es necesario regular la fuerza electromagnética generada por el electroimán. Esto se logra modificando la intensidad de la corriente que circula por su bobinado. Para ello, es necesario diseñar una fuente de alimentación que sea capaz de proveer la corriente requerida.

Para diseñar la fuente de alimentación se debe conocer el comportamiento eléctrico de la planta. Como se analizó en el capítulo \ref{cap:CaracterizacionElectroiman}, el electroimán puede ser modelado como una inductancia variable con una resistencia serie. Es decir, es un circuito RL serie cuya corriente ($I_L$) depende de la tensión aplicada ($V_L$). La expresión \ref{eq_corriente} muestra la relación entre estos parámetros.

\begin{equation} \label{eq_corriente}
\frac{I_L}{V_L}(s)=\frac{1}{sL(Y_g)\ +\ R_L}
\end{equation}

Al aplicar la transformada inversa de Laplace a la expresión  \ref{eq_corriente}, se puede obtener la respuesta temporal de la corriente ante un escalón de tensión de amplitud ``$v_L$'' en la entrada:

\begin{equation} \label{eq_corriente_temporal}
	i_L(t)=\frac{v_L}{R_L}*(1-e^{-\frac{R_L}{L(Y_g)}*t})
\end{equation}

En la expresión \ref{eq_corriente_temporal} se puede observar que la respuesta al escalón está compuesta por dos partes: un término con una exponencial negativa correspondiente al transitorio, y un término constante correspondiente al valor en régimen permanente $\frac{v_L}{R_L}$. El primero provoca que la corriente en el inductor crezca de manera amortiguada, hasta alcanzar el valor de régimen permanente luego de cierto tiempo. Este comportamiento se puede observar en la simulación realizada en la figura \ref{fig:img_respuesta_escalon}. En la parte superior de la figura se observa la tensión de entrada, y en la parte inferior la corriente del electroimán. Este análisis resulta de utilidad para conocer el comportamiento del electroimán y diseñar un controlador de corriente adecuado.


\begin{figure}[H]
	\centering
	\includegraphics[scale=0.5]{corriente_escalon.png}
	\caption{Respuesta ante una entrada en escalón.}
	\label{fig:img_respuesta_escalon}
\end{figure}


\section{Diseño de topología}


Se desea diseñar un sistema de control que modifique la alimentación del electroimán con el objetivo de que circule una corriente continua deseada por su bobinado.  Para ello, se propone implementar un sistema realimentado que compare la corriente que circula por el electroimán con una de referencia, que es la que se desea que circule. En la figura \ref{fig:img_diagrama_bloques_basico} se muestra un diagrama en bloques simplificado del sistema.

\begin{figure}[H]
	\centering
	\includegraphics[scale=1]{Diagrama-en-bloques-basico.png}
	\caption{Diagrama en bloques básico del controlador de corriente.}
	\label{fig:img_diagrama_bloques_basico}
\end{figure}

De este diagrama surgen dos necesidades:
\begin{itemize}
	\item Diseñar un controlador que modifique la alimentación del electroimán según sus entradas.
	\item Medir la corriente que circula en el electroimán para poder utilizarla como entrada al controlador.
\end{itemize}

A continuación se analizan distintas alternativas para satisfacer estas necesidades.

\subsection{Control de corriente mediante transistor en modo lineal}

Como se vio en la introducción del capítulo, el valor en régimen permanente de la corriente depende proporcionalmente de la tensión aplicada al electroimán. Por lo tanto, 

Para mantener una corriente constante controlada se podría utilizar un transistor en modo lineal, es decir, como una fuente de corriente constante. Una posible implementación circuital se muestra en la figura \ref{fig:img_controlador-lineal}.

\begin{figure}[H]
	\centering
	\includegraphics[scale=0.7]{controlador_lineal.png}
	\caption{Control de corriente mediante transistor en modo lineal.}
	\label{fig:img_controlador-lineal}
\end{figure}

En este modo de funcionamiento la tensión colector-emisor ($V_{CE}$) se controla de manera que la diferencia entre esta y la tensión de alimentación, circulando por la resistencia interna del electroimán, generen la corriente deseada. Es decir, en régimen permanente se obtiene:
 
 \begin{equation} \label{eq_corriente_temporal}
 	I_L=\frac{V_{CC}-V_{CE}}{R_L}
 \end{equation}
 
La desventaja de esto es que al circular una corriente constante, la caída de tensión sobre el inductor es prácticamente nula (solo lo correspondiente a la resistencia interna). Por lo tanto, la mayor parte de la potencia disipada cae sobre el transistor. Esta puede calcularse como: 

\begin{equation}
	P_{transistor} = I_L*V_{CE}
\end{equation}

Teniendo en cuenta que la tensión colector-emisor es la resta de la tensión de alimentación y la caida en la resistencia del electroimán:

\begin{equation}
	V_{CE}=V_{CC}-I_L*R_L
\end{equation}

Finalmente:

\begin{equation}
	P_{transistor}=V_{CC}*I_L-I_L^2*R_L
\end{equation}

Esta función llega a un máximo cuando $I_L=\frac{V_{CC}}{2*R_L}$


\begin{equation}\label{eq_pot_transistor_lineal_final}
	P_{transistor_{max}}=\frac{V_{CC}^2}{4*R_L}
\end{equation}

Teniendo en cuenta que la corriente maxima que se requiere es de aproximadamente 30 A se puede obtener que la minima tension de VCC requerida es:

\begin{equation}
	V_{CC}=I_L*R_L+V_{CE}=30 A * 0,2  = 5 V, VCE=0
\end{equation}

En el mercado las fuentes de tension capaces de entregar 30 A mas comunes en el mercado comienzan con valores minimos de 12 V. Si se reemplaza este valor en la ecuacion  \ref{eq_pot_transistor_lineal_final} se obtiene un valor minimo de 180 Watt, esto elevaria demasiado la temperatura del transistor, haciendo que sea necesario el agregado de un gran disipador de calor y pone en riesgo la vida útil de los componentes.
  
En la ecuación \ref{eq_pot_transistor_lineal_fina} se puede ver que para valores de tension normalmente utilizados $(5\:V,10\:V,15\:V)$ la potencia disipada por el transistor varía entre $30\:W$ hasta $280\:W$. Esto eleva demasiado la temperatura del transistor, haciendo que sea necesario el agregado de un gran disipador de calor y pone en riesgo la vida útil de los componentes.

Aunque aún no se define con qué tensión se alimentará el sistema, se puede probar con valores utilizados comunmente para obtener una aproximación de la potencia disipada por el transistor. Por ejemplo, para valores de tensión de alimentación $(5\:V,10\:V,15\:V)$ el valor de potencia que se obtiene varía entre $30\:W$ hasta $280\:W$. Esto eleva demasiado la temperatura del transistor, haciendo que sea necesario el agregado de un gran disipador de calor y pone en riesgo la vida útil de los componentes.

\colorbox{red}{HACER UN MEJOR ENGANCHE}

\subsection{Control de corriente mediante conmutación}

Como se analizó en la sección anterior, al trabajar con corrientes elevadas, no es eficiente utilizar un transistor que trabaje en su zona lineal puesto que el consumo de energía es elevado. Se propone entonces utilizar una fuente conmutada.

%\noindent Para lograr una corriente continua en el electroimán mediante una fuente conmutada se debe alternar la polaridad de la tensión aplicada en sus bornes. De esta forma, se aprovecha el transitorio mostrado en la figura \ref{fig:img_respuesta_escalon} para hacer crecer la corriente hasta llegar a un cierto valor (denominado límite superior), y luego se conmuta la polaridad de la alimentación para que decrezca. Nuevamente, al llegar a un cierto valor (denominado límite inferior), se vuelve a conmutar. Este proceso se repite, y se logra que la corriente oscile en torno al valor medio entre los dos límites, que es el valor de corriente continua deseado. 

Al aplicar tensión en los bornes del electroimán, la magnitud de la corriente aumentará en forma exponencial. Si se desea que la corriente que circule por su bobinado sea aproximadamente continua, es necesario alternar la polaridad de la tensión aplicada ($\pm V_L$) de forma que la corriente oscile en torno al valor medio deseado ($<I_L>$) como se observa en la figura \ref{fig:img_corriente_exponencial}.

\begin{figure}[H]
	\centering
	\includegraphics[scale=0.5]{Forma-de-onda-corriente-exponencial.png}
	\caption{Forma de onda de corriente y tensión en el electroimán.}
	\label{fig:img_corriente_exponencial}
\end{figure}

%En la figura \ref{fig:img_corriente_exponencial} se puede ver que la corriente crece y decrece en forma exponencial dentro de cada ciclo de conmutación. Sin embargo, si se elige un intervalo de tiempo de la conmutación pequeño comparado con la constante de tiempo de la planta, el incremento de corriente será pequeño y puede ser aproximado como una rampa. Por lo tanto, se obtiene una corriente continua con un ripple superpuesto de forma triangular, como la que se muestra en la figura \label{fig:img_corriente_lineal}.

%\begin{figure}[H]
%	\centering
%	\includegraphics[scale=0.5]{forma-de-onda-corriente-lineal.png}
%	\caption{Forma de onda de corriente al disminuir el período de conmutación.}
%	\label{fig:img_corriente_lineal}
%\end{figure}

Por lo tanto, es necesario diseñar un circuito que permita alternar la polaridad de la tensión aplicada al electroimán y controlar el valor medio de su corriente. Entre las distintas topologías de fuentes conmutadas que se utilizan comúnmente, una que tiene la capacidad de alimentar su carga en ambos sentidos es la topología de puente completo con cuatro llaves que se muestra en la figura \ref{fig:img_topologia_simplificada}.

\begin{figure}[H]
	\centering
	\includegraphics[scale=0.5]{puente_con_llaves.png}
	\caption{Topología simplificada.}
	\label{fig:img_topologia_simplificada}
\end{figure} 


El electroimán se conecta entre los puntos medios de cada par de llaves. Cada una de ellas puede ser controlada de manera independiente mediante señales eléctricas. A través de una combinación correcta de las mismas se puede controlar la polaridad de la alimentación y el sentido de circulación de la corriente, según se desee. Para esta aplicación en particular, la fuerza magnética es siempre en el mismo sentido, independientemente del sentido en que circule la corriente. Por lo tanto, se adopta como sentido de circulación positivo de izquierda a derecha como lo indican las flechas en la figura \ref{fig:img_topologia_simplificada}.

Para poder obtener una forma de onda de corriente como la que se muestra en la figura \ref{fig:img_corriente_exponencial}, comenzando desde corriente nula, se debe controlar el estado de las llaves de la siguiente manera:

\begin{itemize}
	\item En principio se cierran las llaves $Q_1$ y $Q_4$ a la vez, generando un circuito entre $V_{cc}$, el electroimán y GND como indican las flechas negras en la figura \ref{fig:img_topologia_simplificada}. De esta forma, la corriente comienza a crecer con forma de exponencial negativa. 
	\item Al llegar al límite superior ($I_{MAX}$), el controlador conmuta el estado de las llaves, de manera que $Q_1$ y $Q_4$ dejan de conducir, y comienzan a hacerlo $Q_2$ y $Q_3$. De esta manera, la corriente seguirá circulando en el mismo sentido como indican las flechas rojas, pero ahora la diferencia de potencial en los bornes del electroimán se opone al paso de la corriente, por lo que su magnitud comienza a decrecer.
	\item Una vez alcanzado el límite inferior ($I_{MIN}$), el controlador vuelve a conmutar el estado de las llaves para que la corriente tenga pendiente positiva.
\end{itemize}

Este ciclo se repite en régimen permanente para obtener la corriente deseada. Si se elige un tiempo de conmutación pequeño comparado a la constante de tiempo de la planta, la fuerza magnética producida por esta corriente oscilante podrá ser filtrada, quedando únicamente su valor medio. Es decir, no habrá variaciones significativas en la fuerza ejercida.

Es importante tener en cuenta que sólo dos llaves pueden encenderse a la vez, y esto debe realizarse de manera diagonal. Es decir, en la figura \ref{fig:img_topologia_simplificada}, $Q_1$ y $Q_4$ pueden estar encendidos, mientras que $Q_3$ y $Q_2$ están apagados, y viceversa. Esto es debido a que se podría generar un cortocircuito entre la fuente de alimentación y GND, produciendo una circulación de corriente elevada que podría dañar el sistema y la fuente de alimentación. Se debe tener en cuenta esta restricción al momento de diseñar el circuito encargado de controlar estas llaves.

Por otro lado, al considerar que el tiempo de conmutación es pequeño comparado a la constante de tiempo de la planta, la forma de onda de la corriente en régimen permanente será triangular puesto que no llegará a tomar forma exponencial. Además, como la tensión aplicada a los bornes del electroimán en cada semiciclo de conmutación es igual en magnitud y el circuito de carga y descarga del electroimán es el mismo, la onda triangular tendrá pendientes simétricas. La forma de onda resultante se puede observar en rojo en la figura \ref{fig:img_corriente_triangular}.

\begin{figure}[H]
	\centering
	\includegraphics[scale=0.5]{Forma-de-onda-corriente-lineal.png}
	\caption{Forma de onda de corriente triangular}
	\label{fig:img_corriente_triangular}
\end{figure} 


Como se mencionó previamente, para poder generar una corriente con el valor medio deseado es necesario controlar el tiempo que se le aplica cierta polaridad de tensión al electroimán. Para poder controlar dicha polaridad, se debe actuar sobre las llaves en función de si se desea aumentar o disminuir el valor medio de la corriente. 

Una manera de lograrlo es mediante un circuito que compare la corriente de referencia con la corriente que está circulando por el electroimán. De esta forma, si la corriente es menor a la de referencia, la polaridad de la tensión aplicada en los bornes del electroimán, será positiva. En cambio, si resulta mayor, será negativa. De esta forma, la corriente aumenta y disminuye respectivamente. Por lo tanto, se plantea el  diagrama en bloques mostrado en la figura \ref{fig:img_diag-en-bloques-comparador-sin-hist}.

\begin{figure}[H]
	\centering
	\includegraphics[width=\textwidth]{Diagrama-en-bloques-comparador-sin-hist.png}
	\caption{Diagrama en bloques simplificado del controlador de corriente.}
	\label{fig:img_diag-en-bloques-comparador-sin-hist}
\end{figure}

Al analizar el diagrama en bloques planteado en la figura \ref{fig:img_diag-en-bloques-comparador-sin-hist} es posible notar que, una vez que la corriente del electroimán (IL(s)) supere infinitesimalmente a la referencia, se produce una conmutación en la polaridad de tensión aplicada al electroimán. Lo mismo sucede cuando es infinitesimalmente menor. El inconveniente que esto presenta es que se producirían conmutaciones extremadamente rápidas en torno al valor medio, por lo que sería necesario alta velocidad en conmutación. Por lo tanto, para evitar esas oscilaciones, resulta conveniente implementar un comparador con histéresis como se muestra en la figura \ref{fig:img_diag-en-bloques}. Este permite definir un margen de corriente $\Delta I_L$ de forma tal que, si la corriente que circula por el electroimán supera a la de referencia, no se producirá un cambio de polaridad en la tensión aplicada hasta que la supere por $\Delta I_L$. Análogamente, cuando comienza a decrecer, seguirá haciéndolo hasta que sea menor a la corriente de referencia menos $\Delta I_L$.

\begin{figure}[H]
	\centering
	\includegraphics[width=\textwidth]{Diagrama-en-bloques.png}
	\caption{Diagrama en bloques del controlador de corriente con un comparador con histéresis.}
	\label{fig:img_diag-en-bloques}
\end{figure}

Debido a que lo único que cambia es el valor de tensión en modulo con el que se exita al electroimán la constante de tiempo del circuito no cambia. Esto da como resultado a que la corriente oscile sobre un valor medio de forma triangular. El valor de el ripple de corriente esta determinado por el periodo de conmutación de la fuentes. Estos parámetros pueden ser definidos ajustando el ancho de histéresis. 


Como se mencionó, la forma de onda tendrá un ripple en torno a un valor medio. La elección del ancho de este ripple determinará la velocidad de respuesta que el comprador pueda tener. Esto quiere decir que el error ($I_{e}$(cambiar desp a Ie en la figura)) aumentará hasta cierto valor definido para luego ser corregido. Si $I_{e}$ alcanza un valor demasiado grande la pieza que se desea mantener levitando caerá debido a que el sistema no conmutó lo suficientemente rápido, caso contrario, si la conmutación es muy rápida traerá los problemas ya mencionados anteriormente.

Sensado de corriente

Como se puede ver en el diagrama en bloques anterior, para poder realizar la comparación es necesario realimentar la corriente del electroimán para que el comprador pueda actuar en función al error (Ie).

Para hacer esto se debe utilizar un sensor que permita una medición de corriente mayor a 31 amperes y con un error máximo tolerable tal que permita actuar al comparador y no afecte al sistema. Además lo de la conmutación q dijo el gusti…

Los sensores de corriente trabajan en transresistencia, por lo tanto, se agrega el bloque H en el lazo de realimentación figura XXDD. Ahora ViL=iL*rm. De esta forma ahora se está trabajando con tensiones por lo tanto la corriente de referencia se afecta por el bloque Kin para obtener su equivalente en tensión. El bloque resultante es el siguiente:

BLOQUE CON KIN Y H. CAMBIAR Iref por VREF y IL por ViL y ie por Ve.

Se puede expresar al error que introduce el sensor de efecto hall como VerrHall. Entonces la expresion del error queda: 

	Ve= Kin*Iref-H*iL+-VerrHall=deltaI+-VerrHall

De esta forma, tomando el  caso extremo Kin*Iref=H*iL el error obtenido Ve=+-VerrHall.
Es decir que el error es el propio error del sensor. Si se toma un ancho de histéresis mucho mayor al error máximo introducido por el sensor, este error no afectaría al sistema ya que seria despreciable a comparación del valor que deltaI debería alcanzar para que el comparador cambie de estado.

El error que introduce el sensor de efecto hall VerrHall debe ser mucho menor al ancho de histéresis del comparador.




\section{Elección de tensión de alimentación}
\colorbox{red}{Esto no iria acá, pero lo pongo para que ya quede y después lo movemos}

Para comenzar el diseño circuital es importante determinar cómo será la alimentación del controlador de corriente. Para definirla se tendrá en cuenta la velocidad de respuesta de la planta.... y nose que mas

analizando la forma de corriente de la FIGURA 3.6, se observan tres secciones de tiempo diferenciadas: desde t= 0 hasta t1, desde t1 hasta t2,  y desde t2 hasta t3. \colorbox{red}{marcar en la imagen}

Se puede encontrar la expresión teórica de la corriente para cada uno de esos instantes de tiempo... Se utiliza la formula:

\begin{equation}
	I_L(t)=\frac{V}{R_L} + (I_o-\frac{V}{R_L})*e^{-\frac{t}{\tau}}
\end{equation} 

Donde:
\begin{itemize}
	\item V es la tensión de alimentación, que puede tomar valores $+V_{cc}$ y $-V_{cc}$.
	\item $I_o$ es la corriente inicial en cada tramo de la imagen.
	\item $\tau$ es la constante de tiempo del electroimán.
\end{itemize}

Como el ancho del ripple de corriente es fijo y conocido (SE SUPONE QUE YA LO DEFINIMOS???) nos interesa NOSE QUE NOS INTERESA...



Para encontrar expresiones del tiempo que tarda en realizarse cada tramo, se reemplaza $I_o$ por la correspondiente y se obtiene el tiempo que tarda en llegar a la corriente final...


\begin{equation}
		T1=-\tau*ln(\frac{V_{cc}-R*I_{max}}{V_{cc}})
\end{equation}


\colorbox{red}{Con esta} podemos ver cuánto tardaríamos de pasar de corriente 0 a corriente máxima, y podemos decir que queremos tardar menos de cierto tiempo.


Por ejemplo usando la ecuación que calcula el tiempo (T) que tarda en desplazarse un objeto en caida libre una altura $\Delta Y$:

\begin{equation}
	T=\sqrt{\frac{2*\Delta Y}{g}}
\end{equation}

Considerando $\Delta Y=1.5\:mm$, se llega a que $T=17.5\:ms$. Por lo tanto se desea que $T1<T$.
finalmente se obtiene: $Vcc=20.9\:V$.



----------------------------------------------------------------
SEPARADOR
------------------------------------------------


Otra forma de calcular la tensión de alimentación que se necesita sería considerando que se desea que los tiempos de crecimiento sean lo mas parecidos posible a los de decrecimiento. Esta diferencia se hace mas pronunciada a medida que aumenta el valor medio de la corriente del electroimán.

\begin{equation}
	T2-T1=-\tau*ln(\frac{I_{min}+\frac{V_{cc}}{R}}{I_{max}+\frac{V_{cc}}{R}})
\end{equation}

\begin{equation}
	T3-T2=-\tau*ln(\frac{I_{max}-\frac{V_{cc}}{R}}{I_{min}-\frac{V_{cc}}{R}})
\end{equation}


Podemos poner todo en función de $I_{min}$ utilizando la relación $I_{max}=I_{min}+\delta I_L$

\begin{equation}
	T2-T1=-\tau*ln(\frac{1}{1+\frac{\Delta I_L}{I_{min}+\frac{V_{cc}}{R}}})
\end{equation}

\begin{equation}
	T3-T2=-\tau*ln(1+\frac{\Delta I_L}{I_{min}-\frac{V_{cc}}{R}})
\end{equation}


Podemos decir que queremos que estos tiempos sean lo mas parecidos posible, dentro de todo el rango de trabajo de la corriente. Para ello consideramos que T3 no debe hacerse mucho mayor que T2 en el caso de que la corriente sea la máxima (30 A). Por elegir un número, considero que T3 debe ser menor a $2*T2$ \colorbox{red}{tuve que elegir que sea menor al doble así me quedaba bien el} cruce de las figuras.  planteamos una inecuación y la resolvemos con un grafico:

\begin{figure}[H]
	\centering
	\includegraphics[width=\textwidth]{tiempos_conmutacion_vs_fuente.png}
	\caption{Tiempos de conmutación vs tensión de alimentación.}
	\label{fig:img_tiempos_conmutacion}
\end{figure}

El punto en que ambas figuras se cruzan es con $Vcc=18\:V$, por lo tanto se eligen una tensión de alimentación de $24\:V$.

 