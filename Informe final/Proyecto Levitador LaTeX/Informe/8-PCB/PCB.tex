\chapter{Diseño del circuito impreso} \chapterlabel{Informe/8-PCB} \label{cap:PCB}

%En este capítulo se mencionan los criterios tenidos en cuenta para el diseño del circuito impreso. Además, se muestran los esquemáticos y las imágenes del PCB desarrollados para la implementación de la placa de control mediante el uso del software Altium Designer. 

%Por otra parte, se describen las consideraciones y circuitos utilizados para las fuentes de alimentación del sistema.

En este capítulo se describen las consideraciones y circuitos utilizados para las fuentes de alimentación del sistema. Ademas se mencionan cuales son los conectores, puntos de prueba y jumpers que posee el PCB.

Luego se mencionan los criterios tenidos en cuenta para el diseño del circuito impreso. Además, se muestran los esquemáticos y las imágenes del PCB desarrollados para la implementación de la placa de control mediante el uso del software Altium Designer.


\section{Fuentes de alimentación}

\subsection{Fuente de alimentación externa de 24 V}

La fuente externa se encarga de alimentar todo el circuito. Debe ser capaz de suministrar $24\:V$ y hasta $30\:A$. Para ello se puede utilizar una fuente de laboratorio o baterías. Esta fuente se conecta a la placa de control mediante una bornera dispuesta en el borde del PCB. Además, se agrega un fusible para proteger la placa ante sobre-consumo y un diodo para evitar que se dañe cuando se ingresa con polaridad invertida.

\subsection{Fuente de alimentación interna de 12 V}

La fuente de $12\:V$ se encarga de alimentar al regulador de $5\:V$ y al \textsl{driver} del puente H. Debido a los bajos consumos de potencia y bajo costo, se utiliza una fuente lineal. Por lo tanto, se decide utilizar el integrado L78M12CDT-TR.


\subsection{Fuente de alimentación interna de 5 V}

La fuente de $5\:V$ se encarga de alimentar los operacionales, el sensor de efecto Hall, el inversor y el regulador de tensión de $2.5\:V$. Debido a los bajos consumos de potencia y bajo costo, se utiliza una fuente lineal. Por lo tanto, se decide utilizar el integrado L78M05CDT.

\section{Conexión entre placa de control y microcontrolador}

Se utiliza un conector tipo DB9 hembra como vía de conexión para las distintas salidas y entradas digitales. Además, en la placa se dispone de un led que se enciende cuando  se detecta una correcta conexión con el microcontrolador.

\section{Puntos de prueba y jumpers}

Se agregan puntos de prueba sobre el PCB para poder medir señales de interés como las de control de los gates de los MOSFET del puente H, la salida del comparador con histéresis, la salida del derivador en la etapa del estimador, entre otros. Cada punto de prueba está referenciado en los esquemáticos como TP.

Por otro lado, se agregan jumpers con el objetivo de hacer mediciones a cada etapa por separado. Cada uno de ellos está referenciado en los esquemáticos como P.
 

\section{Esquemáticos}

\subsection{Principal}
\begin{figure}[H]
	\centering
	\includegraphics[height=0.85 \textheight]{main.png}
	%\caption{Diagrama en bloques de la implementación digital.}
	\label{fig:main}
\end{figure}

\subsection{Controlador de corriente}
\begin{figure}[H]
	\centering
	\includegraphics[height=0.85 \textheight]{contCorrEsq.png}
	%\caption{Diagrama en bloques de la implementación digital.}
	\label{fig:contCorrEsq}
\end{figure}

\subsection{Puente H}
\begin{figure}[H]
	\centering
	\includegraphics[height=0.85 \textheight]{PuenteH.png}
	%\caption{Diagrama en bloques de la implementación digital.}
	\label{fig:PuenteH}
\end{figure}

\subsection{Compensador analógico}
\begin{figure}[H]
	\centering
	\includegraphics[height=0.85 \textheight]{CompAnalog.png}
	%\caption{Diagrama en bloques de la implementación digital.}
	\label{fig:CompAnalog}
\end{figure}

\subsection{Estimador analógico}
\begin{figure}[H]
	\centering
	\includegraphics[height=0.85 \textheight]{EstimAnalog.png}
	%\caption{Diagrama en bloques de la implementación digital.}
	\label{fig:EstimAnalog}
\end{figure}

\subsection{Interfaz con microcontrolador}
\begin{figure}[H]
	\centering
	\includegraphics[height=0.85 \textheight]{InterfazMic.png}
	%\caption{Diagrama en bloques de la implementación digital.}
	\label{fig:InterfazMic}
\end{figure}

\subsection{Fuentes de alimentación}
\begin{figure}[H]
	\centering
	\includegraphics[height=0.85 \textheight]{FuentesAliment.png}
	%\caption{Diagrama en bloques de la implementación digital.}
	\label{fig:FuentesAliment}
\end{figure}

\section{PCB}
\subsection{Consideraciones para el diseño del PCB}
Para el diseño del PCB se tuvieron en cuenta los siguientes aspectos:

\begin{itemize}
	\item Se realizó el diseño en una placa de 2 capas para reducir el costo de fabricación.
	\item Se utilizó un plano de masa para la etapa de potencia y otro para el resto del circuito. Se intentó reducir al mínimo las conexiones entre ambos planos con el objetivo de disminuir el ruido inducido por la etapa de potencia en el resto del circuito.
	\item Se mantuvo separada la etapa de potencia con las de pequeña señal.
	\item Se intentó reducir al mínimo la longitud de las pistas de las señales críticas como la tensión de salida del sensor de efecto Hall.
	\item Se utilizaron pistas gruesas y polígonos para la etapa de potencia que deben soportar la circulación de corrientes elevadas.
	\item Se quitó la capa \textsl{Top Solder} en la etapa de potencia correspondiente al puente H para mejorar la disipación.
	\item Se utilizaron vias stitching en los planos de masa sobre la placa para mejorar la aislación entre componentes y reducir el acople por campos electromagnéticos.
	\item Se ``acostaron'' los MOSFET sobre la placa para ayudar a su disipación de potencia.
	\item Para optimizar el tamaño de la placa, se rutearon las pistas de manera organizada. Para ello, se intentó que las pistas verticales se realicen en la capa superior mientras que las horizontales, en la inferior.
	\item El proyecto se dividió en 7 esquemáticos. Cada uno de ellos se asocia a una etapa en específico (compensación analógica, controlador de corriente, estimador analógico, etc). 
	\item  Los componentes de cada esquemático tiene asociado un rango de valores determinados para los \textsl{designators}. Es decir, todos los componentes con \textsl{designators} entre 0 y 99 pertenecen al primer esquemático, los de 100 a 199 al segundo, y así sucesivamente.  De esta forma, con solo ver el \textsl{designators} en el PCB es posible determinar, fácilmente, a qué etapa pertenece.  
	\item Se siguieron las reglas de diseño estándar de PCBWay.
	
\end{itemize}



 






\subsection{Modelo 2D}
\subsubsection{Vista superior}
\begin{figure}[H]
	\centering
	\includegraphics[scale=0.2]{VistSup.png}
	%\caption{Diagrama en bloques de la implementación digital.}
	\label{fig:VistSup}
\end{figure}

\subsubsection{Vista inferior}
\begin{figure}[H]
	\centering
	\includegraphics[scale=0.2]{VistInf.png}
	%\caption{Diagrama en bloques de la implementación digital.}
	\label{fig:VistInf}
\end{figure}

\subsection{Modelo 3D}
\subsubsection{Vista superior}
\begin{figure}[H]
	\centering
	\includegraphics[scale=0.3]{VistSup3D.png}
	%\caption{Diagrama en bloques de la implementación digital.}
	\label{fig:VistSup3D}
\end{figure}

\subsubsection{Vista inferior}
\begin{figure}[H]
	\centering
	\includegraphics[scale=0.3]{VistInf3D.png}
	%\caption{Diagrama en bloques de la implementación digital.}
	\label{fig:VistInf3D}
\end{figure}

