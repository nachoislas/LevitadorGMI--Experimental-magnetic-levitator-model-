\chapter{Primer prototipo} \chapterlabel{Primer_prototipo} \label{cap:prototipo1}
\section{Introducción}\label{sec:introProto1}

Como se mencionó previamente, el proyecto se desarrolla en forma incremental. El objetivo es desarrollar núcleos pequeños que, una vez completados, evolucionan mediante el agregado de nuevas funcionalidades y el mejoramiento de aquellas que ya poseía.

Se realizan dos prototipos. El primero de ellos se enfoca en lograr la lectura de la patente, la detección de la presencia del vehículo y la manipulación de una barrera, considerando una sola vía de acceso.

Una vez que este primer módulo se pone a punto, se mejoran algunas de las funcionalidades que ya posee y se le añaden nuevas, dando lugar al segundo prototipo. 


\section{Composición del sistema}\label{sec:compSist}

\hl{Sacar alguna foto de la fuente y despu\'es otra de las protos para poner ac\'a}

En este primer prototipo, los periféricos considerados son el detector vehicular magnético, tanto de la entrada como de la salida, una barrera y una cámara IP. Además, se cuenta con la Placa I/O desarrollada para controlar dichos periféricos y la Unidad Central de Control (UCC), que obtiene el número de la patente del vehículo detectado por la cámara.

Dado que se trata de una etapa temprana del proyecto, se decidió simular los detectores y las barreras debido a su elevado precio de compra. Posteriormente, la empresa interesada en el proyecto nos proporcionó los detectores. Las barreras, por su parte, se mantienen simuladas a lo largo de todo el proyecto. A futuro, las mismas serán desarrolladas o adquiridas por la empresa.

En el circuito desarrollado, cada detector fue implementado a partir de un pulsador conectado a un pin digital del microcontrolador. Cuando estos son pulsados, dichos pines reciben una señal en alto, simulando la activación de los detectores. Esto se desarrolló de esta manera debido a que los sensores magnéticos poseen un contacto seco de un relé que permite interrumpir o no el circuito que alimenta los pines digitales del microcontrolador con los que se desea determinar el estado de estos dispositivos. Entonces, la interrupción de los mismos hace que a dichos pines llegue o no la señal en alto que indica la activación.

En cuanto a la barrera, la misma fue implementada mediante dos diodos emisores de luz o leds. Cuando la barrera está abajo, el led de posición baja se encuentra encendido y el otro apagado. Cuando la misma se encuentra levantada, ocurre lo inverso, encendiéndose el otro led.

\textcolor{mPurple}{Por otra parte, como se explicó en el \textcolor{mGreen}{capítulo x (patentes)}, para lograr la detección de las patentes, se utiliza el sistema openalpr, en su modo de funcionamiento alprd, el cual permite poder trabajar con vídeos en tiempo real en lugar de imágenes fijas. De esta forma se obtiene el resultado a partir del vídeo generado por la cámara IP utilizada, la cual se encuentra conectada mediante WiFi al mismo router al que se conecta la UCC.}

\textcolor{mPurple}{Además, la cámara IP se encuentra continuamente filmando. Cada vez que el sistema detecta una patente en el video, la misma es guardada en una cola o “queue”. Posteriormente, se extrae la información de la misma.}

\hl{Begin: Lo marcado entre renglones resaltados ya lo puse en cap 4}

La Placa I/O fue inicialmente implementada con una placa Arduino Mega. Esto se debió a dos cuestiones. Primero, ante nuestra falta de conocimiento, tanto para la elección de microcontroladores como para su uso, docentes del Laboratorio de Comunicaciones (LAC), en el cual se está desarrollando el proyecto, consideraron que era una buena opción para empezar a incursionar en este tema. Esto se debe a que hay una gran cantidad de información acerca del tema, tanto en libros como en la web, y de forma gratuita. Segundo, se determinó que el microprocesador ATmega2560 que esta placa posee permite implementar el sistema planteado.

El uso de esta placa de desarrollo permitió probar el funcionamiento del código realizado. Posteriormente, se desarrolló el circuito correspondiente en protoboards utilizando el microcontrolador ATmega328P. Esto se debió a que, a pesar de haber estado trabajando con el microcontrolador de la placa que es superior, este es adecuado para las funciones que debe desarrollar en el prototipo que resulta de este proyecto. 
 
\hl{End: Lo marcado entre renglones resaltados ya lo puse en cap 4}

Adicionalmente, dado que se planteó que la Placa I/O se comunique en forma inalámbrica con la UCC, se utilizó el módulo WiFi Serial ESP8266, modelo ESP-01. Inicialmente, se lo intentó configurar mediante el envío de comandos AT desde el microcontrolador. Como esto no prosperó, se decidió crear códigos independientes para el módulo y el microcontrolador. Esto permitió la comunicación entre el microcontrolador y la UCC, siendo el módulo WiFi un intermediario que recibe y transmite información. Dichos códigos se encuentran explicados más adelante, en la \textcolor{mGreen}{sección X (“Códigos desarrollados”)}.

Entre el ATmega328P y el módulo la comunicación se encuentra implementada mediante puerto serie. Luego, este último se comunica con la Unidad Central de Control a través de un socket TCP/IP.

Finalmente, la UCC está implementada mediante una computadora. En la misma se encuentra el código encargado de comunicarse con la Placa I/O y de procesar el video que le transmite la cámara para obtener la información de las patentes. La interconexión entre el módulo y la UCC, y la cámara IP y la UCC se desarrolla a través de un router. 

\section{Funcionamiento del sistema}\label{sec:funcSist}

En esta sección se detalla la forma en que funciona este primer prototipo. 

\textcolor{mPurple}{En primer lugar se lleva a cabo una etapa de conexión entre el módulo WiFi, la Placa I/O y la UCC. En está, al encender el sistema la Placa I/O pone en alto el pin de reset del módulo WiFi produciendo el arranque de este. Una vez encendido, el módulo procede a conectarse al router y luego crear un servidor web en su dirección IP.}

\textcolor{mPurple}{Una vez realizadas estas acciones, ambos dispositivos quedan a la espera de que la UCC se conecte al servidor creado. Por esto, la primer acción que debe realizar la UCC es crear un socket TCP/IP,  el cual luego utiliza para conectarse al servidor web creado por el módulo WiFi. De esta manera, se concluye la etapa de conexión y el sistema queda listo para funcionar en modo continuo.}

Cuando se presiona alguno de los dos pulsadores, es decir que un vehículo es detectado mediante alguno de los sensores magnéticos, el microcontrolador de la Placa I/O levanta dicha información y se la envía por puerto serie al Módulo WiFi. Este último, comunica esta información a la UCC vía Wi-Fi, únicamente si algún sensor se encuentra activado. Entonces, cuando a la UCC le llega esta información, a partir del procesamiento de los trabajos levantados por el sistema a partir del video provisto por la cámara IP, se obtiene el número de la patente del vehículo. 

\hl{Poner im\'agenes de lo q se ve en la pantalla y podr\'ia ser de los Jobs tambi\'en.}

En esta etapa, la activación de cualquiera de los pulsadores por separado o de ambos simultáneamente desemboca en la misma condición, la activación del sistema de reconocimiento de patentes sin distinción entre entrada o salida. Esto se debe a que aún no se desarrolló esta distinción, debido a que el principal objetivo de este primer prototipo es la verificación del correcto funcionamiento del reconocimiento.

Luego de que se obtuvo la patente, la Unidad Central se la envía a la Placa I/O, de manera de indicarle que levante la barrera.

\section{Herramientas de software utilizadas}\label{sec:herrSoft}

Además del OpenALPR, explicado en el \textcolor{mGreen}{capítulo X}, se utilizaron las siguientes herramientas de software para la puesta en funcionamiento de este primer prototipo.

\subsection{Arduino IDE}

Tirar algo de info.

\subsection{Eclipse}

Tirar algo de info.


\hl{?`Beanstalk?: ?`ya est\'a explicado en el cap de las Patentes? Si es as\'i, ac\'a se podr\'ia mencionar su uso por lo menos.}

\hl{?`Jsoncons?: aun no est\'a pero imagino que va a ir en el cap de Patentes. Si es as\'i, ac\'a se podr\'ia mencionar su uso por lo menos.}

\hl{Ver si falta alguna m\'as (librer\'ias de OpenCV en eclipse x ej, etc)}

\section{Códigos desarrollados}\label{sec:codDes}

\hl{?Armar ac\'a diagramas en bloque del funcionamiento de los 3 c\'odigos para prototipo 1?}

El del eclipse y los 2 de Arduino, y de qué se encarga cada uno.

Lo de los “job” y eso explicarlo acá, en vez de en “Funcionamiento del sistema”.

\hl{AGREGAR REFERENCIAS: a las p\'ags de los software x ejemplo}

No olvidar:
\begin{itemize}
	\item Detectar patente
	\item Intercomunicación placa i/o, ucc y cámara con ucc
	\item Queue con trabajos, se van borrando los viejos
	\item Código de leer estado de periféricos
	\item Decir que no se plantea en paralelo en análisis de entrada y salida
	\item Se considera que existe tiempo suficiente entre auto y auto como para hacer todo el procesamiento de la información (no hay superposición de tareas de 2 autos en el mismo acceso: entrada o salida)
\end{itemize}










