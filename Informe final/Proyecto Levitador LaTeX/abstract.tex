%% !TeX spellcheck = en_GB
%
%Multiphase power converters have become a very attractive solution for the current control on high power applications, due to the power dissipation reduction in the active and passive components, and the improvement in the total ripple characteristics.
%
%These applications, which commonly arise in the high-energy physics field, require the control of high current (on the order of kiloamperes) with high precision (on the order of several hundreds of ppm). Additionally, large variations in the current reference and perturbations in the input or output voltages are present. These variations generate transient errors that have to be corrected in several hundreds of microseconds. Given the current level and the technological limitations of the semiconductor devices, the maximum allowable switching frequency implies transient times on the order of a few switching periods. Under these conditions, several requirements arise that cannot be satisfied by the existent current control techniques, due to their limited transient response, or sensitivity to the converter operating point or parasitic elements.
%
%According to these problematics, the present thesis studies the current control in multiphase power converters, and proposes solutions to achieve the transient and precision requirements above mentioned.
%
%The first proposal is the current ripple characterization, as a function of the differences across each phase inductance value. This characterization allows to find expressions for the total ripple that, unlike the methodologies present in the literature, are valid for any number of phases and duty cycle. These expressions allow to calculate others magnitudes, such as peak value, RMS value and the harmonic content of the total current ripple.
%
%The proposed characterization allows to develop a mitigation methodology that significantly improves the total current ripple in practical conditions, with respect to the only methodology present in the literature based on the same principle. This improvement is achieved without significantly increasing the calculation complexity.
%
%Finally, a current control for multiphase power converters is proposed. The proposed control is able to precisely track the current reference, and achieve transient times on the order of a few switching periods, under changes on the current reference or perturbations in the converter input or output voltages.
%
%The proposals have been validated by means of simulations or experimental measurements on a low-scale prototype.


