\chapter{Introducción} \chapterlabel{intro} \label{cap:intro}


A lo largo de los últimos años, el parque automotor de la República Argentina creció a grandes pasos. Solo en seis años, entre 2011 y 2017, aumentó en un 30\% \cite{Noticia1}. Cada vez hay más autos circulando por las calles y los espacios disponibles para estacionar sobre las mismas se están agotando. A pesar de ello, los autos se siguen fabricando y patentando. Es por esto que las playas de estacionamiento cobran fuerza.

Por otra parte, en nuestro país, el mercado inmobiliario es uno de los mejores refugios de capital, y las cocheras resultan una gran alternativa para quienes desean invertir\cite{Noticia2}.

Debe destacarse, además, la falta de avances tecnológicos \textcolor{mPurple}{(o algún sinónimo de “automatización”)} en los establecimientos de la ciudad y también a nivel país. Se hace hincapié en esto debido a que la tendencia actual en el plano internacional es automatizar los espacios de estacionamiento, ya sea en aeropuertos, cocheras, centros de compras, edificios privados, etc \cite{kinouwell}.

Es por estas razones que se propone desarrollar un Sistema Automatizado de Estacionamiento (SAE) destinado en primera instancia a controlar y gestionar el ingreso, egreso y estadía de vehículos en playas de estacionamiento, pero que a futuro podrá aplicarse a cocheras de edificios comerciales y privados, control de ingreso y egreso a barrios cerrados, etc. 

El proyecto es impulsado por una empresa privada de la ciudad de Mar del Plata quien, en base a una investigación realizada, determinó que los sistemas implementados actualmente en más del 80\% de las playas de estacionamiento del país son muy elementales, y requieren de la intervención de un operador para registrar en forma manual el ingreso y egreso de los vehículos. Esto genera una serie de vulnerabilidades al momento de asegurar el control total de las transacciones.

El objetivo de este trabajo es el desarrollo de un proyecto que cumpla con los requerimientos planteados por la empresa dentro del marco de un proyecto final de carrera de la Universidad Nacional de Mar del Plata.

Las condiciones que debe cumplir el prototipo inicial a diseñar y posteriormente desarrollar quedan impuestas por el escenario en el que se lo va a aplicar. Inicialmente se considera un estacionamiento techado, por lo que se estima que la iluminación es siempre la misma (artificial). Además, el establecimiento cuenta con una cabina para el operador, en la cual se encuentra el equipo encargado de recopilar la información enviada desde los distintos componentes del sistema, para la posterior toma de decisiones. 

Para la planificación y desarrollo del proyecto consideramos apropiada la aplicación de metodologías ágiles de trabajo \cite{Bahit}.  

Estas técnicas se basan en un desarrollo iterativo e incremental, con iteraciones cortas y entregas de prototipos al cliente en forma periódica. La gran interacción con el cliente es una de las ventajas de trabajar de esta forma, permitiendo refinar el producto a partir de la realimentación de los mismos. Debido a que el proyecto es de interés para una empresa privada, se propuso implementar este tipo de metodología de trabajo ya que posibilita obtener un prototipo que cumpla en forma adecuada todos los requerimientos planteados por la misma.

Asimismo, otra ventaja de estas técnicas es la flexibilidad ante el cambio. Es posible realizar modificaciones en diferentes etapas a lo largo del proyecto sin la necesidad de volver a empezar desde el inicio. Esto se debe a que se trabaja sobre núcleos pequeños que, una vez que fueron completados, evolucionan mediante el agregado de nuevas funcionalidades.

Entre las metodologías ágiles existentes, se ha seleccionado para aplicar el método Kanban \cite{Bahit} \cite{ApunteWingu}. Esto se debe a que el mismo, además de las ventajas ya mencionadas, permite una óptima planificación de las tareas a desarrollar y una mejor visualización del trabajo en proceso, pendiente y realizado. Esta técnica favorece una organización mucho más eficiente facilitando el seguimiento detallado de cada etapa del proyecto tanto a nivel individual como grupal y favoreciendo el trabajo en equipo con fuerte interacción entre los distintos actores.

En cuanto a la organización del trabajo se presentan dos núcleos principales: el reconocimiento automático de patentes de vehículos, conocido como ALPR por su nombre en inglés \textit{Automatic License Plates Recognition}, y el diseño y construcción de la placa principal que permitirá administrar a nivel de hardware todos los componentes del sistema.

El capítulo 2 describe la estructura y funcionamiento del sistema completo a partir del cual se desarrollará el prototipo. El capítulo 3 presenta los aspectos vinculados a ALPR desde los sistemas existentes, su estructura general, etapas y características, hasta la evaluación, elección y puesta a punto del sistema a utilizar. El capítulo 4 abarca la descripción de los periféricos a utilizar, los componentes principales de la placa principal del sistema y, el diseño y la construcción de la misma. El capítulo 5…  \textcolor{mPurple}{(COMPLETAR CON TODOS LOS CAPÍTULOS QUE HAYA)}.








%Como se mencionó previamente, el proyecto se desarrolla en forma incremental. Por lo tanto, se realizan dos prototipos. El primero de ellos se enfocará en lograr la lectura de la patente con un alto porcentaje de efectividad, en diseñar y construir una primera versión de la placa encargada de administrar los periféricos en la zona de ingreso y egreso (Placa I/O), y en lograr la comunicación de la misma con la Unidad Central de Control.

%Una vez que este primer módulo se pone a punto, se busca mejorar algunas de las funcionalidades que ya posee y añadirle nuevas, dando lugar al segundo prototipo. Se plantea que el mismo cuente con la detección del tipo de vehículo (motocicleta, automóvil o camioneta) que desea ingresar al establecimiento, el desarrollo y la interacción con una base de datos sencilla, y el diseño y construcción de la Placa I/O definitiva.


