% Generated by GrindEQ Word-to-LaTeX 
\documentclass{article} % use \documentstyle for old LaTeX compilers

\usepackage[utf8]{inputenc} % 'cp1252'-Western, 'cp1251'-Cyrillic, etc.
\usepackage[english]{babel} % 'french', 'german', 'spanish', 'danish', etc.
\usepackage{amsmath}
\usepackage{amssymb}
\usepackage{txfonts}
\usepackage{mathdots}
\usepackage[classicReIm]{kpfonts}
\usepackage{graphicx}

% You can include more LaTeX packages here 


\begin{document}

%\selectlanguage{english} % remove comment delimiter ('%') and select language if required


\noindent 

\noindent 

\noindent UNMDP-FI

\noindent  

\noindent  

\noindent  

\noindent Carrera: Ingenier\'{i}a Electr\'{o}nica

\noindent  

\noindent  

\noindent Proyecto: Levitador GMI

\noindent  

\noindent Especificaci\'{o}n T\'{e}cnica

\noindent  

\noindent Levitador GMI

\noindent  

\noindent  

\noindent  

\noindent  

\noindent  

\noindent  

\noindent  

\noindent  

\noindent  

\noindent  

\noindent  

\noindent 

\noindent 

\noindent 

\noindent 

\noindent 

\noindent 

\noindent 

\noindent 

\noindent 

\noindent 

\noindent 

\noindent 

\noindent 

\noindent 

\noindent 

\noindent 

\noindent 

\noindent 

\noindent 

\noindent 

\noindent 

\noindent Versi\'{o}n: 1.0      Autores: Juan Manuel Guariste 

         Javier Mosconi

         Ignacio Islas

\begin{tabular}{|p{1.1in}|p{1.1in}|p{1.1in}|p{1.3in}|} \hline 
Fecha & Versi\'{o}n & Descripci\'{o}n & Autor/a \\ \hline 
04/04/2022 & 1.0 & Primera versi\'{o}n del documento; se describe la especificaci\'{o}n t\'{e}cnica de los requerimientos. & Juan Manuel Guariste, Javier Mosconi,\newline Ignacio Islas \\ \hline 
\end{tabular}

\textbf{}

\noindent \eject 

\noindent 

\noindent 1. Introducci\'{o}n 61.1. Prop\'{o}sito del documento 61.2. Alcance 61.3. Personal involucrado 71.4. Definiciones, acr\'{o}nimos y abreviaturas 81.5. Referencias 91.6. Estructura del Proyecto 92. Descripci\'{o}n del dispositivo 93. Caracterizaci\'{o}n del Electroim\'{a}n 113.1. C\'{a}lculo de las dimensiones del electroim\'{a}n 113.2. C\'{a}lculo del bobinado 123.3. C\'{a}lculo de la inductancia y resistencia interna del electroim\'{a}n 133.4. Linealizaci\'{o}n de la expresi\'{o}n de la inductancia 143.5. Modelo de estado de la planta 154. Implementaci\'{o}n anal\'{o}gica 184.1. Dise\~{n}o y modelado del Controlador de Corriente 184.1.1. Caracter\'{i}sticas del sistema 184.1.2. Circuito del controlador de corriente 194.1.2.1. Simulaciones de formas de onda 214.1.2.2. Simulaci\'{o}n de un escal\'{o}n en la referencia de corriente 214.1.2.3. Implementaci\'{o}n circuital del puente H 224.1.2.3.1. Descripci\'{o}n general de la topolog\'{i}a 224.1.2.3.2. Dimensionamiento de capacitor de bootstrap 254.1.2.3.3. Resistencia entre gate y source 274.1.2.3.4. Protecci\'{o}n del gate 274.1.2.3.5. Tiempo muerto 284.1.2.4. Dimensionamiento de los capacitores de fuente 284.1.2.5. Conmutaci\'{o}n de alta frecuencia para el bootstrap 304.1.2.6. Simulaci\'{o}n del sistema con oscilador auxiliar 314.1.3. Caracter\'{i}sticas est\'{a}ticas y din\'{a}micas del controlador 324.1.3.1. Corriente media del electroim\'{a}n 324.1.3.2. Frecuencia de conmutaci\'{o}n de la corriente 324.1.3.3. Ancho de banda del controlador 324.1.4. Transferencia lineal del controlador de corriente 334.2. Dise\~{n}o y modelado del Estimador Analogico 344.2.1. An\'{a}lisis de la estimaci\'{o}n 344.2.2. Modelo circuital del estimador de posici\'{o}n 364.2.3. Circuito del derivador compensado 374.2.4. Dise\~{n}o del LPF 404.2.5. Compensaci\'{o}n I*R 414.2.6. Rectificador, Restador y Filtrado 454.2.6.1. Rectificador 454.2.6.2. Restador 464.2.6.3. Etapa de filtrado 474.2.7. Circuito completo 484.2.8. Simulaci\'{o}n de estimador completo 484.2.9. Transferencia final del estimador de posici\'{o}n: 494.3. Dise\~{n}o del Compensador Analogico 514.3.1. Dise\~{n}o de compensador por adelanto de fase 514.3.2. Dise\~{n}o circuital 524.3.3. Compensador con integrador 534.3.3.1. Implementaci\'{o}n circuital del integrador 534.3.3.2. C\'{a}lculo de ganancia de entrada 544.3.3.3. Implementaci\'{o}n circuital del bloque de ganancia de entrada ``F'' 555. Implementaci\'{o}n digital 575.1. Descripci\'{o}n general 575.2. Determinaci\'{o}n de la frecuencia de muestreo 585.3. Adquisici\'{o}n y procesamiento de las muestras 595.4. Estimaci\'{o}n digital de la posici\'{o}n 605.5. Resoluci\'{o}n en posici\'{o}n 625.6. Acondicionamiento de se\~{n}ales para el ADC 645.6.1. Referencia de posici\'{o}n 645.6.2. Componente  alterna de corriente del electroim\'{a}n 655.6.3. Componente  continua de corriente del electroim\'{a}n 655.7. Acondicionamiento de se\~{n}ales para el DAC 665.8. Transferencias de la planta y del controlador de corriente 675.9. Dise\~{n}o de Compensador 675.9.1. An\'{a}lisis de estabilidad con masa de 30 Kg 675.9.2. An\'{a}lisis de estabilidad con masa de 1 Kg 735.10.  Dise\~{n}o de lazo de realimentaci\'{o}n externo 755.11.  C\'{a}lculo de los coeficientes del controlador 785.12.  Conexi\'{o}n entre el PCB y el microcontrolador 796. Fuentes de Alimentaci\'{o}n 806.1. Fuente de alimentaci\'{o}n externa de 24V. 806.2 Fuente de alimentaci\'{o}n interna de 12V 806.3 Fuente de alimentaci\'{o}n interna de 5V 807. Bibliograf\'{i}a 818. Anexo 828.1  Esquem\'{a}ticos 828.1.1 Principal 828.1.2. Controlador de corriente 838.1.3. Puente H 848.1.4 Compensador anal\'{o}gico 858.1.5 Estimador anal\'{o}gico 868.1.6 Interfaz con microcontrolador 878.1.7 Fuentes de alimentaci\'{o}n 888.2 PCB 898.2.1 Modelo 2D 8911.2.1.1 Vista superior 898.2.1.2 Vista inferior 908.2.2 Modelo 3D 918.2.2.1 Vista superior 918.2.2.2 Vista inferior 928.3 Lista de Materiales 93

\noindent 

\noindent 

\noindent 
\section{\eject }

\noindent 
\section{1. Introducci\'{o}n}

\noindent Este documento corresponde a la Especificaci\'{o}n T\'{e}cnica para el producto Levitador GMI. Esta especificaci\'{o}n se ha estructurado bas\'{a}ndose en la informaci\'{o}n mencionada en los documentos de Especificaci\'{o}n de Requerimientos (ER) y Especificaci\'{o}n Funcional (EF) del proyecto.

\noindent 
\subsection{1.1. Prop\'{o}sito del documento}

\noindent El presente documento tiene como prop\'{o}sito proveer informaci\'{o}n detallada sobre c\'{o}mo se dise\~{n}ar\'{a}n y comportar\'{a}n las distintas partes que componen el producto, para que en base a esta informaci\'{o}n pueda ser construido.

\noindent 

\noindent 
\subsection{1.2. Alcance}

\noindent El alcance de este documento es detallar la implementaci\'{o}n t\'{e}cnica de las soluciones propuestas en cada EF definida en el documento ``Especificaci\'{o}n funcional'' del producto Levitador GMI.

\noindent 

\noindent 

\noindent \eject 

\noindent 
\subsection{1.3. Personal involucrado}

\noindent 

\begin{tabular}{|p{1.3in}|p{2.4in}|} \hline 
Nombre &  Juan Manuel Guariste \\ \hline 
Rol &  Analista funcional, t\'{e}cnico, desarrollador,    tester \\ \hline 
Categor\'{i}a Profesional &  Estudiante de Ingenier\'{i}a Electr\'{o}nica \\ \hline 
Responsabilidad &  Dise\~{n}o, construcci\'{o}n y testeo del producto \\ \hline 
Informaci\'{o}n de contacto &  juanmaguariste@gmail.com \\ \hline 
\end{tabular}

\textbf{}

\noindent 

\noindent 

\begin{tabular}{|p{1.3in}|p{2.4in}|} \hline 
Nombre &  Javier Mosconi \\ \hline 
Rol &  Analista funcional, t\'{e}cnico, desarrollador, tester \\ \hline 
Categor\'{i}a Profesional &  Estudiante de Ingenier\'{i}a Electr\'{o}nica \\ \hline 
Responsabilidad &  Dise\~{n}o, construcci\'{o}n y testeo del producto \\ \hline 
Informaci\'{o}n de contacto &  javi.mosconi@gmail.com \\ \hline 
\end{tabular}

\textbf{}

\noindent 

\noindent 

\begin{tabular}{|p{1.3in}|p{2.4in}|} \hline 
Nombre &  Ignacio Islas \\ \hline 
Rol &  Analista funcional, t\'{e}cnico, desarrollador,   tester \\ \hline 
Categor\'{i}a Profesional &  Estudiante de Ingenier\'{i}a Electr\'{o}nica \\ \hline 
Responsabilidad &  Dise\~{n}o, construcci\'{o}n y testeo del producto \\ \hline 
Informaci\'{o}n de contacto &  nacho.th.21@gmail.com \\ \hline 
\end{tabular}


\subsection{}

\noindent 

\noindent 

\noindent 

\noindent 

\noindent 

\noindent 

\noindent 

\noindent 

\noindent 
\subsection{1.4. Definiciones, acr\'{o}nimos y abreviaturas}

\noindent 

\begin{tabular}{|p{2.0in}|p{2.3in}|} \hline 
Usuarios & Personas que usar\'{a}n el sistema \\ \hline 
PCB & Placa de circuito Impreso \\ \hline 
Electroim\'{a}n & Tipo de im\'{a}n en el que el campo magn\'{e}tico se produce mediante el flujo de una corriente el\'{e}ctrica. \\ \hline 
uC  & Microcontrolador \\ \hline 
Sistema de control & Sistema capaz de controlar variables de salida en funci\'{o}n de variables de entrada, y de la misma salida.  \\ \hline 
Controlador o compensador anal\'{o}gico & Circuito electr\'{o}nico compuesto por amplificadores operacionales cuya funci\'{o}n es constituir un sistema de control. \\ \hline 
Controlador o compensador digital & Un sistema de control implementado en un microcontrolador. \\ \hline 
Switch & Llave selectora. \\ \hline 
Kg & Abreviatura de Kilogramo, unidad de medida de masa. \\ \hline 
mm & Abreviatura de mil\'{i}metro, unidad de medida de distancia. \\ \hline 
A & Abreviatura de Amper, unidad de medida de magnitud de corriente el\'{e}ctrica. \\ \hline 
DAC & Conversor digital/anal\'{o}gico. \\ \hline 
ER & Especificaci\'{o}n de requerimiento \\ \hline 
RF & Requerimiento funcional \\ \hline 
RNF & Requerimiento no funcional \\ \hline 
USB & Bus serie universal; protocolo de comunicaci\'{o}n \\ \hline 
PC & Computadora \\ \hline 
ADC & Conversor anal\'{o}gico/digital. \\ \hline 
EF & Especificaci\'{o}n funcional \\ \hline 
\end{tabular}


\subsection{1.5. Referencias}

\noindent 

\begin{tabular}{|p{4.3in}|} \hline 
T\'{i}tulo del Documento \\ \hline 
Especificaci\'{o}n de requerimientos - Levitador GMI \\ \hline 
Especificaci\'{o}n Funcional - Levitador GMI \\ \hline 
\end{tabular}

\textbf{}

\noindent 
\subsection{1.6. Estructura del Proyecto}

\noindent Este documento aborda cada etapa que compone al sistema haciendo referencia a cada uno de los requerimientos especificados.

\noindent 
\section{2. Descripci\'{o}n del dispositivo}

\noindent 

\noindent El Levitador GMI es un dispositivo capaz de mantener un objeto en suspensi\'{o}n  mediante una fuerza electromagn\'{e}tica generada por un electroim\'{a}n a una distancia $Y_0$ variable entre 3 mm y 5 mm. La distancia de separaci\'{o}n puede ser configurada por el usuario y el peso del objeto debe ser menor a 30 kg. 

\noindent 

\noindent El producto consta de 2 partes principales: un electroim\'{a}n y una placa de control. El electroim\'{a}n consiste en dos piezas formadas por l\'{a}minas de acero: una con forma de ``E'' que tiene un cable bobinado en su n\'{u}cleo, y otra con forma de ``I'' que es atra\'{i}da por la pieza en forma de ``E'' por medio de una fuerza electromagn\'{e}tica. Esto deja un espacio o ``\textit{gap}'' de aire entre ambas de longitud $Y_0$. Por otro lado, de la pieza en forma de ``I'' se puede colgar el objeto que se desea levantar.

\noindent 

\noindent El control de la fuerza electromagn\'{e}tica es realizado por una placa de control con el objetivo de mantener fija la distancia $Y_0$, a pesar de las perturbaciones externas que el sistema pueda recibir. 

\noindent 

\noindent Es importante aclarar que este sistema s\'{o}lo puede ejercer fuerza verticalmente, por lo tanto no puede controlar la posici\'{o}n horizontal.

\noindent 

\noindent El sistema est\'{a} conformado por los bloques que se muestran en la \textbf{F}igura \textbf{2.1}. Se utilizan dos controladores distintos: uno anal\'{o}gico y otro digital. Cada uno de ellos se compone de un compensador y un estimador de posici\'{o}n.  El usuario decidir\'{a} cual de estas implementaciones ejercer\'{a} el control mediante la utilizaci\'{o}n de un \textit{switch}, por lo que solo una estar\'{a} activa al mismo tiempo. El sistema anal\'{o}gico est\'{a} formado por un conjunto de componentes pasivos y amplificadores operacionales, mientras que el digital est\'{a} basado en un microcontrolador re-programable. Adem\'{a}s, el estimador de posici\'{o}n se encarga de  entregar una tensi\'{o}n proporcional al \textit{gap }de aire real en funci\'{o}n de  la corriente que circula por el electroim\'{a}n.

\noindent 

\noindent \includegraphics*[width=6.25in, height=2.93in]{image1}

\noindent \textbf{Figura 2.1. }Diagrama en bloques del sistema.

\noindent 

\noindent El usuario puede modificar el gap de aire que desea mediante un potenci\'{o}metro presente en la placa de control, que entrega una tensi\'{o}n proporcional a la misma. Tanto la implementaci\'{o}n anal\'{o}gica como la digital reciben como entrada esta tensi\'{o}n. Luego, es comparada con la estimaci\'{o}n y se utiliza como entrada para el compensador.

\noindent 

\noindent La funci\'{o}n del compensador es garantizar la estabilidad del sistema. Esto lo logra al modificar la referencia del controlador de corriente mediante una acci\'{o}n de control. El controlador de corriente se encarga de proveer corriente al electroim\'{a}n de forma tal que le permita generar la fuerza electromagn\'{e}tica necesaria para mantener el gap de aire. 

\noindent 

\noindent Por otra parte, se utiliza un software de PC para modificar los coeficientes de la implementaci\'{o}n digital.

\noindent \eject 

\noindent 
\section{3. Caracterizaci\'{o}n del Electroim\'{a}n}

\begin{tabular}{|p{2.1in}|p{2.1in}|} \hline 
Requisito relacionado & Descripci\'{o}n \\ \hline 
RF01 & Mantener un objeto en suspensi\'{o}n \\ \hline 
\end{tabular}

\textbf{Tabla 3.1. }RF

\noindent 
\subsection{3.1. C\'{a}lculo de las dimensiones del electroim\'{a}n}

\noindent En la Figura \textbf{3.1} se puede observar una representaci\'{o}n f\'{i}sica del problema.

\noindent \includegraphics*[width=5.31in, height=2.87in, trim=0.95in 0.00in 0.00in 0.00in]{image2}

\noindent \textbf{Figura 3.1. }Representaci\'{o}n f\'{i}sica del problema.\textbf{ }

\noindent 

\noindent 

\noindent A partir del modelado f\'{i}sico del electroim\'{a}n se llega a la expresi\'{o}n de la inductancia (L) en funci\'{o}n del gap de aire (Y) (ecuaci\'{o}n 3.1) y de la fuerza magn\'{e}tica ejercida por el electroim\'{a}n (Fm) (ecuaci\'{o}n 3.2):
\begin{equation} \label{GrindEQ__3_1_} 
L(Y)\simeq \frac{N^2*A*{\mu }_0}{2*Y} 
\end{equation} 
\begin{equation} \label{GrindEQ__3_2_} 
\left|Fm\right|=\frac{i^2*N^2*{\mu }_0*A}{4*y^2} 
\end{equation} 
Seg\'{u}n se observa en la \textbf{Figura 3.1, }la fuerza magn\'{e}tica tiene sentido contrario a la fuerza que ejerce la gravedad sobre la pieza con forma de I. Para que el sistema alcance el equilibrio, estas fuerzas deben igualarse en m\'{o}dulo. Por lo tanto, igualando la ecuaci\'{o}n 3.2 y  el peso (m*g), se obtiene:

\noindent 
\[\left|Fm\right|=m*g\ =>\frac{i^2*N^2*{\mu }_0*A}{4*y^2}=m*g\ \] 
Se despeja el producto N*I:    

\noindent 
\begin{equation} \label{GrindEQ__3_3_} 
N*I=y*\sqrt{\frac{4*m*g}{{\mu }_0*A}} 
\end{equation} 


\noindent Se desea obtener una estimaci\'{o}n del tama\~{n}o de electroim\'{a}n que se debe utilizar. Este se define por el \'{a}rea del n\'{u}cleo.

\noindent 

\noindent Se tiene una relaci\'{o}n entre la inductancia del bobinado, la cantidad de vueltas, la corriente, la densidad de campo magn\'{e}tico y el \'{a}rea transversal del electroim\'{a}n (considerando el caso de m\'{a}xima densidad de campo magn\'{e}tico) mediante:

\noindent 
\begin{equation} \label{GrindEQ__3_4_} 
L*Imax\ =\ N*A*Bmax 
\end{equation} 


\noindent Utilizando las ecuaciones 3.1 y 3.4:

\noindent 

\begin{tabular}{|p{3.9in}|p{0.4in}|} \hline 
$Bmax=\mu 0*\frac{N*Imax}{2*y}$ & \eqref{GrindEQ__3_5_}  \\ \hline 
\end{tabular}



\noindent Sustituyendo 3.3 en 3.5:

\noindent 

\begin{tabular}{|p{3.9in}|p{0.4in}|} \hline 
$Bmax=\sqrt{\frac{m*g*{\mu }_0}{A}}$ & \eqref{GrindEQ__3_6_}  \\ \hline 
\end{tabular}

Resultando

\begin{tabular}{|p{3.9in}|p{0.4in}|} \hline 
$A\ge \frac{m*g*{\mu }_0}{{Bmax}^2}$ & \eqref{GrindEQ__3_7_}  \\ \hline 
\end{tabular}

Para evitar saturar el n\'{u}cleo del electroim\'{a}n cuando  por \'{e}l circule la m\'{a}xima corriente, se debe elegir uno con un \'{a}rea transversal que cumpla con la condici\'{o}n que impone la expresi\'{o}n  3.7. Considerando que se utiliza acero de silicio, se adopta una densidad de campo magn\'{e}tico m\'{a}xima  ${(B}_{MAX})$ de 0.7 Tesla. Por lo tanto, se obtiene que el \'{a}rea debe ser mayor a $20\ {cm}^2$.\textbf{}

\noindent 
\subsection{3.2. C\'{a}lculo del bobinado}

\noindent 

\noindent Se desea dimensionar la cantidad de vueltas del bobinado y la corriente requerida para las condiciones del problema. Partiendo de la ecuaci\'{o}n 3.4, y considerando un \'{a}rea transversal $A\approx 25\ {cm}^2$  y las condiciones m\'{a}s exigentes, con Y= 5 mm y m=30 kg, se obtiene:

\noindent 

\begin{tabular}{|p{3.9in}|p{0.4in}|} \hline 
$N*i_{Lmx}=3060,69$ & \eqref{GrindEQ__3_8_}  \\ \hline 
\end{tabular}



\noindent Debido a que ya se dispone de un electroim\'{a}n con N = 150, se impone una corriente m\'{a}xima de 20,4 A. Para poder tener un margen en la masa m\'{a}xima soportada, se adopta una corriente m\'{a}xima de 21 A, resultando un N*I = 3150.

\noindent 

\noindent Se dispone de un electroim\'{a}n  de laminaci\'{o}n normalizada sin desperdicio \#600. Est\'{a} compuesto por dos piezas: una con forma de ``E'' y otra con forma de ``I''. En su n\'{u}cleo tiene un bobinado de 150 vueltas (N) de cobre esmaltado con un di\'{a}metro de 2.5 mm. El n\'{u}cleo es de secci\'{o}n cuadrada ya que esto maximiza el \'{a}rea mientras que disminuye el per\'{i}metro, reduciendo as\'{i} la longitud media de las espiras y ahorrando material. 

\noindent 

\noindent \includegraphics*[width=4.10in, height=2.10in, trim=0.00in 0.25in 0.00in 0.00in]{image3}

\noindent \textbf{Figura 3.2. }Dimensiones del electroim\'{a}n\textbf{}

 

\noindent El electroim\'{a}n utilizado presenta las siguientes caracter\'{i}sticas:

\begin{enumerate}
\item  $a\ =\ 50mm$

\item  $A\approx 25\ {cm}^2$

\item  ${\textrm{\'{A}}rea\ de\ ventana:\ A}_w=0.75{*a}^2$

\item  $Factor\ de\ ocupaci\textrm{\'{o}}n\ de\ ventana:\ K_w=\ 0.6$

\item  $Longitud\ de\ espira\ media:\ MLT\ =\ 6*a\ =\ 300mm$

\item  $Densidad\ de\ corriente\ resultante:\ J=\ \frac{N*i_{Lmx}}{K_w*A_w}\approx 2.8\ [\frac{A}{{mm}^2}]$
\end{enumerate}

\noindent 
\subsection{3.3. C\'{a}lculo de la inductancia y resistencia interna del electroim\'{a}n}

\noindent 

\noindent Con las dimensiones del electroim\'{a}n y con las expresiones 3.1 y 3.3 se realiz\'{o} el siguiente cuadro:

\noindent 

\noindent 

\noindent 

\noindent 

\noindent 

\noindent 

\noindent 

\begin{tabular}{|p{0.6in}|p{0.8in}|p{0.9in}|p{1.9in}|} \hline 
Y [mm] & Masa [kg] \newline Con i=21A & Corriente [A] \newline con [m=30 kg] & Inductancia [mHy] Te\'{o}rica, sin considerar inductancia de dispersi\'{o}n.\newline   \\ \hline 
2 & 198.8 & 8.16 & 17.7 \\ \hline 
3 & 88.4 & 12.23 & 11.8 \\ \hline 
4 & 49.7 & 16.31 & 8.84 \\ \hline 
5 & 31.8 & 20.39 & 7.07 \\ \hline 
6 & 22.09 & 24.4 & 5.89 \\ \hline 
\end{tabular}

\textbf{Tabla 3.2. }Inductancia en funci\'{o}n de la distancia

\noindent \textbf{}

 \includegraphics*[width=3.30in, height=2.34in, trim=0.00in 0.00in 0.00in 0.19in]{image4}   

\noindent \textbf{Figura 3.3. }Inductancia (ordenadas) en funci\'{o}n de la distancia (abscisas)\textbf{.}

\noindent 

\noindent Para calcular la resistencia del bobinado se utiliza la f\'{o}rmula 3.9:

\noindent 

\noindent $Rcu\ =\frac{4*\rho *N*MLT}{\pi *{\phi }^2}=$ 0.156 ohm         \eqref{GrindEQ__3_9_}     

\noindent 

\noindent El di\'{a}metro del alambre es $\phi $ = 2.5 mm y MLT = 300 mm.

\noindent Utilizando un alambre de cobre de di\'{a}metro 2.5 mm se obtiene una resistencia aproximada de 0.156 Ohm. Lo que se traduce en p\'{e}rdidas de calor iguales a  68.796 W.  

\noindent 
\subsection{3.4. Linealizaci\'{o}n de la expresi\'{o}n de la inductancia}

\noindent 

\noindent Se realiza una expansi\'{o}n por serie de Taylor de la ecuaci\'{o}n 3.1, despreciando los t\'{e}rminos de orden mayor o igual a 2, con el objetivo de obtener una expresi\'{o}n lineal para la inductancia. Se llega al resultado de:

\noindent 

\begin{tabular}{|p{3.9in}|p{0.4in}|} \hline 
$L(Y)[mH]\simeq \ -2.2089\ *\ Y\ \ +\ 17.67$ & \eqref{GrindEQ__3_10_}  \\ \hline 
\end{tabular}



\noindent En la ecuaci\'{o}n 3.10 la distancia Y puede ser expresada directamente en mm y el resultado se obtiene en mH.

\noindent \includegraphics*[width=6.23in, height=3.28in]{image5}

\noindent \textbf{Figura 3.4.}Inductancia real en funci\'{o}n de la posici\'{o}n (azul). Inductancia linealizada entre 2 y 5 mm (naranja).\textbf{ }

\noindent 
\subsection{3.5. Modelo de estado de la planta}

\noindent De la modelizaci\'{o}n del problema f\'{i}sico tenemos que la din\'{a}mica del problema se describe como se muestra en la ecuaci\'{o}n 3.11, en la que solo se consideran desplazamientos verticales:
\[\textrm{⅀}{F=\textrm{⅀}F}_y+{\textrm{⅀}F}_x\] 

\[{\textrm{⅀}F}_x=0\] 


\begin{tabular}{|p{3.9in}|p{0.4in}|} \hline 
$\textrm{⅀}F_y=m*a=>\ m*g-F_m=m*\textrm{\"{y}}$ & \eqref{GrindEQ__3_11_}  \\ \hline 
\end{tabular}



\noindent Siendo Fm la fuerza que ejerce el electroim\'{a}n. Utilizando la ecuaci\'{o}n 3.2 se obtiene que:

\noindent 

\begin{tabular}{|p{3.9in}|p{0.4in}|} \hline 
$\left|F_m\right|=\frac{{[i(t)]}^2*N^2*{\mu }_0*A}{4*[{y(t)]}^2}=\frac{N^2*{\mu }_0*A}{4}*(\frac{i(t)}{{y(t)}})^2$ & \eqref{GrindEQ__3_12_}  \\ \hline 
\end{tabular}



\noindent Por lo tanto, utilizando la ecuaci\'{o}n 3.11 es posible llegar a:

\noindent 

\begin{tabular}{|p{3.9in}|p{0.4in}|} \hline 
$m*\textrm{\"{y}}\ =\ m*g-K\ *\ (\frac{i(t)}{y(t)})^2$ & \eqref{GrindEQ__3_13_}  \\ \hline 
\end{tabular}



\noindent Donde K es una constante de valor:

\noindent 

\begin{tabular}{|p{3.9in}|p{0.4in}|} \hline 
$K\ =\ \frac{N^2*{\mu }_0*A}{4}$ &  \\ \hline 
\end{tabular}



\noindent Teniendo en cuenta que $i(t)$ es la entrada a la planta, se pueden plantear las variables de estado:

\noindent 

\begin{tabular}{|p{3.9in}|p{0.4in}|} \hline 
$X_1(t)\ =\ y(t)$ &   \\ \hline 
\end{tabular}



\begin{tabular}{|p{3.9in}|p{0.4in}|} \hline 
$X_2(t)\ =\ \frac{dX_1(t)}{dt}={\textrm{ẋ}}_1$ &   \\ \hline 
\end{tabular}



\begin{tabular}{|p{3.9in}|p{0.4in}|} \hline 
$u(t)\ =i(t)$  &   \\ \hline 
\end{tabular}



\noindent 

\noindent Por lo tanto se obtienen las siguientes ecuaciones de estado:

\noindent 

\noindent 

\begin{tabular}{|p{3.9in}|p{0.4in}|} \hline 
${\textrm{ẋ}}_1=X_2(t)$ & \eqref{GrindEQ__3_14_}  \\ \hline 
\end{tabular}



\begin{tabular}{|p{3.9in}|p{0.4in}|} \hline 
${\textrm{ẋ}}_2=g-\frac{K}{m}*(\frac{u}{X_1(t)})^2$ & \eqref{GrindEQ__3_15_}  \\ \hline 
\end{tabular}



\noindent El modelo es alineal, por lo tanto se linealizan las ecuaciones 3.14 y 3.15 mediante la serie de Taylor. Para ello, primero se encuentran los puntos de equilibrio del modelo. Para la variable X1, por las condiciones del problema se define como X1o = Yo = 4mm. El resto de los puntos de equilibrio se encuentran igualando las derivadas a cero, estos son:

\noindent 

\begin{tabular}{|p{3.9in}|p{0.5in}|} \hline 
$X_{10}=y_0=4\ mm$ &  \\ \hline 
\end{tabular}



\begin{tabular}{|p{3.9in}|p{0.4in}|} \hline 
$X_{20}=0$ &   \\ \hline 
\end{tabular}



\begin{tabular}{|p{3.9in}|p{0.5in}|} \hline 
$uo\ =\ \sqrt{\frac{m*g}{K}}*X_{10}$ &   \\ \hline 
\end{tabular}



\noindent Por lo tanto el modelo de estado linealizado queda:

 

\begin{tabular}{|p{3.9in}|p{0.4in}|} \hline 
${{\textrm{ẋ}}_1}^*=X^{\ \ *}_2\ $ & \eqref{GrindEQ__3_16_}  \\ \hline 
\end{tabular}



\begin{tabular}{|p{3.9in}|p{0.5in}|} \hline 
${{\textrm{ẋ}}_2}^*=2*\frac{K}{m}*\frac{uo^2}{X^{\ \ 3}_{10}}*{{X_1}^*}-2*\frac{K}{m}*\frac{uo}{X^{\ 2}_{10}}*u^*$ & \eqref{GrindEQ__3_17_}  \\ \hline 
\end{tabular}

Las matrices del modelo son:

\noindent 

\noindent \includegraphics*[width=1.72in, height=1.00in, trim=0.22in 0.13in 0.04in 0.19in]{image6}     \includegraphics*[width=1.84in, height=1.10in]{image7}
\[C=[1\ \ \ 0]\ \] 


\noindent Por lo tanto es posible obtener la funci\'{o}n transferencia de la planta:

\noindent 

\begin{tabular}{|p{3.9in}|p{0.4in}|} \hline 
$G_{Planta}\ =\ C{*(SI-A)}^{-1}*B\ $ & \eqref{GrindEQ__3_18_}  \\ \hline 
\end{tabular}



\noindent Finalmente se obtiene, para m = 30 kg e Yo = 4mm:

\noindent 

\begin{tabular}{|p{3.9in}|p{0.4in}|} \hline 
$G_{Planta}\ =\ -\ \frac{\frac{2}{yo}*\sqrt{\frac{K*g}{m}}}{S^2-\frac{2*g}{Yo}}=\frac{-1.201}{S^2-\ 4900}$ & \eqref{GrindEQ__3_19_}  \\ \hline 
\end{tabular}



\noindent La planta tiene dos polos, que coinciden con los autovalores en $\pm \sqrt{\frac{2*g}{Yo}}=\ \pm 70\ rad/s\ \ @\ Y_0=4mm$.  Es decir uno en el semiplano derecho y otro en el izquierdo. Debido a esto la planta es inestable y se debe buscar una estrategia de control apropiada.

\noindent 

\noindent 
\section{4. Implementaci\'{o}n anal\'{o}gica}

\begin{tabular}{|p{2.1in}|p{2.1in}|} \hline 
Requisito relacionado & Descripci\'{o}n \\ \hline 
RF01 & Mantener un objeto en suspensi\'{o}n \\ \hline 
RF02 & Implementar un controlador digital y anal\'{o}gico \\ \hline 
RF 05:  & Variar la distancia de separaci\'{o}n $Y_0$\newline  \\ \hline 
\end{tabular}

\textbf{Tabla 4.1. }RF.

\noindent 
\subsection{4.1. Dise\~{n}o y modelado del Controlador de Corriente }

\noindent Para regular la fuerza ejercida por el electroim\'{a}n es necesario controlar la corriente que circula por \'{e}l. Para ello, se modela a la planta como la impedancia de un inductor con una resistencia serie, cuya inductancia var\'{i}a con el \textit{gap} de aire: 

\noindent 

\begin{tabular}{|p{3.9in}|p{0.4in}|} \hline 
$\frac{1}{sL(y)\ +\ Rl}$\textbf{} &  \\ \hline 
\end{tabular}



\noindent Para realizar este control se utiliza un sistema realimentado, como el que se muestra en la \textbf{Figura 4.1}. Se puede ver que se ingresa con una tensi\'{o}n de referencia (Vin) proporcional a la corriente de salida deseada, que luego se multiplica por la ganancia de entrada (Kin). La corriente del electroim\'{a}n se realimenta en forma de una tensi\'{o}n proporcional a ella (ViF). Ambas tensiones son restadas y el resultado (e) ingresa al bloque de comparador con hist\'{e}resis, que act\'{u}a en conmutaci\'{o}n, por lo que su salida tiene dos estados posibles: $\pm V_L$. Al ser aplicadas al inductor se producir\'{a} una rampa de corriente: si la tensi\'{o}n es positiva, la rampa crece, y si es negativa decrece. De esta forma, debido a la conmutaci\'{o}n del comparador se obtiene, a la salida, una forma de onda triangular $I_L$, cuyo valor medio es la corriente deseada y se corresponde a la tensi\'{o}n de referencia\includegraphics*[width=6.23in, height=1.83in]{image8}

\noindent \textbf{Figura 4.1}. Diagrama en bloques simplificado del controlador de corriente.

\noindent 
\paragraph{4.1.1. Caracter\'{i}sticas del sistema}

\begin{enumerate}
\item  Para sensar la corriente se utiliza un sensor de efecto Hall HO 15-NP, con una transconductancia de H(s) = $53.3\ mV/A$.

\item  Para la ganancia de entrada Kin se utiliza un valor de $0.32$ puesto que Vin var\'{i}a entre $0\ V$ y $5\ V$ y debe mapearse con una corriente variable entre $0\ A$ y $30\ A$.

\item  Se adopta una variaci\'{o}n de la corriente en torno a su valor medio (ripple) de $500\ mA$, por lo que resulta en un ancho de hist\'{e}resis de $26.665\ mV$.

\item  Seg\'{u}n mediciones realizadas sobre el electroim\'{a}n, la inductancia en el punto de equilibrio $y_0=4mm$ es de $16.44\ mHy$ (considerando la inductancia de dispersi\'{o}n de$8.89\ mHy$) y la resistencia serie es de $0.2\ \mathit{\Omega}$.

\item  La tensi\'{o}n aplicada sobre el electroim\'{a}n es $+24\ V$ para el estado ON y $-24\ V$ para el estado OFF.

\item  Se utiliza un driver de corriente que trabaja en conmutaci\'{o}n mediante un puente H con 4 N-MOS. 
\end{enumerate}

\noindent 
\paragraph{4.1.2. Circuito del controlador de corriente}

\noindent Se comienza planteando la etapa de entrada que consiste en la ganancia de entrada y el restador con la realimentaci\'{o}n. El objetivo es imponer una ganancia de entrada de 0.32, y que la salida de esta etapa tenga un punto de operaci\'{o}n de 2.5V para poder utilizar una fuente de alimentaci\'{o}n entre 0 y 5 V para los operacionales. Para lograr esto se utiliza un circuito como el que se muestra en la \textbf{Figura 4.2}.

\noindent 

\noindent \includegraphics*[width=3.05in, height=3.11in]{image9}

\noindent \textbf{Figura 4.2}. Etapa de entrada.

\noindent 

\noindent Para la implementaci\'{o}n del comparador con hist\'{e}resis se utiliza un amplificador operacional realimentado positivamente. Se implementa un ancho de hist\'{e}resis de $26.665\ mV$,  alrededor de un punto de operaci\'{o}n de 2.5 V, como se muestra en la \textbf{Figura 4.3}.

\noindent 

\noindent \includegraphics*[width=2.23in, height=2.20in]{image10}

\noindent \textbf{Figura 4.3}. Comparador con hist\'{e}resis.

\noindent 

\noindent Para controlar la corriente en el electroim\'{a}n se utiliza una topolog\'{i}a en puente H, que permite conmutar la polaridad de la tensi\'{o}n aplicada a la bobina. Para medir la corriente se utiliza un sensor de efecto Hall, que es modelado en la simulaci\'{o}n como  una fuente de tensi\'{o}n controlada por corriente, con una ganancia de 53.3 mV/A correspondiente a su transconductancia. Esta implementaci\'{o}n puede observarse en la \textbf{Figura 4.4a}. Luego, su salida es realimentada a la etapa de entrada luego de restarle la tensi\'{o}n de referencia  $V_{bias}$ de 2.5V, como se muestra en la \textbf{Figura 4.4b}. 

\noindent 

\noindent \includegraphics*[width=5.37in, height=3.59in]{image11}

\noindent \textbf{Figura 4.4a. }Puente H y sensor de efecto Hall

\noindent .

\noindent \includegraphics*[width=2.95in, height=1.98in]{image12}

\noindent \textbf{Figura 4.4b. }Resta del Vbias al sensor de efecto Hall

\noindent 

\noindent 

\noindent 
\subparagraph{4.1.2.1. Simulaciones de formas de onda}

\noindent 

\noindent \includegraphics*[width=6.23in, height=3.54in]{image13}

\noindent \textbf{Figura 4.5}. Formas de onda de corriente en el electroim\'{a}n y salida del comparador.

\noindent 

\noindent En la \textbf{Figura 4.5} se pueden observar dos formas de onda. La inferior (violeta) se corresponde con la salida del comparador con hist\'{e}resis, que conmuta. La onda triangular (verde) es la corriente en el electroim\'{a}n. Para la simulaci\'{o}n se utiliz\'{o} una tensi\'{o}n de referencia de entrada de 1 V, por lo tanto el valor medio de la corriente en la salida es 6 A con un \textit{ripple }de 500 mA. Esto fue verificado en la simulaci\'{o}n mediante cursores.

\noindent 

\noindent 
\subparagraph{4.1.2.2. Simulaci\'{o}n de un escal\'{o}n en la referencia de corriente}

\noindent En la \textbf{Figura 4.6} se muestra c\'{o}mo cambia la corriente en el electroim\'{a}n al aplicarle a la entrada del controlador un escal\'{o}n de tensi\'{o}n entre 1 y 3 V. Se puede observar c\'{o}mo la conmutaci\'{o}n del comparador se detiene para ajustar la corriente con la referencia.

\noindent 

\noindent \includegraphics*[width=6.19in, height=3.02in]{image14}

\noindent \textbf{Figura 4.6}. Respuesta al escal\'{o}n del circuito.

\noindent 
\subparagraph{4.1.2.3. Implementaci\'{o}n circuital del puente H}

\noindent La corriente que se desea controlar es la que circula por el electroim\'{a}n y, debido a que el sistema va a trabajar con corrientes elevadas, es importante que la implementaci\'{o}n del controlador de corriente sea eficiente. Por lo tanto, para disminuir la disipaci\'{o}n de potencia del circuito se utiliza un controlador que funciona en conmutaci\'{o}n. 

\noindent 
{\bf 4.1.2.3.1. Descripci\'{o}n general de la topolog\'{i}a}

\noindent Para lograr una corriente cont\'{i}nua en el electroim\'{a}n utilizando una fuente conmutada se debe alternar la polaridad de la tensi\'{o}n aplicada en los bornes del inductor. Al hacer esto, la corriente crece y decrece (seg\'{u}n la polaridad) con forma exponencial debido a la resistencia interna del electroim\'{a}n. Sin embargo, como el intervalo de tiempo que se mantiene la fuente en positivo o negativo es peque\~{n}o comparado con la constante de tiempo de la planta, el incremento de corriente ser\'{a} peque\~{n}o y puede ser aproximado a una recta. Por lo tanto se obtiene una corriente cont\'{i}nua (valor medio) con un ripple superpuesto de forma triangular. 

\noindent 

\noindent Para lograr alternar la polaridad de la fuente sobre el inductor se utiliza una topolog\'{i}a en puente H con 4 MOSFET que funcionan con un ciclo de trabajo determinado (manejado por el controlador por hist\'{e}resis) como se observa en la \textbf{Figura 4.7}. Pueden diferenciarse dos semiciclos de trabajo: uno de estado ON y otro de estado OFF. El estado ON se define como el semiciclo durante el cual la corriente en el inductor crece (pendiente positiva), mientras que el estado OFF se da cuando la corriente decrece.

\noindent 

\noindent \includegraphics*[width=4.02in, height=3.15in]{image15}

\noindent \textbf{Figura 4.7}. Topolog\'{i}a elemental del puente H.

\noindent 

\noindent El electroim\'{a}n se conecta entre los puntos medios de cada par de transistores. De esta manera se puede conmutar la polaridad de la tensi\'{o}n que se le aplica. S\'{o}lo se permite que dos transistores se enciendan a la vez, y esto se realiza de manera diagonal. Es decir, en la \textbf{Figura 4.7}, Q1 y Q4 pueden estar encendidos, mientras que Q3 y Q2 est\'{a}n apagados, y viceversa. Es necesario evitar que se enciendan Q1 y Q2 a la vez, o Q3 y Q4, ya que ocurrir\'{i}a un cortocircuito entre la fuente de alimentaci\'{o}n y GND, lo que producir\'{i}a una circulaci\'{o}n de corriente denominada shoot-through. 

\noindent 

\noindent Los 4 MOSFET utilizados para el puente H son de tipo N (pues es complicado conseguir un MOS tipo P de potencia adecuado). Para que estos puedan funcionar correctamente en conmutaci\'{o}n es necesario que en el estado ON, la diferencia de tensi\'{o}n entre gate y source sea mayor o igual a 7V. Esto no es un problema para los dos MOS inferiores del puente H (Q2 y Q4), ya que la tensi\'{o}n en source est\'{a} fijada en GND y el driver puede aplicar 12V al gate (superando los 7V entre gate y source). El problema radica en los transistores superiores del puente H, ya que la tensi\'{o}n en source var\'{i}a entre 0 V y 24 V, por lo que en el gate deber\'{i}a haber, por lo menos, 31V con respecto a GND. Sin embargo, la tensi\'{o}n m\'{a}xima disponible entregada por la fuente es de 24V. Para resolver este problema se utiliza un driver flotante con bootstrap.

\noindent 

\noindent Para controlar la conmutaci\'{o}n se utiliza un mosfet driver HIP4081A que se encarga de encender y apagar los transistores seg\'{u}n las entradas de control. Adem\'{a}s permite la configuraci\'{o}n de un tiempo muerto para evitar que se enciendan dos transistores de un lado a la vez. Tambi\'{e}n provee la circuiter\'{i}a necesaria para implementar la fuente flotante que enciende los mosfet del lado superior para lo cual solo se debe agregar un diodo y un capacitor de manera externa. Para la implementaci\'{o}n circuital se van a utilizar los MOSFET IPB160N04.

\noindent 

\noindent En la \textbf{Figura 4.8} se observa solo una de las mitades del puente H (lado A)  junto con las se\~{n}ales de control provistas por el driver HIP4081A. El an\'{a}lisis para la otra mitad es an\'{a}logo, por lo que se evita por simplicidad. La implementaci\'{o}n del driver bootstrap permite obtener en el gate del MOS superior, una tensi\'{o}n de 36V respecto a GND, logrando as\'{i} una diferencia de tensi\'{o}n mayor a 7V entre gate y source. 

\noindent 

\noindent \includegraphics*[width=3.52in, height=2.94in]{image16}

\noindent \textbf{Figura 4.8}. Configuraci\'{o}n Bootstrap simplificada.

\noindent 

\noindent El driver bootstrap consiste en un capacitor ($C_{BS}$), un diodo, y la circuiter\'{i}a interna del HIP4081A. Para garantizar el correcto funcionamiento del bootstrap, al encender el sistema, la secuencia de inicio del HIP4081A enciende las dos salidas de la parte inferior del puente H: ALO y BLO con el fin de encender Q2 y Q4 durante un tiempo que se conoce como periodo de refresco de bootstrap. De esta forma, los capacitores de bootstrap de ambos lados quedan conectados a GND y se pueden cargar completamente. Durante este tiempo, las salidas a los gates AHO y BHO se mantienen en bajo continuamente lo que asegura que no se produzca corriente de shoot-through durante el per\'{i}odo nominal de refresco del bootstrap. Al final de este per\'{i}odo las salidas responden normalmente al estado de las se\~{n}ales de entrada de control.

\noindent 

\noindent Para comprender su funcionamiento se har\'{a} un breve an\'{a}lisis del sistema. Para ello, se parte suponiendo que el sistema se encuentra funcionando: con el transistor Q2 encendido (ALO = Vcc), Q1 apagado (AHO = AHS = 0 V) y la corriente circulando de izquierda a derecha como lo indica la \textbf{Figura 4.8}. En ese caso, el capacitor $C_{BS}$ se carga a 12V, ya que en un terminal tiene la fuente de 12V (a trav\'{e}s del diodo $D_{BS}$) y el otro est\'{a} conectado a GND por medio de Q2.

\noindent 

\noindent Una vez que se apaga el transistor inferior, empieza a transcurrir el tiempo muerto. Teniendo en cuenta que la carga es inductiva, el valor medio de la corriente mantiene su sentido circulando por los diodos antiparalelos del MOS inferior del lado A y el superior del lado B. Esto provoca que el source del MOS superior del lado A tenga una tensi\'{o}n negativa igual a la ca\'{i}da de tensi\'{o}n en directa del diodo antiparalelo de Q2. 

\noindent 

\noindent Una vez finalizado el tiempo muerto, se enciende el MOS Q1. Para encenderlo, la se\~{n}al AHO se pone en nivel alto. Durante el tiempo que Q1 pasa de estar apagado a encendido, la tensi\'{o}n en el source cambia de -Vd a Vbus de manera gradual mientras se carga el gate, y AHO pasa a ser igual a AHB, que es igual a la tensi\'{o}n entregada por el capacitor de bootstrap sumada a la tensi\'{o}n en el source de Q1. De esta manera se logra una tensi\'{o}n de 36V con respecto a GND en el gate y genera una diferencia entre gate y source de 12V.

\noindent 

\noindent Para lograr un funcionamiento adecuado del Boostrap es necesario dimensionar correctamente al capacitor $C_{BS}$ con el fin de que pueda proveer la carga suficiente durante el tiempo en el que el MOS est\'{e} encendido. 

\noindent 

\noindent \includegraphics*[width=4.27in, height=3.29in]{image17}

\noindent \textbf{Figura 4.9. }Puente H.

\noindent 
{\bf 4.1.2.3.2. Dimensionamiento de capacitor de bootstrap}

\noindent Para el dimensionamiento se tuvieron en cuenta sugerencias y procedimientos descripto en [1] y [2].

\noindent 

\noindent Para encender un NMOS es necesario proveer corriente a su gate hasta cargar las capacidades par\'{a}sitas entre gate-source y gate-drain. Una vez cargadas, el MOS queda en estado encendido y no consume m\'{a}s corriente en el gate. En el caso de los MOS del lado superior, esta corriente proviene del capacitor de bootstrap. 

\noindent 

\noindent En la implementaci\'{o}n del puente H se decidi\'{o} colocar resistencias entre gate y source. Estas aparecen como R1, R2, R3 y R4 en la \textbf{Figura 4.9}. Debido a la diferencia de tensi\'{o}n entre gate-source, se genera una corriente constante en estas resistencias durante el tiempo que el MOS est\'{e} encendido, que tambi\'{e}n debe ser provista por el  bootstrap.

\noindent 

\noindent Por otro lado, el capacitor debe entregar corriente al diodo de bootstrap cuando este queda en inversa ($I_{DR}$), y tambi\'{e}n entregar una corriente de fuga al circuito integrado HIP ${(I}_{QBS})$.

\noindent Esta \'{u}ltima se desprecia ya que es compensada internamente por la bomba de carga del HIP.

\noindent 

\noindent Por lo tanto, para poder dimensionar correctamente el capacitor de bootstrap es necesario tener en cuenta todos estos efectos mencionados anteriormente. Para ello se parte planteando la carga que almacena el capacitor bootstrap. Esta se obtiene como:

\noindent 

\begin{tabular}{|p{3.9in}|p{0.4in}|} \hline 
$Q_B=C_B*V_B$ & \eqref{GrindEQ__4_1_} \\ \hline 
\end{tabular}



\noindent En la ecuaci\'{o}n 4.1\textbf{,} $Q_B$ es la carga total del capacitor de bootstrap, $C_B$ es la propia capacidad del capacitor, y $V_B$ es la diferencia de  tensi\'{o}n entre sus terminales. 

\noindent 

\noindent Para evitar sufrir una ca\'{i}da de tensi\'{o}n tal que afecte el encendido de los MOS, es necesario que $Q_B$ pueda abastecer tambi\'{e}n al gate, a la resistencia entre gate-source y al diodo en inversa. Por lo tanto:

\begin{tabular}{|p{3.9in}|p{0.4in}|} \hline 
$Q_B{\ >\ Q}_G\ +\ Q_{RR}\ +\ \frac{(I_{DR}+I_{GS})}{f_{PWM}}$ & \eqref{GrindEQ__4_2_}  \\ \hline 
\end{tabular}



\noindent Donde:

\begin{enumerate}
\item  $Q_G$= Carga total que se debe entregar al gate del MOS.

\item  $Q_{RR}$ = Carga entregada al diodo en inversa durante el tiempo de recovery (cuando pasa de modo conducci\'{o}n a inversa).

\item  $I_{DR}$ = Corriente de fuga del diodo en inversa

\item  $I_{GS}$ = Corriente que circula por la resistencia de gate-source

\item  $f_{PWM}=\ $frecuencia de conmutaci\'{o}n
\end{enumerate}

\noindent 

\noindent 

\noindent Por lo tanto, al reemplazar la ecuaci\'{o}n 4.1 en la 4.2 se obtiene:

\noindent 

\begin{tabular}{|p{3.9in}|p{0.4in}|} \hline 
$C_{BS}>\frac{Q_G+Q_{RR}+\frac{(I_{DR}+I_{GS})}{f_{PWM}}}{{\mathit{\Delta}V}_B}$ & \eqref{GrindEQ__4_3_} \\ \hline 
\end{tabular}



\noindent Seg\'{u}n la hoja de datos [5] del MOSFET IPB160N04, $Q_G=\ 170\ nC$. Por lo tanto, adoptando una ca\'{i}da de tensi\'{o}n tolerable en el capacitor de $\Delta$VB= 0.1V y considerando la informaci\'{o}n brindada por las hojas de datos, es posible dimensionar el capacitor para que este posea una carga suficiente para mantener siempre encendido al MOSFET utilizando la inecuaci\'{o}n 4.3.

\noindent 

\noindent Para el c\'{a}lculo de la carga de recuperaci\'{o}n $Q_{RR}$ se puede considerar que la forma de onda de la corriente de recuperaci\'{o}n es triangular. De esta forma,  $Q_{RR}$ es aproximadamente igual a la mitad del producto entre el pico de la magnitud de corriente inversa y la duraci\'{o}n del tiempo de recuperaci\'{o}n.  Debido a que se usa el diodo RSX205LAM30TR, se obtiene a partir de [3] que  $I_R$ es igual a 0.1A  y  el tiempo de recuperaci\'{o}n de inversi\'{o}n es de 12.5 ns. Por lo tanto, la carga de recuperaci\'{o}n resulta de 0.625 nC.

\noindent 

\noindent Para la corriente inversa de fuga del diodo de bootstrap se obtiene un valor de $I_{DR}=2\ mA\ (@\ T=75{}^\circ ,\ V_R=\ 24V)$.

\noindent 

\noindent La corriente $I_{GS}$ tiene forma exponencial pero se aproxima a una constante debido a que el intervalo de tiempo es peque\~{n}o. Por lo tanto, puede calcularse como la diferencia de tensi\'{o}n del capacitor de bootstrap (VB=12V) dividido el valor de la resistencia gate-source. Esta resistencia es de 4.7kOhm. Por lo tanto Igs=2.55mA. 

\noindent 

\noindent Debido a que el controlador por hist\'{e}resis no asegura que haya una conmutaci\'{o}n en un tiempo constante (como se observa en la \textbf{Figura 4.6}), se decidi\'{o} superponer una conmutaci\'{o}n auxiliar de 50 kHz. 

\noindent 

\begin{tabular}{|p{3.9in}|p{0.4in}|} \hline 
$Cbs>{{\frac{Q_G+Q_{RR}+\frac{(I_{DR}+I_{GS})}{f_{PWM}}}{{\mathit{\Delta}V}_B}}}>{\frac{170\ nC\ +\ 0.625\ nC\ +\frac{(2\ mA+2.55\ mA)}{50\ kHz}}{0.1V\ }}\gtrsim \ 2.61\ uF$ &  \\ \hline 
\end{tabular}

Por lo tanto, una capacidad mayor a 2.61 uF resultar\'{a} en una ca\'{i}da menor a 0.1 V en el capacitor de bootstrap durante el tiempo de encendido de los MOSFET. Podr\'{i}a usarse un capacitor m\'{a}s peque\~{n}o, a costa de permitir una mayor ca\'{i}da de tensi\'{o}n en el capacitor. 

\noindent 

\noindent Finalmente, se decidi\'{o} utilizar 2 capacitores de bootstrap en paralelo de 5.6 uF cada uno con el objetivo de reducir la resistencia serie.

\noindent 
{\bf 4.1.2.3.3. Resistencia entre gate y source}

\noindent Se colocan resistencias que conectan el gate y el source de cada MOS en el puente H. Estas se observan en la \textbf{Figura 4.9 }como R1, R2, R3, R4\textbf{.} Su prop\'{o}sito es evitar que el gate del mosfet se encuentre cargado cuando el circuito se enciende y el driver de corriente a\'{u}n no puede descargarlo. Adem\'{a}s ayuda a evitar que se encienda el mosfet por ruido acoplado capacitivamente. \textbf{}

\noindent 

\noindent Se utiliza una resistencia de 4k7 debido a que permite que el gate se descargue en un tiempo r\'{a}pido, consumiendo solo 2.55mA del capacitor de bootstrap.

\noindent 

\noindent 
{\bf 4.1.2.3.4. Protecci\'{o}n del gate}

\noindent El gate de los MOS es sensible a las sobretensiones. Soporta como m\'{a}ximo $\mathrm{\pm}$ 20V. Una descarga electrost\'{a}tica (ESD) puede sobrepasar ampliamente este valor de tensi\'{o}n, pudiendo da\~{n}ar el MOS al acercar la mano o la sonda del osciloscopio. Para protegerlo en estos casos se coloca un diodo TVS entre el gate y source, de manera de limitar la tensi\'{o}n que se desarrolla en el gate a un valor seguro.

\noindent 

\noindent Se eligen los TVS SMAJ15 con una tensi\'{o}n bidireccional de $\mathrm{\pm}$ 15V. y se colocan entre el gate y source de cada transistor.

\noindent 
{\bf 4.1.2.3.5. Tiempo muerto}

\noindent Para evitar generar un cortocircuito durante la conmutaci\'{o}n de los transistores, el driver HIP4081A permite configurar un tiempo muerto que debe transcurrir desde que se apaga un transistor y se enciende el pr\'{o}ximo. Esto se configura mediante dos resistencias conectadas a los pines LDEL y HDEL del HIP4081A.

\noindent 

\noindent Para saber el tiempo muerto necesario, debe conocerse el tiempo que tarda en apagarse un mosfet IPB160N04. De [5] se obtiene que este tiempo es 63 ns. Este valor se obtiene de tener en cuenta el Toff y el Tfall de la hoja de datos . Por lo tanto se elige que el deadtime sea de 100 ns, para tener un margen, adem\'{a}s de que esta aplicaci\'{o}n espec\'{i}fica no requiere un tiempo de encendido r\'{a}pido de los mosfets.

\noindent 

\noindent Seg\'{u}n la hoja de datos del HIP4081A, para obtener ese tiempo muerto, las resistencias en HDEL y LDEL deben ser $200\ K\mathit{\Omega}$.

\noindent 
\subparagraph{4.1.2.4. Dimensionamiento de los capacitores de fuente}

\noindent Para reducir el consumo de potencia de la red se utilizan capacitores en paralelo a la fuente de +24V. Esto permite que, una vez que la fuente carg\'{o} inicialmente el inductor, en las conmutaciones sucesivas la carga del inductor pase a dichos capacitores en un semiciclo y viceversa en el otro ciclo de conmutaci\'{o}n. Idealmente, esta transferencia de energ\'{i}a no tiene p\'{e}rdidas. Por lo tanto, el consumo de potencia queda reducido a la perdida por disipaci\'{o}n de los MOSFET y los dem\'{a}s componentes del controlador de corriente. 

\noindent 

\noindent Estos capacitores deben tener una baja resistencia equivalente serie (ESR) ya que, de lo contrario, disipan mucha potencia en forma de calor y se acorta su vida \'{u}til. Adem\'{a}s generan ripple en la tensi\'{o}n Vbus.

\noindent 

\noindent En la \textbf{Figura 4.9} los capacitores de la fuente est\'{a}n representados por C1 y C2. Para poder dimensionarlos correctamente hay que tener en cuenta que la forma de onda de la corriente que circula por el electroim\'{a}n en r\'{e}gimen permanente es aproximadamente triangular. Esta corriente es conducida durante medio ciclo desde estos capacitores y hacia el electroim\'{a}n por Q1 y Q4. Luego, durante la otra mitad del ciclo, la corriente regresa a estos capacitores a trav\'{e}s de Q2 y Q3. Esto provoca que la corriente en los capacitores sea, durante el semiciclo encendido, igual al valor medio de la corriente del electroim\'{a}n, con $\pm \frac{{\mathit{\Delta}}_{I_L}}{2}$. Similarmente ocurrir\'{a} en el semiciclo apagado, pero con valor medio ${-I}_L$.  Por lo tanto,  la corriente tendr\'{a} la forma que se muestra en la \textbf{Figura 4.10}.

\noindent 

\noindent \includegraphics*[width=4.69in, height=3.12in]{image18}

\noindent \textbf{Figura 4.10. }Forma de onda de la corriente en C1 y C2.

\noindent \textbf{}

\noindent Sabiendo que por el electroim\'{a}n circular\'{a} una corriente media de 21A en condiciones normales de trabajo, la carga del capacitor se puede calcular como:

\noindent 

\begin{tabular}{|p{3.9in}|p{0.4in}|} \hline 
$\mathit{\Delta}Q=\int I\ dt$ &  \\ \hline 
\end{tabular}



\begin{tabular}{|p{3.9in}|p{0.4in}|} \hline 
${\mathit{\Delta}Q^+\ =\ \frac{T_S}{2}*{\mathit{\Delta}I}_L*\frac{1}{2}\ +\ (}{<I}_L>\ -{\mathit{\Delta}}_{I_L}/2)*\frac{T_S}{2}$ &  \\ \hline 
\end{tabular}



\begin{tabular}{|p{3.9in}|p{0.4in}|} \hline 
$\mathit{\Delta}Q^+={<I}_L>\ *\frac{T_S}{2}\ $ &  \\ \hline 
\end{tabular}



\noindent Considerando $\ {\mathit{\Delta}I}_L=500\ m$$A$ y $\mathit{\Delta}T=0.47ms$ que corresponde a Y = 2mm seg\'{u}n la \textbf{Tabla 4.1}.

\noindent 

\begin{tabular}{|p{3.9in}|p{0.4in}|} \hline 
$\mathit{\Delta}Q=21\ A*\frac{0.47\ mSeg}{2}\ \approx \ 5\ mC$ &  \\ \hline 
\end{tabular}



\noindent Consideramos que un ripple de $\mathit{\Delta}V=500\ mV$ es aceptable, obtenemos un valor de:

\noindent 

\begin{tabular}{|p{3.9in}|p{0.5in}|} \hline 
$C=\frac{\mathit{\Delta}Q}{\mathit{\Delta}V}=10\ mF$  &  \\ \hline 
\end{tabular}



\noindent Para obtener este valor de capacidad utilizamos varios capacitores en paralelo para disminuir la ESR, como se muestra en la \textbf{Figura 4.11}. Esto es porque por los capacitores circula una corriente de hasta 21.25 A. Por lo tanto, al colocarlos en paralelo se reduce la corriente que circula (en partes iguales), por cada capacitor:

\noindent 

\noindent \includegraphics*[width=4.24in, height=2.35in]{image19}

\noindent \textbf{Figura 4.11.} Capacitores de la fuente.

\noindent 

\begin{tabular}{|p{3.9in}|p{0.5in}|} \hline 
$C=C1+C2+...+Cn$ &  \\ \hline 
\end{tabular}



\noindent Si todos los valores de ESR son iguales obtenemos

\noindent 

\begin{tabular}{|p{3.9in}|p{0.5in}|} \hline 
$Rt=\frac{R(esr)}{n}$ &  \\ \hline 
\end{tabular}



\noindent Por lo tanto reemplazando se puede calcular la potencia que disipan como:

\noindent 

\begin{tabular}{|p{3.9in}|p{0.4in}|} \hline 
$P=I^2*Rt={21.25}^2*\frac{R(esr)}{n}$ &    \eqref{GrindEQ__4_4_}  \\ \hline 
\end{tabular}



\noindent Se decidi\'{o} utilizar 6 capacitores  de 2200 uF/50V con una ESR de 17 mOhm (datos obtenidos de [4]). Por lo tanto reemplazando en la \textbf{ecuaci\'{o}n 4.4} se obtiene que la potencia disipada es de: 

\noindent 

\begin{tabular}{|p{3.9in}|p{0.4in}|} \hline 
$P=1.28\ Watt$ &  \\ \hline 
\end{tabular}


\subparagraph{4.1.2.5. Conmutaci\'{o}n de alta frecuencia para el bootstrap }

\noindent Cuando el mosfet driver recibe una entrada que activa un MOS del lado superior, este comienza a cargar el gate con ayuda de la tensi\'{o}n que brinda el capacitor de bootstrap asociado a ese MOS. El capacitor de bootstrap entrega energ\'{i}a durante la carga del gate y durante todo el tiempo que el MOS est\'{e} activo (debido a la resistencia Rgs). Para poder recargar el capacitor, debe esperarse a que el driver reciba la entrada necesaria para apagar el MOS. Debido a que la implementaci\'{o}n del driver de corriente utiliza un controlador por hist\'{e}resis, no es posible asegurar que haya una conmutaci\'{o}n en un periodo regular.

\noindent 

\noindent Para poder asegurar un periodo de conmutaci\'{o}n constante y conocido se agrega un bloque que superpone una conmutaci\'{o}n de alta frecuencia a la se\~{n}al de control que ingresa al mosfet driver. De esta manera se producen conmutaciones en un intervalo regular y se cargan regularmente los capacitores de bootstrap. 

\noindent 

\noindent Se adopta una frecuencia de conmutaci\'{o}n auxiliar de 50KHz y se hace variar el ciclo de trabajo de la salida del comparador con hist\'{e}resis entre dos valores. Durante la carga del inductor, el ciclo de trabajo ser\'{a} del 90\% mientras que durante la descarga ser\'{a} del 10\%.

\noindent 

\noindent Para generar esta conmutaci\'{o}n se agrega el oscilador que se observa en la \textbf{Figura 4.12} a la salida del comparador con hist\'{e}resis. La frecuencia de conmutaci\'{o}n se puede obtener en funci\'{o}n de C1 como:

\noindent 

\begin{tabular}{|p{3.9in}|p{0.4in}|} \hline 
${{F_{aux}}}\ =\ \frac{4.5*10^{-5}}{C1}[Hz]$ &   \\ \hline 
\end{tabular}

Esta frecuencia debe ser mucho mayor a la fundamental de la corriente triangular para evitar problemas en el funcionamiento del sistema y, adem\'{a}s, debe ser lo suficientemente alta para poder ser filtrada sin inconvenientes en la etapa de estimaci\'{o}n de posici\'{o}n. Por lo tanto, al adoptar una frecuencia auxiliar de 50 KHz, resulta en $C1$= 900 pF.

\noindent 

\noindent \includegraphics*[width=2.50in, height=2.46in]{image20}

\noindent \textbf{Figura 4.12.} Circuito oscilador de frecuencia auxiliar.

\noindent 
\subparagraph{4.1.2.6. Simulaci\'{o}n del sistema con oscilador auxiliar}

\noindent En la \textbf{Figura 4.13} se muestran las formas de onda obtenidas considerando el oscilador auxiliar necesario para el funcionamiento del bootstrap. 

\noindent 

\noindent \includegraphics*[width=6.24in, height=2.47in]{image21}

\noindent \textbf{Figura 4.13}. Simulaci\'{o}n de corriente en el electroim\'{a}n, salida del comparador, y conmutaci\'{o}n auxiliar.

\noindent 
\paragraph{4.1.3. Caracter\'{i}sticas est\'{a}ticas y din\'{a}micas del controlador}

\noindent 
\subparagraph{4.1.3.1. Corriente media del electroim\'{a}n}

\noindent Para saber la corriente media que habr\'{a} a la salida con cierta tensi\'{o}n de entrada, se utiliza la transferencia de lazo cerrado (sin considerar polos, y suponiendo una alta ganancia de lazo abierto):

\noindent 

\begin{tabular}{|p{3.9in}|p{0.4in}|} \hline 
$I_L\ =\ V_{i_-ref}\ *\ \frac{Kin}{H(s)}=V_{i_-ref}*6\ \frac{A}{V}$ &   \\ \hline 
\end{tabular}


\subparagraph{4.1.3.2. Frecuencia de conmutaci\'{o}n de la corriente}

\noindent La frecuencia de conmutaci\'{o}n del sistema se obtiene con:

\noindent 

\begin{tabular}{|p{3.9in}|p{0.4in}|} \hline 
$fsw\ =\ \frac{V_{bus}}{2*\mathit{\Delta}I_L*L(y)}$\textbf{} & \eqref{GrindEQ__4_5_}  \\ \hline 
\end{tabular}



\noindent Para $y\ =\ 4\ mm$ se tiene una inductancia $L(4mm)\ =\ 16.44\ mHy$, lo cual resulta en una frecuencia $fsw\ =\ 1460\ Hz$.

\noindent 
\subparagraph{4.1.3.3. Ancho de banda del controlador}

\noindent La din\'{a}mica del controlador, al depender de la inductancia, lo hace tambi\'{e}n del \textit{gap} de aire. El ancho de banda (o velocidad con que responde) est\'{a} limitado por la constante de tiempo del inductor con su resistencia serie. Juntas forman un sistema lineal de primer orden, con un polo en:

\noindent 

\begin{tabular}{|p{3.9in}|p{0.5in}|} \hline 
${\boldsymbol{f}}_{\boldsymbol{polo}}\boldsymbol{\ =}\frac{\boldsymbol{1}}{\boldsymbol{2}\boldsymbol{\pi }\boldsymbol{*}\boldsymbol{\tau }}\boldsymbol{=\ }\frac{\boldsymbol{R}}{\boldsymbol{2}\boldsymbol{\pi }\boldsymbol{*}\boldsymbol{L}\boldsymbol{(}\boldsymbol{y}\boldsymbol{)}}$ &  \\ \hline 
\end{tabular}



\noindent Al convertirlo a frecuencia angular:

\begin{tabular}{|p{3.9in}|p{0.4in}|} \hline 
${\omega }_{\boldsymbol{polo}}\boldsymbol{\ =}\frac{\boldsymbol{R}}{\boldsymbol{L}\boldsymbol{(}\boldsymbol{y}\boldsymbol{)}}$ & \eqref{GrindEQ__4_6_}  \\ \hline 
\end{tabular}



\noindent Tomando las condiciones del problema con Yo = 4mm, L = 7.55 mH + 8.89 mH, y $Rl=0.2\mathit{\Omega}$ el polo se ubica:

\noindent 

\begin{tabular}{|p{3.9in}|p{0.4in}|} \hline 
${\omega }_{polo}\ =\frac{0.2\ \mathit{\Omega}}{16.44\ mH}=12.17\ rad/s$ &   \\ \hline 
\end{tabular}



\noindent La \textbf{Tabla 4.2} muestra  entre qu\'{e} valores de frecuencia se ver\'{a} afectada la forma de onda al modificarse la distancia de separaci\'{o}n.

\noindent 

\begin{tabular}{|p{3.9in}|p{0.5in}|} \hline 
$\mathit{\Delta}T(seg)=\frac{\mathit{\Delta}Il*(L(y)\ +\ L_{\infty })}{Vbus}$ & \eqref{GrindEQ__4_7_}  \\ \hline 
\end{tabular}



\noindent Considerando $Rl=0.2\mathit{\Omega}\ $     

\noindent 

\noindent En la \textbf{ecuaci\'{o}n 4.7,} $\mathit{\Delta}T$ representa el tiempo de crecimiento o de decrecimiento de la rampa de corriente (despreciando la resistencia del bobinado) en torno al valor nominal. El doble de este tiempo es igual al periodo de la corriente triangular $(2*\mathit{\Delta}T=\frac{1}{f_{SW}})$.

\noindent Seg\'{u}n las mediciones de inductancia realizadas y aplicando las \textbf{ecuaciones 4.5, 4.6 }y\textbf{ 4.7} se arm\'{o} la \textbf{Tabla 4.2}.

\noindent 

\begin{tabular}{|p{0.9in}|p{0.7in}|p{0.7in}|p{0.8in}|p{0.8in}|} \hline 
$y(mm)$ & $L(y)\ [mHy]$ & $\mathit{\Delta}T$ [mS] & $f_{SW}$[Hz] & $W_{polo}[rad/s]$  \\ \hline 
0 & 76.45 & 1.59 & 313.93 & 2.62 \\ \hline 
1 & 33.42 & 0.70 & 718.13 & 5.98 \\ \hline 
2 & 22.64 & 0.47 & 1,060.07 & 8.83 \\ \hline 
3 & 18.8 & 0.39 & 1,276.60 & 10.64 \\ \hline 
4,4 & 15.5 & 0.32 & 1,548.39 & 12.90 \\ \hline 
5,2 & 14.7 & 0.31 & 1,632.65 & 13.61 \\ \hline 
6,5 & 14.4 & 0.30 & 1,666.67 & 13.89 \\ \hline 
8,23 & 12.4 & 0.26 & 1,935.48 & 16.13 \\ \hline 
inf & 8.89 & 0.19 & 2,699.66 & 22.5 \\ \hline 
\end{tabular}

\textbf{Tabla 4.2. }Valores calculados y medidos en funci\'{o}n del Gap de aire

\noindent 
\paragraph{4.1.4. Transferencia lineal del controlador de corriente}

\noindent En la \textbf{ecuaci\'{o}n 4.8 }se muestra la transferencia linealizada del controlador de corriente.

\begin{tabular}{|p{3.9in}|p{0.4in}|} \hline 
$TLC_{CC}=\frac{6A}{(1+\frac{s}{W_{Polo}})}$ & \eqref{GrindEQ__4_8_}  \\ \hline 
\end{tabular}


\subsection{4.2. Dise\~{n}o y modelado del Estimador Analogico}

\noindent Para controlar la distancia de separaci\'{o}n del entrehierro del electroim\'{a}n es necesario conocer el gap de aire para poder realimentarlo en el lazo de control.  Para ello, se utiliza un estimador de posici\'{o}n que aprovecha la forma de onda triangular de la corriente que circula por el electroim\'{a}n. 

\noindent 

\noindent Para estimar la distancia se hace la derivada de la corriente, puesto que las pendientes de crecimiento y decrecimiento var\'{i}an con la separaci\'{o}n. Es importante tener en cuenta que durante el dise\~{n}o de la etapa de controlador de corriente, se eligi\'{o} una topolog\'{i}a que mantiene el sistema conmutando cont\'{i}nuamente (incluso para corriente nula) para tener siempre una estimaci\'{o}n disponible.

   

\noindent \includegraphics*[width=6.19in, height=1.06in]{image22}

\noindent \textbf{Figura 4.14.} Diagrama en Bloques del Estimador

\noindent 

\noindent Se implementa un estimador compuesto por los bloques mostrados en la \textbf{Figura 4.14}. A este le ingresa una tensi\'{o}n triangular (ViL) que es la salida del sensor de efecto Hall. Para eliminar las componentes de alta frecuencia se aplica un filtro pasa bajos dejando pasar hasta la quinta arm\'{o}nica. Esta se\~{n}al filtrada conserva la forma triangular de la corriente. 

\noindent 

\noindent Al ingresar al derivador con ViL, la forma de onda resultante a su salida es aproximadamente cuadrada, y sus valores de alto y bajo se corresponden con las pendientes de bajada y subida multiplicadas por una constante de tiempo del derivador. Estas pendientes deber\'{i}an ser sim\'{e}tricas alrededor del punto de operaci\'{o}n de 2.5V, pero no lo son debido a la resistencia interna del electroim\'{a}n, que provoca que la pendiente de bajada sea mayor (en m\'{o}dulo) que la de subida. Por ello, se implementa la compensaci\'{o}n I*R, cuya salida ingresa al derivador y logra mantener la simetr\'{i}a alrededor de 2.5V. Esta se\~{n}al ingresa al \'{u}ltimo bloque que rectifica y filtra la forma de onda, obteni\'{e}ndose una tensi\'{o}n continua (Vout) proporcional a la distancia de separaci\'{o}n del gap (Yo).

\noindent 
\paragraph{4.2.1. An\'{a}lisis de la estimaci\'{o}n}

\noindent La ecuaci\'{o}n que gobierna la corriente en el electroim\'{a}n se puede calcular aplicando las leyes de Kirchoff correspondientes al circuito que se ve en la \textbf{Figura 4.15.}

\noindent \includegraphics*[width=4.44in, height=2.32in]{image23}

\noindent \textbf{Figura 4.15.} Circuito del electroim\'{a}n con el driver de corriente.

\noindent 

\noindent Sabiendo que$\ L(y)$ se puede aproximar como en la \textbf{ecuaci\'{o}n} \textbf{4.9}, y que $L_{\infty }$(inductancia de dispersi\'{o}n) es la inductancia del electroim\'{a}n sin la pieza en forma de ``I'' :

\noindent 

\begin{tabular}{|p{3.9in}|p{0.4in}|} \hline 
$L(y)\ \approx \ \mu o\frac{N^2*A}{2Y}$ & \eqref{GrindEQ__4_9_}  \\ \hline 
\end{tabular}



\begin{tabular}{|p{3.9in}|p{0.4in}|} \hline 
$\pm V_{BUS}-\ L(y)*\left|\frac{{di}_L}{dt}\right|-L_{\infty }*\left|\frac{{di}_L}{dt}\right|-R_L*I_L=0$ &  \\ \hline 
\end{tabular}



\noindent Asumiendo que:

 

\begin{tabular}{|p{3.9in}|p{0.4in}|} \hline 
$V_{BUS}>>i_L*R_L$ &  \\ \hline 
\end{tabular}



\noindent Se aproxima la derivada de la corriente como:

\noindent 

\begin{tabular}{|p{3.9in}|p{0.4in}|} \hline 
$\left|\frac{{di}_L}{dt}\right|\simeq \frac{V_{BUS}}{L(y)+L_{\infty }}=\frac{V_{BUS}}{L_T(y)}$ & \eqref{GrindEQ__4_10_}  \\ \hline 
\end{tabular}



\noindent Seg\'{u}n mediciones realizadas, se tienen los valores de $L_T(y)$ correspondientes a cada posici\'{o}n. En base a ellos se hace una aproximaci\'{o}n lineal para obtener la expresi\'{o}n de la derivada de la \textbf{ecuaci\'{o}n 4}.\textbf{10.}

\noindent 

\begin{tabular}{|p{3.9in}|p{0.4in}|} \hline 
${\left|\frac{{di}_L}{dt}\right|}_{Lineal}=\ 194690\ *\ Y[m]+676\ A/s$ & \eqref{GrindEQ__4_11_}  \\ \hline 
\end{tabular}



\noindent 

\noindent \textbf{}

\noindent 
\paragraph{4.2.2. Modelo circuital del estimador de posici\'{o}n}

\noindent Para poder obtener $\left|\frac{{di}_L}{dt}\right|$ se utiliza un circuito derivador con un amplificador operacional como se observa en la \textbf{Figura 4.16}.

\noindent \includegraphics*[width=3.27in, height=2.36in]{image24}

\noindent \textbf{Figura 4.16.} Circuito derivador.

\noindent 

\noindent La salida del circuito, $V_{yf}(t)$, ante una entrada $V_L$ es:

\noindent 

\begin{tabular}{|p{3.9in}|p{0.4in}|} \hline 
$V_{yf}(t)\ =\ 2.5V\ -\ \frac{dV_L}{dt}*C_1*R_1$\textbf{} &   \\ \hline 
\end{tabular}



\noindent Considerando $V_L=K_h*i_L$, donde Kh es la constante del sensor de efecto Hall, se obtiene: 

\noindent 

\begin{tabular}{|p{3.9in}|p{0.4in}|} \hline 
$V_{yf}(t)\ =2.5V\ -\frac{diL}{dt}*K_h*C_1*R_1$\textbf{} & \eqref{GrindEQ__4_12_} \\ \hline 
\end{tabular}



\noindent $V_{yf}(t)$ tiene variaciones alrededor del setpoint de 2.5 V. Por lo tanto, para evitar la saturaci\'{o}n del derivador se debe cumplir que:

\noindent 

\begin{tabular}{|p{3.9in}|p{0.4in}|} \hline 
$\left|-\frac{diL}{dt}*K_h*C_1*R_1\right|\ \le 2.5V$ & \eqref{GrindEQ__4_13_} \\ \hline 
\end{tabular}



\noindent Por lo tanto, con la ecuaci\'{o}n \eqref{GrindEQ__4_10_} y \eqref{GrindEQ__4_13_}:

\noindent 

\begin{tabular}{|p{3.9in}|p{0.4in}|} \hline 
$C_1*R_1<=\frac{2.5\ V\ *L_{min}}{V_{BUS}*K_h}$\textbf{} &  \\ \hline 
\end{tabular}



\noindent Con $L_{min}=\ L_T(5\ mm)=\ 14.9\ mH$ (teniendo en cuenta la inductancia de dispersi\'{o}n) se obtiene: 

\begin{tabular}{|p{3.9in}|p{0.5in}|} \hline 
$C_1*R_1<=\ 29.1\ ms$\textbf{} &   \\ \hline 
\end{tabular}



\noindent Este derivador tendr\'{a} como salida una onda pulsada, cuyo flanco superior  es proporcional a la pendiente de bajada de la corriente en el electroim\'{a}n, y el flanco inferior es proporcional a la pendiente de subida de la corriente. 

\noindent Para los c\'{a}lculos se utiliz\'{o} $C_1*R_1=\ 25\ mS$, para dar un margen y evitar la saturaci\'{o}n del amplificador operacional.  

\noindent 

\noindent Usando la \textbf{ecuaci\'{o}n 4.11 }y \textbf{4.12}, y considerando una variaci\'{o}n en torno a 2.5V se obtiene:

\textbf{}

\begin{tabular}{|p{3.9in}|p{0.5in}|} \hline 
$Vyf(y)\ =\ |Kh*C_1*R_1*di/dt)|\ +2.5V=0.2595*y(mm)+3.4V$\textbf{} & \textbf{} \\ \hline 
\end{tabular}



\noindent Se puede observar en la \textbf{Tabla 4.3} que para el rango de valores posibles en los que el electroim\'{a}n trabajar\'{a}, el estimador posee un rango de salida ${\mathit{\Delta}{Vyf}_{Lineal}}(5-2\ mm)=\ 0.78\ V$.

\noindent 

\begin{tabular}{|p{2.1in}|p{2.1in}|} \hline 
Y(mm) & ${Vyf(y)}_{Lineal}$ \\ \hline 
2 & 3.92 \\ \hline 
3 & 4.18 \\ \hline 
4 & 4.44 \\ \hline 
5 & 4.7 \\ \hline 
\end{tabular}

\textbf{Tabla 4.3. }Vyf en funci\'{o}n de la posici\'{o}n

\noindent 
\paragraph{4.2.3. Circuito del derivador compensado}

\noindent Puesto que los circuitos derivadores pueden presentar inestabilidad a alta frecuencia, es necesario compensarlo agregando una resistencia en serie al capacitor, para que genere un cero en la transferencia de realimentaci\'{o}n (\textbf{ecuaci\'{o}n 4.15}), como se observa en la \textbf{figura 4.17}. 

\noindent 

\noindent \includegraphics*[width=4.71in, height=2.94in]{image25}

\noindent \textbf{Figura 4.17.} Circuito derivador compensado.

\noindent 

\noindent Se eligi\'{o} el amplificador operacional MCP660 [6], que tiene una ganancia continua de 125 dB, y un polo en 20 Hz (ecuaci\'{o}n 4.14). Adem\'{a}s, presenta un slew rate de 32 V/us.

\noindent 

\noindent El operacional es internamente compensado, por lo que todos sus otros polos los tiene luego de el cruce por 0 dB de la ganancia. Para simplificar el an\'{a}lisis no se tienen en cuenta estos, ya que est\'{a}n fuera de la zona de inter\'{e}s. 

\noindent 

\begin{tabular}{|p{3.9in}|p{0.5in}|} \hline 
$A(w)=\frac{1778279}{(\frac{s}{2\pi *20}+1)}$ & \eqref{GrindEQ__4_14_}  \\ \hline 
\end{tabular}



\begin{tabular}{|p{3.9in}|p{0.5in}|} \hline 
$\frac{1}{H(w)}=\frac{1+s*C_1*(R_1+R_2)}{1+s*C_1*R_2}\simeq \frac{1+s*C_1*R_1}{1+s*C_1*R_2}$ & \eqref{GrindEQ__4_15_}  \\ \hline 
\end{tabular}



\noindent Para compensar el circuito se coloca un polo en 16 kHz, dando como resultado $R2=10\ ohm$, $C1=1\ uF$ y $R1=25\ kOhm\ $y un margen de fase de $\phi =49.6{}^\circ $, como se puede observar en la \textbf{Figura 4.18.}

\noindent \includegraphics*[width=5.41in, height=3.86in]{image26}

\noindent 

\noindent \textbf{Figura 4.18}. G*H del derivador compensado. 

\noindent 

\noindent 

\noindent \includegraphics*[width=5.16in, height=4.22in]{image27}

\noindent \textbf{Figura 4.19.} Transferencia de lazo cerrado.

\noindent 

\noindent Como se observa en la \textbf{figura 4.19, }la transferencia de lazo cerrado (TLC) tiene un comportamiento derivativo en las frecuencias cercanas a 2 kHz, como es deseado.

\noindent 

\noindent A continuaci\'{o}n se muestra la TLC del circuito derivador:

\noindent 

\begin{tabular}{|p{3.9in}|p{0.4in}|} \hline 
${Tlc}_{derivador}=\frac{V_{yf}}{Vil}=\frac{-0.025*s}{1+(\frac{2*0.473}{94,5\ krad/s})*s+(\frac{s}{94,5\ krad/s})^2}$ & \eqref{GrindEQ__4_16_}  \\ \hline 
\end{tabular}


\paragraph{4.2.4. Dise\~{n}o del LPF}

\noindent Debido a que el derivador amplifica las se\~{n}ales de alta frecuencia es necesario agregar un filtro pasa bajos en su entrada. Como la se\~{n}al que va a ingresar al derivador es ViL, la cual es una onda triangular de frecuencia fundamental de 2KHz se dejar\'{a} pasar hasta la 5º arm\'{o}nica. Para su implementaci\'{o}n se utiliza un filtro activo Butterworth de orden 2, con una frecuencia de corte en 20 KHz. En la \textbf{Figura 4.20 }se puede ver el filtro utilizado y en la \textbf{Figura 4.21}, su respuesta en frecuencia.

\noindent 

\noindent \includegraphics*[width=3.66in, height=2.56in]{image28}

\noindent \textbf{Figura 4.20. }Filtro para la entrada del derivador

\noindent 

\noindent \includegraphics*[width=5.86in, height=2.46in]{image29}

\noindent \textbf{Figura 4.21. }Respuesta en frecuencia del filtro activo.

\noindent 
\paragraph{4.2.5. Compensaci\'{o}n I*R}

\noindent Al circular corriente siempre en el mismo sentido por el electroim\'{a}n, se produce una ca\'{i}da de tensi\'{o}n casi constante en la resistencia interna, haciendo que no siempre est\'{e}n aplicados $\pm 24V$ al electroim\'{a}n sino que durante el $T_{ON}$ se aplican $24V-I*R$ y durante el $T_{OFF}$ se aplican $-24V-I*R$, haciendo que las pendientes sean distintas. 

\noindent 

\begin{tabular}{|p{3.9in}|p{0.4in}|} \hline 
$\pm V_{BUS}-L(y)*\left|\frac{{di}_L}{dt}\right|-L_{\infty }*\left|\frac{{di}_L}{dt}\right|-R_L*I_L=0$ &  \\ \hline 
\end{tabular}



\noindent C\'{o}mo  $R_L=0.2\ \mathit{\Omega}$ y suponiendo una corriente de $21\ A$  

\noindent 

\begin{tabular}{|p{3.9in}|p{0.5in}|} \hline 
$\pm V_{BUS}-R_L*I_L=\ \pm 24-4.2$ & \eqref{GrindEQ__4_17_}  \\ \hline 
\end{tabular}



\noindent Por lo tanto, para $V_{BUS}=24\ V$:

\begin{tabular}{|p{3.9in}|p{0.4in}|} \hline 
$V_{BUS}-R_L*I_L=\ +24-4.2=\ 19.8V$ &   \\ \hline 
\end{tabular}



\noindent Para $V_{BUS}=-24\ V$

\noindent 

\begin{tabular}{|p{3.9in}|p{0.5in}|} \hline 
$V_{BUS}-R_L*I_L=\ -24-4.2=\ 28.2V$ &  \\ \hline 
\end{tabular}



\noindent Por lo tanto, sobre el electroim\'{a}n se aplicar\'{a}n dos tensiones distintas, en valor absoluto, durante la carga y descarga. Esto provoca que la rampa de corriente sea asim\'{e}trica.

\noindent 

\noindent Como luego se utilizar\'{a} un rectificador de onda completa, se desea que la rectificaci\'{o}n de cada una de estas pendientes resulte en el mismo valor. En la \textbf{Figura 4.22 }se muestra el efecto luego de la rectificaci\'{o}n sin realizar ninguna compensaci\'{o}n:

\noindent 

\noindent \includegraphics*[width=6.23in, height=1.17in]{image30}

\noindent \textbf{Figura 4.22.} Forma de onda luego de rectificar sin compensaci\'{o}n I*R.

\noindent 

\noindent Se busca corregir esto en la estimaci\'{o}n variando el setpoint de la salida del derivador. Para lograrlo se debe cambiar la tensi\'{o}n en la entrada no inversora ($V_{bias}$) como se muestra en la \textbf{Figura 4.23}. 

\noindent 

\noindent \textbf{\includegraphics*[width=6.26in, height=2.60in]{image31}}

\noindent \textbf{Figura 4.23}. Esquema circuital del derivador

\noindent 

\noindent Se tiene que la pendiente de bajada de la onda triangular, en m\'{o}dulo, es mayor que la de subida. Por lo tanto, al derivar (con la inversi\'{o}n de signo), esta quedar\'{a} por encima del setpoint, y la pendiente de subida quedar\'{a} por debajo. Se debe compensar ese setpoint para que la forma de onda sea sim\'{e}trica alrededor de 2.5 V. 

\noindent 

\noindent Para la pendiente de bajada, la salida del derivador ser\'{a}:

\noindent 

\begin{tabular}{|p{3.9in}|p{0.4in}|} \hline 
${Vyf}_{off}\ =\ V_{bias}+Kh\ *\ \tau *\frac{Vbus\ +\ Il*R}{L}\ $ &   \\ \hline 
\end{tabular}



\noindent Para la pendiente de subida se tiene:

\noindent 

\begin{tabular}{|p{3.9in}|p{0.4in}|} \hline 
${Vyf}_{on}\ =\ V_{bias}\ -\ Kh\ *\ \tau *\frac{Vbus\ -\ Il*R}{L}$ &   \\ \hline 
\end{tabular}



\noindent Queremos que se cumpla:

\noindent 

\begin{tabular}{|p{3.9in}|p{0.4in}|} \hline 
${Vyf}_{off}\ -2.5\ V=\ 2.5\ V\ -{Vyf}_{on}$ &   \\ \hline 
\end{tabular}



\noindent Se despeja $V_{bias}$ y se llega a:

\noindent 

\begin{tabular}{|p{3.9in}|p{0.4in}|} \hline 
$V_{bias}\ =2.5\ V-\ Kh\ *Il*\ \tau *\frac{\ R}{L}$ & \eqref{GrindEQ__4_18_}  \\ \hline 
\end{tabular}



\noindent Se tiene $Kh\ =\ 53,3\ \frac{mV}{A},\ R\ =\ 0.2\ \mathit{\Omega},\ \tau \ =\ 25\ ms$. En cuanto a la inductancia, se utiliza:

\noindent 

\noindent $L_T(y)\ =\ 16,44\ mHy$ (para Yo = 4mm).

\noindent 

\noindent ViL es la tensi\'{o}n de salida del sensor de efecto Hall menos un setpoint de 2.5V. Sin embargo, debido al offset agregado al sensor para llevar su valor medio a 2.6V, al restarle 2.5V no se produce una cancelaci\'{o}n completa sino que quedan 0.1V de error. Por ello, para implementar la \textbf{ecuaci\'{o}n} \textbf{4.18} se utiliza el circuito mostrado en la \textbf{Figura 4.24.} Este circuito compensa la diferencia de pendientes, el error de 0.1V y genera $V_{bias}$ para ingresar al derivador. 

\noindent 

\noindent \includegraphics*[width=5.17in, height=3.33in]{image32}

\noindent \textbf{Figura 4.24. }Generaci\'{o}n de Vbias

\noindent 

\noindent A partir del circuito de la \textbf{Figura 4.24 }se obtiene:

\noindent 

\begin{tabular}{|p{3.9in}|p{0.4in}|} \hline 
$V_{bias}\ =-\frac{R_4}{R_3}(K_hI_L+\ 0.1V)+V_{Ref_{bias}}(1+\frac{\ R_4}{R_3})*(\frac{R_1}{R_1+R_2})$ &   \\ \hline 
\end{tabular}



\noindent Para poder llegar a la expresi\'{o}n de la \textbf{ecuaci\'{o}n} \textbf{4.17 }se debe cumplir que:

\noindent 

\begin{enumerate}
\item  $-\frac{R_4}{R_3}=-\ \tau *\frac{\ R}{L}=\ -0.304$  

\item  $-\frac{R_4}{R_3}(\ 0.1V)+V_{Ref_{bias}}(1+\frac{\ R_4}{R_3})*(\frac{R_1}{R_1+R_2})\ =\ 2.5V$     
\end{enumerate}

\noindent 

\noindent Por lo tanto, resolviendo la condici\'{o}n A) se elige R4 = 304 $\mathit{\Omega}$ y se obtiene $R_3=1\ k\mathit{\Omega}$. Luego, resolviendo la condici\'{o}n B) con $V_{Ref_{bias}}=2.5V$ se elige $R_1=1k\mathit{\Omega}$ y se obtiene $R_{2\ }=291.8\mathit{\Omega}.$

\noindent 

\noindent En la \textbf{Figura 4.25} se muestra como cambia la forma de onda.

\noindent \includegraphics*[width=5.75in, height=3.86in]{image33}

\noindent \textbf{Figura 4.25}. Formas de onda obtenidas en la simulaci\'{o}n.

\noindent 

\noindent La onda superior corresponde a la corriente en el electroim\'{a}n (verde), la onda que se encuentra al medio (amarilla) corresponde a la salida del derivador ${[V}_{bias}$$]$ y la inferior (roja) corresponde a la onda rectificada con la correcci\'{o}n de I*R.

\noindent 

\noindent 

\noindent 
\paragraph{4.2.6. Rectificador, Restador y Filtrado}

\noindent 
\subparagraph{4.2.6.1. Rectificador}

\noindent \includegraphics*[width=3.94in, height=3.49in]{image34}

\noindent \textbf{Figura 4.26}. Rectificador y restador.

\noindent 

\noindent Para poder tener finalmente la estimaci\'{o}n de la posici\'{o}n, debemos rectificar la salida del derivador alrededor de 2.5V. Esto se hace para hacer coincidir la pendiente positiva de la corriente triangular, con la pendiente negativa, y tener a la salida del estimador una se\~{n}al aproximadamente continua. 

\noindent 

\noindent Para entender el funcionamiento de este rectificador, se comienza analizando \'{u}nicamente la etapa del primer amplificador operacional. Se parte de la suposici\'{o}n de que en un amplificador ideal, la tensi\'{o}n diferencial ($V_d$) es igual a cero. Por lo tanto, como la entrada no inversora est\'{a} fijada en 2.5V, la misma tensi\'{o}n se encuentra en la entrada inversora. 

\noindent 

\noindent Al analizar la corriente en la resistencia $R_{25}$ (adoptando sentido positivo hacia la izquierda) en funci\'{o}n de $V_{deriv}$, resulta:

\noindent 

\begin{tabular}{|p{3.9in}|p{0.4in}|} \hline 
$I_{R25}=\frac{2.5V\ -\ V_{deriv}}{R_{25}}$ &   \\ \hline 
\end{tabular}



\noindent En el caso de que $V_{deriv}$ $\mathrm{<}$ 2.5V, la corriente ser\'{a} positiva. Esta misma corriente proviene desde la salida del operacional, a trav\'{e}s del diodo D5 y por la resistencia $R_{26}$. Si se desprecia la tensi\'{o}n del diodo en directa, nos queda que la salida del operacional es igual a V+, y esta es igual a:

\begin{tabular}{|p{3.9in}|p{0.4in}|} \hline 
$V^+=I_{R25}*R_{26}+2.5V=\frac{2.5V-V_{deriv}\ }{R25}*R_{26}+2.5V\ $ &   \\ \hline 
\end{tabular}

Como $R_{25}=R_{26}$

\begin{tabular}{|p{3.9in}|p{0.4in}|} \hline 
$V^+\ =\ 2.5V\ -\ V_{deriv}\ +2.5V\ =\ 5V\ -\ V_{deriv}\ $ &   \\ \hline 
\end{tabular}

An\'{a}logamente, si $V_{deriv}$ $\mathrm{>}$ 2.5V, se puede encontrar:

\begin{tabular}{|p{3.9in}|p{0.4in}|} \hline 
$V^-\ =\ {5V\ -V}_{deriv}\ $ &   \\ \hline 
\end{tabular}

Cuando D5 est\'{a} activo, $V^-=2.5\ V$ y cuando lo est\'{a} D6, $V^+\ $= 2.5 V

\noindent 

\noindent 

\noindent 
\subparagraph{4.2.6.2. Restador}

\noindent Se utiliza un amplificador operacional en modo diferencial como restador como se observa en la \textbf{Figura 4.27 }y se obtiene lo siguiente.

\noindent 

\noindent Cuando $V_{deriv}$ $\mathrm{<}$ 2.5V:

\begin{tabular}{|p{3.9in}|p{0.4in}|} \hline 
$Vestim=V^+-\ V^-\ +2.5V=\ (5V\ -\ V_{deriv})-(2.5\ V)+2.5V=5\ V\ -\ V_{deriv}\ $ &   \\ \hline 
\end{tabular}



\noindent Cuando $V_{deriv}$ $\mathrm{>}$ 2.5V: 

\begin{tabular}{|p{3.9in}|p{0.4in}|} \hline 
${Vestim=V}^+-\ V^-+2.5V=\ 2.5V\ -\ (5V-\ V_{deriv})+\ 2.5V=\ \ V_{deriv}\ $  &   \\ \hline 
\end{tabular}



\noindent Si tomamos a $V_{deriv}$ como $V_{deriv}=\mathit{\Delta}V_{deriv}\ +\ 2,5\ V$, reemplazando en los dos casos

\noindent obtenemos que:

\noindent 

\begin{tabular}{|p{3.9in}|p{0.4in}|} \hline 
$Vestim=\ 2,5\ V\ +\ |\mathit{\Delta}V_{deriv}|$\newline  &   \\ \hline 
\end{tabular}

\includegraphics*[width=3.15in, height=2.77in]{image35}

\noindent \textbf{Figura 4.27. }Restador

\noindent 
\subparagraph{4.2.6.3. Etapa de filtrado}

\noindent En el restador se implementa un filtrado adicional a la se\~{n}al de salida como se observa en la \textbf{Figura 4.28}. De esta \'{u}ltima etapa, considerando que $C_5=C_6=C\ $y $R_{33}=R_{31}=R$, se obtiene:

\begin{tabular}{|p{3.9in}|p{0.4in}|} \hline 
${Vestim}=\frac{1}{1+S*C*R}*{(V}^+-V^-\ +\ 2.5V)\ =\ \frac{1}{1+S*C*R}*(2,5\ V\ +\ |\mathit{\Delta}V_{deriv}|)$ &  \\ \hline 
\end{tabular}



\begin{tabular}{|p{3.9in}|p{0.4in}|} \hline 
$Vestim\ =\ \frac{1}{1+S*C*R}*\ |\mathit{\Delta}V_{deriv}|\ +2.5V$ &   \\ \hline 
\end{tabular}



\noindent Puesto que la salida Vestim debe ser una continua, es importante eliminar cualquier posible ripple permitiendo solo el paso de continua. Por ello, se escogen los siguientes valores para los componentes: 

\noindent 

\begin{enumerate}
\item  $C=10\ nF$

\item  $R=100\ Kohm$

\item  $\frac{1}{2*\pi *C*R}=159.2\ Hz$
\end{enumerate}

\noindent 

\noindent \includegraphics*[width=3.13in, height=3.24in]{image36}

\noindent \textbf{Figura 4.28}. Esquema circuital del restador con una etapa de filtrado en $159.2\ Hz$ 

\noindent 
\paragraph{4.2.7. Circuito completo}

\noindent En la \textbf{figura 4.29 }se puede observar el circuito completo utilizado para la implementaci\'{o}n del rectificador, restador y filtrado.

\noindent 

\noindent \includegraphics*[width=5.72in, height=3.21in]{image37}

\noindent \textbf{Figura 4.29. }Circuito estimador de posici\'{o}n completo.

\noindent 
\paragraph{4.2.8. Simulaci\'{o}n de estimador completo}

\noindent En la \textbf{Figura 4.30} se pueden observar 3 formas de onda. La superior (verde) corresponde a la corriente del electroim\'{a}n, la del medio (amarilla) a la salida del derivador y la inferior (roja) es la salida Vestim. Utilizando cursores se midi\'{o} un ripple de 52.66mV en Vestim.

\noindent 

\noindent \includegraphics*[width=5.66in, height=2.40in]{image38}

\noindent \textbf{Figura 4.30. }Simulaci\'{o}n final del estimador.

\noindent 

\noindent En la \textbf{Tabla 4.4} se muestran valores medidos de Vestim en funci\'{o}n de la posici\'{o}n.

\noindent 

\begin{tabular}{|p{1.6in}|p{1.6in}|p{1.6in}|} \hline 
y [mm] & L(y) [mH] & Vestim [V] \\ \hline 
2 & 22.64 & 3.86 \\ \hline 
3 & 18.8 & 4.13 \\ \hline 
4 & 16.44 & 4.36 \\ \hline 
5 & 14.9 & 4.55 \\ \hline 
\end{tabular}

\textbf{Tabla 4.4. }Resultados de simulaci\'{o}n del estimador

\noindent 
\paragraph{4.2.9. Transferencia final del estimador de posici\'{o}n:}

\noindent 

\noindent Para la transferencia de lazo cerrado del derivador se obtiene:

\noindent 

\begin{tabular}{|p{3.9in}|p{0.4in}|} \hline 
${Tlc}_{derivador}=\frac{V_{deriv}}{Vil}=\frac{-0.025*s}{1+(\frac{2*0.473}{94,5\ krad/s})*s+(\frac{s}{94,5\ krad/s})^2}$ &   \\ \hline 
\end{tabular}



\noindent De esta forma, se obtiene un sistema con un cero en el origen y dos polos complejos conjugados con una $W_n=2\pi *15000\ Hz$ y un $\xi =0.473$.

\noindent 

\noindent Por otro lado, al considerar el polo aportado por la etapa de restado y filtrado situado en 10 Krad/s y considerar la inversi\'{o}n de signo que genera el rectificador, se obtiene:

\noindent 

\begin{tabular}{|p{3.9in}|p{0.4in}|} \hline 
${Tlc}_{estimador}=\frac{V_{estim}}{Vil}=\frac{0.025*s}{(1+\frac{s}{1\ Krad/s})*[1+(\frac{2*0.473}{94,5\ Krad/s})*s+(\frac{s}{94,5\ Krad/s})^2]}$ &   \\ \hline 
\end{tabular}



\noindent Para poder obtener la transferencia del diagrama en bloques: Vestim/Y:

\noindent 

\begin{tabular}{|p{3.9in}|p{0.4in}|} \hline 
${Tlc'}=\frac{V_{estim}}{s*Vil}=\frac{V_{estim}}{kh*s*I_L}=\frac{0.025}{(1+\frac{s}{1\ Krad/s})*[1+(\frac{2*0.473}{94,5\ Krad/s})*s+(\frac{s}{94,5\ Krad/s})^2]}$ &   \\ \hline 
\end{tabular}



\noindent Como $s*I_L\equiv \frac{dI}{dt}$, se puede usar la expresi\'{o}n linealizada:

\noindent 

\begin{tabular}{|p{3.9in}|p{0.4in}|} \hline 
${\left|\frac{{di}_L}{dt}\right|}_{Lineal}=\ 194690\ *\ Y[m]+676\ A/s$ &   \\ \hline 
\end{tabular}



\noindent De esta forma,  $s*I_L\equiv 194690*Y$ (sin considerar la componente de continua)

\noindent 

\noindent Reemplazando se obtiene:

 

\begin{tabular}{|p{3.9in}|p{0.4in}|} \hline 
${Tlc'}=\frac{V_{estim}}{Y[m]}=\frac{259.6}{(1+\frac{s}{1\ Krad/s})*[1+(\frac{2*0.473}{94,5\ Krad/s})*s+(\frac{s}{94,5\ Krad/s})^2]}$ &   \\ \hline 
\end{tabular}



\noindent Considerando la etapa de filtrado de la entrada, que tiene dos polos en $2\pi *10\ KHz\ \simeq 60\ Krad/s$ se obtiene:

\noindent 

\begin{tabular}{|p{3.9in}|p{0.4in}|} \hline 
${Tlc'}=\frac{V_{estim}}{Y[m]}=\frac{259.6}{(1+\frac{s}{1\ Krad/s})*{[(1+\frac{s}{60\ Krad/s})]}^2*[1+(\frac{2*0.473}{94,5\ Krad/s})*s+(\frac{s}{94,5\ Krad/s})^2]}$\textbf{} &   \\ \hline 
\end{tabular}



\noindent Finalmente, despreciando los polos en alta frecuencia, la transferencia queda:

\noindent 

\begin{tabular}{|p{3.9in}|p{0.4in}|} \hline 
${Tlc'}=\frac{V_{estim}}{Y[m]}=\frac{259.6}{(1+\frac{s}{1\ Krad/s})*{(1+\frac{s}{60\ Krad/s})}^2}$ &   \\ \hline 
\end{tabular}


\subsection{\eject }

\noindent 
\subsection{4.3. Dise\~{n}o del Compensador Analogico}

\noindent Se plantea una compensaci\'{o}n como la que se muestra en la \textbf{figura 4.31} Est\'{a} compuesta por un lazo de control interno con un controlador por adelanto de fase para lograr estabilizar el sistema, y un lazo de control externo con un integrador para eliminar el error en r\'{e}gimen permanente.

\noindent \includegraphics*[width=6.26in, height=2.20in]{image39}

\noindent \textbf{Figura 4.31. }Diagrama del sistema completo

\noindent 

\noindent 
\paragraph{4.3.1. Dise\~{n}o de compensador por adelanto de fase}

\noindent A partir de las transferencias de la planta, el controlador de corriente y el estimador de posici\'{o}n, se realiz\'{o} el dise\~{n}o de un compensador anal\'{o}gico por el m\'{e}todo de adelanto de fase. Se lleg\'{o} a la siguiente transferencia:

\noindent 

\begin{tabular}{|p{3.9in}|p{0.4in}|} \hline 
$G_c(s)=10*{[20.346*\frac{(s+44.3)}{(s+902.1)}]}^2$ &  \\ \hline 
\end{tabular}



\noindent A continuaci\'{o}n se dise\~{n}a un circuito anal\'{o}gico correspondiente a este compensador.

\noindent 
\paragraph{4.3.2. Dise\~{n}o circuital}

\noindent 
\section{\includegraphics*[width=4.51in, height=3.42in]{image40}}

\noindent \textbf{Figura 4.32. }Dise\~{n}o circuital de una red de adelanto de fase.

\noindent 

\noindent Para cada etapa del compensador por adelanto, se utilizar\'{a} la topolog\'{i}a mostrada en la \textbf{Figura 4.32}. Consiste en  un polo y cero con ganancia unitaria (si $Ra\ =\ Rb$). Luego se agrega la ganancia como una etapa separada.

\noindent 

\noindent La transferencia de lazo cerrado de esta etapa es:

\noindent 

\begin{tabular}{|p{3.9in}|p{0.4in}|} \hline 
$\frac{Vout}{Vin}\ =\ -\ \frac{Ra}{Rb}\ *\frac{1\ +\ S*C*(Rx+R1)}{1\ +\ S*C*Rx}\ \ $ &  \\ \hline 
\end{tabular}



\noindent Por lo tanto, para tener polo=902.1 y zero=44.3, y eligiendo el capacitor C = 1uF, resulta $Rx\ =\ 1100\mathit{\Omega}$ y $R1\ =\ 21.5K\mathit{\Omega}$. Adem\'{a}s, se elige $Ra\ =\ Rb\ =\ 200k\mathit{\Omega}$ para ganancia unitaria. Luego, la ganancia del compensador se obtiene con una etapa amplificadora.

\noindent Para ello, se utiliza un amplificador operacional como se muestra en la \textbf{Figura 4.33. }Para lograr una ganancia de $K=10$ se utiliza $R_{322}=1K\mathit{\Omega}$ y $R_{323}=10K\mathit{\Omega}$..

\noindent \textbf{\includegraphics*[width=4.42in, height=2.89in]{image41}}

\noindent \textbf{Figura 4.33. }Etapa de ganancia del compensador.

\noindent 
\paragraph{4.3.3. Compensador con integrador}

\noindent 

\noindent Para el an\'{a}lisis se tiene que la realimentaci\'{o}n es 

\begin{tabular}{|p{3.9in}|p{0.4in}|} \hline 
$H_{estim}=\frac{V_{estim}}{Y[m]}=\frac{259.6}{(1+\frac{s}{1\ Krad/s})*{(1+\frac{s}{60\ Krad/s})}^2}$ &  \\ \hline 
\end{tabular}

 

\noindent Y la cadena de avance, con m=30 Kg e Yo=5mm, es

\noindent 

\begin{tabular}{|p{3.9in}|p{0.4in}|} \hline 
$G[m=30]=Tlc[m=30]*G_{Integrador}$ &  \\ \hline 
\end{tabular}



\noindent Podemos plantear un compensador del tipo :

\noindent 

\begin{tabular}{|p{3.9in}|p{0.4in}|} \hline 
$Gint\ =\ kint\ *\ \frac{1}{s/P_{int}+1}$ &  \\ \hline 
\end{tabular}



\noindent Por medio de la t\'{e}cnica de lugar de ra\'{i}ces y considerando $P_{int}=0.1\ rad/s$ se concluye que la ganancia del integrador que garantiza la estabilidad del sistema es Kint = 50.

\noindent 

\noindent 
\subparagraph{4.3.3.1. Implementaci\'{o}n circuital del integrador}

\noindent En la\textbf{ Figura 4.34} se puede observar la topolog\'{i}a y los valores utilizados en cada componente para el dise\~{n}o del circuito integrador. 

\noindent 

\noindent \includegraphics*[width=3.62in, height=3.27in]{image42}

\noindent \textbf{Figura 4.34. }Implementaci\'{o}n circuital del integrador.

\noindent 
\subparagraph{4.3.3.2. C\'{a}lculo de ganancia de entrada}

\noindent Tomando la tlc' que corresponde a la ganancia total de los bloques con el integrador ya incorporado, la ganancia resulta:

\noindent 

\begin{tabular}{|p{3.9in}|p{0.4in}|} \hline 
${Ganancia}_{tlc'}\textrm{⩬}-\frac{1}{H_{estim}}\textrm{⩬}-\frac{1}{260}$ &  \\ \hline 
\end{tabular}



\noindent \includegraphics*[width=6.27in, height=3.00in]{image43}

\noindent \textbf{Figura 4.35. }Diagrama en bloques final.

\noindent 

\noindent Por lo tanto teniendo tomando F=-1 y los rangos de posici\'{o}n de 1 mm a 5 mm como m\'{i}nimo y m\'{a}ximo respectivamente se llega a lo siguiente:

\noindent 

\begin{tabular}{|p{3.9in}|p{0.4in}|} \hline 
$Y[m]=F*-\frac{1}{260}*Vin=\frac{1}{260}*Vin\ $ &  \\ \hline 
\end{tabular}



\noindent La realimentaci\'{o}n tiene un set-point de 3.4 V por lo tanto se le suma a Vin el mismo valor.

\noindent 

\noindent Los valores finales son:

\noindent 

\begin{tabular}{|p{2.1in}|p{2.1in}|} \hline 
$Y[mm]$ & $Vin[V]$ \\ \hline 
5 & 4.7 \\ \hline 
4 & 4.44 \\ \hline 
3 & 4.18 \\ \hline 
2 & 3.92 \\ \hline 
\end{tabular}

\textbf{Tabla 4.5. }Tensi\'{o}n de referencia $[Vin]$ Vs separaci\'{o}n deseada $[Y]$.

\noindent 
\subparagraph{4.3.3.3. Implementaci\'{o}n circuital del bloque de ganancia de entrada ``F''}

\noindent Para poder modificar la distancia de separaci\'{o}n se ingresa al sistema con una tensi\'{o}n variable, la cual corresponde a una posici\'{o}n de referencia. Para ello se utiliza el circuito mostrado en la \textbf{figura 4.36}.

\noindent \includegraphics*[width=1.99in, height=2.93in]{image44}

\noindent \textbf{Figura 4.36. }Etapa de entrada.

\noindent 

\noindent Se utiliza una resistencia variable de $1K\mathit{\Omega}$ y dos fijas. Para poder excursionar la tensi\'{o}n de referencia entre 3.92V y 4.7V, los valores de las resistencias R1 y R3 deben ser de $4911\mathit{\Omega}$ y $313.5\mathit{\Omega}$ respectivamente. 

\noindent 

\noindent Por lo tanto, adoptando un valor comercial para ellas, resulta:
\[R1\ =\ 316\mathit{\Omega}\] 
\[R3\ =\ 4990\mathit{\Omega}\] 


\noindent De esta forma, los valores de tensi\'{o}n para la referencia de posici\'{o}n quedan:

\begin{enumerate}
\item  Tensi\'{o}n m\'{a}xima= 4.69V

\item  Tensi\'{o}n m\'{i}nima = 3.96V
\end{enumerate}

\noindent 
\section{}

\noindent \eject 

\noindent 
\section{5. Implementaci\'{o}n digital}

\begin{tabular}{|p{2.1in}|p{2.1in}|} \hline 
Requisito relacionado & Descripci\'{o}n \\ \hline 
RF01 & Mantener un objeto en suspensi\'{o}n \\ \hline 
RF02 & Implementar un controlador digital y anal\'{o}gico \\ \hline 
RF 05:  & Variar la distancia de separaci\'{o}n $Y_0$\newline  \\ \hline 
\end{tabular}

\textbf{Tabla 5.1. }RF.\textbf{ }

\noindent 
\subsection{5.1. Descripci\'{o}n general}

\noindent La implementaci\'{o}n digital consiste, b\'{a}sicamente, en realizar la estimaci\'{o}n de posici\'{o}n y el control de la planta por medio de un microcontrolador. Se utiliza un kit de desarrollo basado en el microcontrolador STM32F072, que contiene un ADC de 12 bits y 3.3V de referencia, un DAC 12 bits y 3.3V de referencia.

\noindent 

\noindent En la \textbf{Figura 5.1 }se muestra un diagrama en bloques general de la implementaci\'{o}n digital del sistema. Es posible observar que se ingresa al microcontrolador a trav\'{e}s de un ADC, con una tensi\'{o}n de referencia (Vref) proporcional a la distancia de separaci\'{o}n deseada. Esa posici\'{o}n de referencia es comparada con la posici\'{o}n estimada Y(z) y el resultado e(z) es afectado por el compensador digital C(z). Por medio de un DAC, la salida del compensador ingresa al controlador de corriente $G_{iL}(s)$, el cual act\'{u}a sobre la planta $G_P(s)$, modificando as\'{i} la distancia de separaci\'{o}n.

\noindent 

\noindent Por medio de un ADC y el sensor de Efecto Hall, se muestrea una tensi\'{o}n proporcional a la corriente que circula por el electroim\'{a}n. De esta forma, es posible obtener una posici\'{o}n estimada Y(z) al multiplicar esta tensi\'{o}n por la transferencia H(Z).\underbar{}

\noindent \underbar{\includegraphics*[width=6.22in, height=1.83in]{image45}}

\noindent \textbf{Figura 5.1.} Diagrama en bloques de la implementaci\'{o}n digital. 

\noindent 

\noindent Abstray\'{e}ndonos de la matem\'{a}tica que se realiza dentro del micro para la estimaci\'{o}n de posici\'{o}n, podemos simplificar el diagrama al que se muestra en la \textbf{Figura 5.2, }en la que$G_T(s)=G_P(s)*G_{iL}(s)$.

\noindent 

\noindent \includegraphics*[width=6.24in, height=2.33in]{image46}

\noindent \textbf{Figura 5.2}. Diagrama en bloques de la etapa digital simplificado.

\noindent 
\subsection{5.2. Determinaci\'{o}n de la frecuencia de muestreo}

\noindent Se desea realizar una estimaci\'{o}n de la posici\'{o}n del electroim\'{a}n Y(z)  a partir de las muestras tomadas por el ADC de la tensi\'{o}n de salida del sensor de efecto HALL.

\noindent 

\noindent La forma de onda de la salida del sensor es triangular y presenta una frecuencia variable en funci\'{o}n de la inductancia del electroim\'{a}n, la que, depende de la distancia de separaci\'{o}n. Se puede calcular como:

\noindent 
\begin{equation} \label{GrindEQ__5_1_} 
F_{SW}=\frac{V_{BUS}}{2*L(y)*\mathit{\Delta}i_H} 
\end{equation} 
Dadas las mediciones realizadas sobre el electroim\'{a}n, se obtuvieron los valores de inductancia al variar la distancia de separaci\'{o}n del entrehierro. Al aplicar en la ecuaci\'{o}n\textbf{ }5.1 los valores de inductancia obtenidos en la medici\'{o}n $(L[mHy])$, se calcula la frecuencia de conmutaci\'{o}n  $(f_{sw}[Hz])$. Los resultados se muestran en la \textbf{Tabla 5.3}.

\begin{tabular}{|p{1.4in}|p{1.4in}|p{1.4in}|} \hline 
$Y[mm]$ & $L[mHy]$ & $f_{sw}[Hz]$ \\ \hline 
\textbf{2} & 22.64 & 1060 \\ \hline 
\textbf{3} & 18.8 & 1276 \\ \hline 
\textbf{4.4} & 15.5 & 1548 \\ \hline 
\textbf{5.2} & 14.7 & 1632 \\ \hline 
\textbf{6.5} & 14.4 & 1666 \\ \hline 
\end{tabular}



\noindent \textbf{Tabla 5.3. }Valores de frecuencia calculados a partir de las mediciones de inductancia realizadas.

\noindent 

\noindent Para la estimaci\'{o}n de la posici\'{o}n es necesario medir la pendiente de la onda triangular. Por lo que, para reconstruir su forma de onda es necesario muestrear la se\~{n}al con cierta cantidad de arm\'{o}nicos para no afectar demasiado la pendiente. Se determin\'{o} que la frecuencia de muestreo del ADC debe ser al menos el doble de la frecuencia de la 5º arm\'{o}nica para el caso de la mayor frecuencia. Por lo tanto, se adopta 2.5 veces. Es decir:

\noindent 
\[F_S\ge 2.5*5*f_{max}\Rightarrow F_S\ge 2.5*5*1666\ Hz\Rightarrow F_S\ \ge 20825\ Hz\] 


\noindent De esta forma, se adopta una frecuencia de muestreo para el ADC de  25 kHz. Por lo tanto, es posible obtener 15 muestras en un per\'{i}odo de la triangular para el caso de la frecuencia m\'{a}xima. Como la se\~{n}al crece o decrece durante medio ciclo, se pueden tomar 7 muestras para identificar la pendiente. En el caso de que la se\~{n}al presente la frecuencia m\'{i}nima, se pueden tomar 23 muestras en un ciclo, lo cual se traduce en 11 muestras para la pendiente de subida o bajada. 

\noindent 

\noindent 
\subsection{5.3. Adquisici\'{o}n y procesamiento de las muestras}

\noindent Considerando el caso de m\'{a}xima frecuencia, en el que solo se podr\'{a}n tomar 7 muestras durante el tiempo de crecimiento o decrecimiento, se describe el procedimiento para determinar la posici\'{o}n estimada.

\noindent \includegraphics*[width=6.17in, height=4.18in]{image47}

\noindent 

\noindent \textbf{Figura 5.3}. Diagrama de flujo del procesamiento de las muestras adquiridas.

\noindent 

\noindent Como se observa en el diagrama de flujo de la \textbf{Figura 5.3,} cada muestra de tensi\'{o}n tomada del sensor de efecto hall, se almacena en un buffer de 7 posiciones. Para poder discernir entre pendientes de bajada y de subida, se verifica en cada muestra si el valor le\'{i}do es mayor o menor al almacenado en la posici\'{o}n anterior. En caso de que sea mayor al anterior, significa que se est\'{a} muestreando la pendiente positiva de la onda triangular. El hecho de poder discernir entre pendientes positivas y negativas, permite aplicar la compensaci\'{o}n I*R al igual que se realiza en el estimador anal\'{o}gico. 

\noindent 

\noindent Cada vez que el buffer se complete, se realiza el c\'{a}lculo de la derivada con el valor m\'{a}ximo y m\'{i}nimo almacenado. Con este resultado, se hace la estimaci\'{o}n de la posici\'{o}n y se actualiza la entrada al compensador digital.

\noindent 

\noindent En caso de haber completado las 7 posiciones del buffer y la pendiente persiste con el mismo signo, el buffer comienza a llenarse nuevamente desde la posici\'{o}n inicial, sobreescribiendo los valores de mayor vejez. Por lo tanto, pueden ocurrir dos situaciones. La primera es que se detecte un cambio de pendiente antes de completar nuevamente el buffer, con lo cual se calcula la derivada con los valores extremos almacenados sin importar su vejez y se actualiza la entrada al compensador. La segunda, es que se vuelva a completar el buffer, en cuyo caso tambi\'{e}n se hace la actualizaci\'{o}n. La diferencia entre estas dos situaciones es el tiempo transcurrido. En este \'{u}ltimo, se hace cada 7 per\'{i}odos de muestreo mientras que en el primero se realiza ``N'' per\'{i}odos de muestreo luego de la \'{u}ltima actualizaci\'{o}n, donde ``N'' representa la cantidad de muestras que se almacenaron en el buffer incompleto.

\noindent 

\noindent Luego de cada actualizaci\'{o}n, el proceso vuelve a iniciar con el buffer vac\'{i}o.

\noindent 

\noindent Utilizando este m\'{e}todo de estimaci\'{o}n, puede ocurrir que se obtenga una nueva estimaci\'{o}n en 7 periodos de muestreo del ADC, o incluso en menos. Por lo tanto se podr\'{i}a decir que se tiene un estimador de posici\'{o}n con frecuencia de actualizaci\'{o}n variable. Esto es importante al momento de dise\~{n}ar un compensador digital para el sistema. Para hacerlo, se debe considerar el caso en que la frecuencia de actualizaci\'{o}n es la menor, por lo tanto podr\'{i}amos decir que el compensador digital se debe dise\~{n}ar con una frecuencia de muestreo de 25/7 KHz = 3.5 KHz.

\noindent 
\subsection{5.4. Estimaci\'{o}n digital de la posici\'{o}n}

\noindent De las mediciones realizadas se lleg\'{o} a la expresi\'{o}n que relaciona la distancia de separaci\'{o}n con la pendiente de la corriente en el electroim\'{a}n:

\noindent 
\[\left|\frac{{di}_L}{dt}\right|\ [A/s]\ =\ 194690\ *\ Y[m]\ +\ \ 676\ [A/s]\] 


\noindent Por lo tanto, la posici\'{o}n en metros puede despejarse como$:$

\noindent 
\[Y=5.136*10^{-6}*\left|\frac{{di}_L}{dt}\right|\ -3.472*10^{-3}\ \ [m]\] 


\noindent Es importante notar que la resistencia interna (R) del electroim\'{a}n genera una ca\'{i}da de tensi\'{o}n cuando circula corriente. Esta ca\'{i}da provoca que la tensi\'{o}n efectiva aplicada sobre la inductancia sea distinta para el semiciclo de subida que el de bajada, generando que la onda triangular tenga diferentes pendientes (en valor absoluto) para cada caso. Esta se representa como$\ (\frac{{di}_L}{dt})_{Real}$ y es la que se mide al utilizar el ADC.

\noindent 

\noindent Es decir:

\noindent 
\[(\frac{{di}_L}{dt})_{Real}=(\frac{{di}_L}{dt})_{Te\textrm{\'{o}}rica}-\frac{{R*i}_L}{L(y)}\ \] 


\noindent Aproximando la derivada real como la resta entre la muestra en un instante menos el anterior sobre el per\'{i}odo de muestreo y compensando el error que introduce la resistencia interna, se obtiene:

\noindent 
\[Y=5.136*10^{-6}\left|\frac{I_L\ [n]-I_L\ [n-1]}{T_s}+\frac{{R*i}_L}{L(y)}\right|-3.472*10^{-3}\ \ [m]\ \] 


\noindent Considerando a $V_h$ como la tensi\'{o}n entregada por el sensor de efecto hall, proporcional a la corriente que circula por el electroim\'{a}n multiplicada por una ganancia Kh de 53.3mV/A, donde  ($\widehat{V_h}$)  corresponde a la componente alterna de tensi\'{o}n y ($\underline{V_h}$) la continua, resulta:

\noindent 
\[v_h[n]=\underline{V_h}[n]+\widehat{V_h}[n]=Kh*(\underline{I_L}[n]+\widehat{I_L}[n])\ \] 


\noindent Para la estimaci\'{o}n de la posici\'{o}n se utiliza el t\'{e}rmino de alterna mientras que para compensar el error introducido por la resistencia interna del electroim\'{a}n se utiliza el de continua. Por lo tanto, se obtiene:

\noindent 
\[Y=5.136*10^{-6}*\left|\frac{\widehat{V_h}[n]-\widehat{V_h}[n-1]}{kh*T_s}+\frac{R*\underline{V_h[}n]}{K_h*L(y)[n-1]}\right|-3.472*10^{-3}\ \ [m]\ \] 


\noindent El valor de corriente $\underline{V_h}[n]$ se obtiene de sensar el valor medio de tensi\'{o}n entregado por el sensor de efecto Hall mediante otro canal del ADC.

\noindent 

\noindent Por otro lado, el valor de $L(y)[n-1]$ se obtiene de aplicar el valor anterior estimado de posici\'{o}n en la ecuaci\'{o}n\textbf{ 5.2}. El c\'{a}lculo de esta expresi\'{o}n se obtiene a partir de la linealizaci\'{o}n de la inductancia en funci\'{o}n de las mediciones realizadas sobre el electroim\'{a}n.

\noindent 
\begin{equation} \label{GrindEQ__5_2_} 
L(y)[n-1]\ =-2.56\ *\ Y[n-1]+\ \ 0.0271\ Hy 
\end{equation} 


\noindent Por lo tanto la ecuaci\'{o}n correspondiente en el tiempo discreto:

\noindent 
\[Y[n]=5.136*10^{-6}*\left|\frac{\widehat{V_h}[n]-\widehat{V_h}[n-1]}{kh*T_s}+\frac{R*\underline{V_h[}n]}{K_h*(2.56\ *\ Y[n-1]\ +\ \ 0.0271)}\right|-3.472*10^{-3}\ \ [m]\ \] 

\[Y[n]=96.3*10^{-6}*\left|\frac{\widehat{V_h}[n]-\widehat{V_h}[n-1]}{T_s}+\frac{R*\underline{V_h[}n]}{(2.56\ *\ Y[n-1]\ +\ \ 0.0271)}\right|-3.472*10^{-3}\ \ [m]\ \] 


\noindent Donde n representa el n\'{u}mero de muestras. Es decir, $V_h[n]$ se refiere a la muestra m\'{a}s reciente en el buffer y $V_h[n-1]$ a la m\'{a}s vieja.

\noindent 

\noindent Es importante notar que los coeficientes deben calcularse antes de actualizar el compensador en funci\'{o}n de la cantidad de per\'{i}odos de muestreo transcurridos desde la \'{u}ltima actualizaci\'{o}n. Estos deben calcularse en ese momento puesto que el compensador digital presenta una frecuencia de actualizaci\'{o}n variable y los coeficientes del estimador dependen de ella.

\noindent 

\noindent Por otro lado, el bloque K mostrado en la\textbf{ Figura 5.2} resulta en una transferencia unitaria.

\noindent 
\subsection{5.5. Resoluci\'{o}n en posici\'{o}n}

\noindent Una variaci\'{o}n de posici\'{o}n ($\mathit{\Delta}Y$) produce un cambio de inductancia ($\mathit{\Delta}L[y]$) que se traduce en un cambio de frecuencia ($\mathit{\Delta}f$). Para poder detectar el m\'{i}nimo cambio de posici\'{o}n en un per\'{i}odo de muestreo se debe tener una resoluci\'{o}n tal que permita discernir ese cambio de frecuencia.

\noindent 

\noindent A partir de los valores de inductancia  obtenidos con las mediciones, es posible realizar una aproximaci\'{o}n lineal como se muestra en la ecuaci\'{o}n\textbf{ 5.3.}

\noindent 
\begin{equation} \label{GrindEQ__5_3_} 
L[Hy]\ =\ -2.56\ *\ Y[m]\ +\ \ 0.0271\ Hy 
\end{equation} 


\noindent Aplicando la expresi\'{o}n linealizada de la inductancia y la ecuaci\'{o}n\textbf{ 5.1 }es posible obtener el valor de frecuencia para una separaci\'{o}n de $Y=2.1mm$. Este resulta de$f_{sw}(2.1mm)=1104.8Hz$. De esta forma, conociendo el valor de frecuencia para $2\ mm$, el cual es de $f_{SW}(2mm)=1060Hz$, es posible obtener la variaci\'{o}n de frecuencia para un $\triangle Y$ m\'{i}nimo de 0.1mm. Este valor puede obtenerse como:

\noindent  
\[{\triangle Fsw(te\textrm{\'{o}}rico)=f}_{SW}(2.1mm)-f_{SW}(2mm)=44,8\ Hz\] 


\noindent Las pendientes para el peor caso se da con la menor variaci\'{o}n de tensi\'{o}n entre muestras. Es decir, para el caso de frecuencia m\'{i}nima. En la ecuaci\'{o}n\textbf{ 5.4 }se muestra el c\'{a}lculo de la pendiente de la onda triangular en funci\'{o}n de la frecuencia de conmutaci\'{o}n

\noindent .
\begin{equation} \label{GrindEQ__5_4_} 
P(f_{SW})\ =\frac{\mathit{\Delta}V}{T_{SW}/2}=2*K_H*\mathit{\Delta}i_L*f_{SW}=2*0.0533*0.5{*f}_{SW} 
\end{equation} 


\noindent A partir de la ecuaci\'{o}n\textbf{ 5.4 }es posible obtener el valor de la pendiente para la frecuencia m\'{i}nima de conmutaci\'{o}n y la de su incremento correspondiente a una variaci\'{o}n en la posici\'{o}n de 0.1mm. Esta situaci\'{o}n se representa en la\textbf{ Figura 5.4}.

\noindent 
\[\ P({Fsw}_{min})=56.49\ [V/s]\] 
\[\ P({Fsw}_{min}+\triangle Fsw)=58,89\ [V/s]\] 


\noindent \includegraphics*[width=3.50in, height=2.63in]{image48}

\noindent \textbf{Figura 5.4. }Variaci\'{o}n de pendiente ante m\'{i}nimo cambio de posici\'{o}n.

\noindent 

\noindent Por lo tanto, para poder diferenciar las pendientes, la resoluci\'{o}n del ADC debe ser menor o igual a DeltaV.

\noindent 
\[V1=P({Fsw}_{min}+\triangle Fsw)*Ts\] 
\[V2=P({Fsw}_{min})*Ts\] 


\noindent Al considerar $Ts\ =\ \frac{1}{25\ kHz}$:

\noindent 
\[\triangle VADC=Ts*[P({Fsw}_{min}+\triangle Fsw)-P({Fsw}_{min})]=96\ uV\] 


\noindent Este resultado indica que al usar un ADC de 12 bits, se necesitar\'{i}a una tensi\'{o}n de referencia Vref = 0.393216 V. Sin embargo, este valor resulta demasiado bajo y no sirve si se quiere medir la salida del sensor de efecto Hall de manera directa. Por lo tanto, se decide dise\~{n}ar un circuito que permita realizar la estimaci\'{o}n manteniendo la tensi\'{o}n de referencia en 3.3V

\noindent 

\noindent La corriente que circula por el electroim\'{a}n presenta una componente de continua y otra de alterna. La primera excursiona entre 0A y 30A mientras que la segunda var\'{i}a entre $\pm 250\ mA$ en torno al valor medio, con forma de onda triangular. Es posible hacer una adquisici\'{o}n separada de ambas componentes con el ADC para que luego sean procesadas. La se\~{n}al que ingresa al circuito corresponde a la tensi\'{o}n de salida del sensor de efecto Hall sin el set point de 2.5V.

\noindent 

\noindent Si se tiene en cuenta la ganancia del sensor de efecto Hall, a su salida se obtiene una se\~{n}al cuyo valor medio var\'{i}a entre 0 y 1.6 V, y un valor de alterna de 26.7 mVpp.

\noindent 

\noindent Debido a que el ADC permite una excursi\'{o}n entre 0V y 3.3V, la m\'{a}xima ganancia posible es de 60 veces para la se\~{n}al de alterna. Por otro lado, para medir con la resoluci\'{o}n en posici\'{o}n deseada de 0.1 mm se debe amplificar la se\~{n}al triangular 9 veces como m\'{i}nimo.

\noindent  

\noindent Por lo tanto, se adopta una ganancia de 50, obteniendo as\'{i} una excursi\'{o}n m\'{a}xima de 3.17V (0.67V sobre el set-point).

\noindent 

\noindent Las caracter\'{i}sticas del circuito son:

\begin{enumerate}
\item  Ganancia: 50

\item  Set-point de 2.5V 

\item  Frecuencia de corte inferior: 100 Hz

\item  Frecuencia de corte superior: 12,5 kHz 
\end{enumerate}

\noindent 

\noindent Teniendo en cuenta la ganancia elegida,  la pendiente de la onda triangular resulta:

\noindent 
\[\ P(Fsw)=50*[0.0533*0.5*(Fsw*2)][\frac{V}{s}]\] 


\noindent Reemplazando para el incremento de frecuencia se obtiene los valores 

\noindent 

\noindent $\ P({Fsw}_{min})=2824.9$ y $\ P({Fsw}_{min}+\triangle Fsw)=2944.29\ [\frac{V}{s}]$           

\noindent 

\noindent Entonces, 

\noindent 
\[\triangle VADC=Ts*[P({Fsw}_{min}+\triangle Fsw)-P({Fsw}_{min})]=0.1177V\ -\ 0.1129V\ =\ 4.7mV\] 


\noindent Por lo tanto, como la resoluci\'{o}n del ADC es de 0.8mV, resulta suficiente para identificar el m\'{i}nimo cambio de pendiente.

\noindent 
\subsection{5.6. Acondicionamiento de se\~{n}ales para el ADC}

\noindent 
\paragraph{5.6.1. Referencia de posici\'{o}n}

\noindent 

\noindent Para indicar al microcontrolador la distancia de separaci\'{o}n deseada se utiliza una se\~{n}al continua como referencia que se ajusta desde un potenci\'{o}metro ubicado en el PCB (al igual que para el compensador anal\'{o}gico) e ingresa al circuito mostrado en la \textbf{Figura 5.5}. Debido a que entrega una tensi\'{o}n entre 3.96V y 4.69V, se implementa un circuito de acondicionamiento para esta se\~{n}al.

\noindent 

\noindent A la se\~{n}al de entrada se le resta el setpoint de 2.5 V, para lograr se\~{n}ales que van desde 1.42V a 2.2V. Luego dentro del microcontrolador se debe mapear el valor le\'{i}do por el ADC con la posici\'{o}n deseada usando la ganancia del estimador anal\'{o}gico seg\'{u}n la f\'{o}rmula:

\noindent 
\[Y_{ref}\ =\frac{Vpo{s_{ref}}_{ADC}\ +\ 2.5}{259.6}\ [m]\] 


\noindent Adem\'{a}s se implementa un filtro anti-aliasing con frecuencia de corte en 9.9 kHz.

\noindent \includegraphics*[width=6.27in, height=2.16in]{image49}

\noindent \textbf{Figura 5.5. }Circuito acondicionador para la referencia de posici\'{o}n.

\noindent 
\paragraph{5.6.2. Componente  alterna de corriente del electroim\'{a}n}

\noindent Para obtener solamente la componente alterna de la corriente, se implementa un circuito con caracter\'{i}stica pasa-banda que se muestra en la F\textbf{igura 5.6}. La frecuencia de corte inferior  es de 100Hz, con el objetivo de eliminar el valor medio de se\~{n}al. Por otro lado, la superior es de 12 KHz, que act\'{u}a como filtro anti-aliasing. Luego la salida es amplificada con una ganancia de 50 veces (con el objetivo de mejorar la medici\'{o}n de la pendiente por el ADC) y montada sobre un set-point de 2.5V.

\noindent 

\noindent \includegraphics*[width=6.28in, height=2.37in]{image50}

\noindent \textbf{Figura 5.6. }Circuito acondicionador para componente alterna de corriente del electroim\'{a}n.\textbf{}

\noindent 
\paragraph{5.6.3. Componente  continua de corriente del electroim\'{a}n}

\noindent Para obtener la componente de cont\'{i}nua se utiliza un filtro pasa-bajos con frecuencia de corte en 106 Hz. Se eligi\'{o} esta frecuencia para que se ubique por lo menos una d\'{e}cada por debajo de la frecuencia fundamental de la onda triangular. La implementaci\'{o}n circuital puede observarse en la \textbf{Figura 5.7.}

\noindent 
\subsection{\includegraphics*[width=6.29in, height=2.11in]{image51}}

\noindent \textbf{Figura 5.7}. Circuito acondicionador para componente continua de corriente del electroim\'{a}n.

\noindent 
\subsection{5.7. Acondicionamiento de se\~{n}ales para el DAC}

\noindent Para convertir los valores digitales de la estimaci\'{o}n de posici\'{o}n y de la compensaci\'{o}n al dominio anal\'{o}gico, se utilizan los DAC del microcontrolador. La tensi\'{o}n entregada es afectada por una circuiter\'{i}a de filtrado, ganancia y protecci\'{o}n como se muestra en la \textbf{Figura 5.8 }y \textbf{5.9. }Debido a que el DAC se actualiza con una frecuencia m\'{i}nima de 3.5 KHz, se utilizan filtros con frecuencia de corte en 1.75KHz.

\noindent 

\noindent Por otro lado, como el controlador de corriente funciona con tensiones de hasta $5\ V$ en su entrada y el compensador fue dise\~{n}ado teniendo en cuenta este nivel de tensi\'{o}n, se agrega una ganancia por firmware de $0,66$, mapeando as\'{i} los $5\ V$ a $3,3\ V$, que es la m\'{a}xima tensi\'{o}n entregada por el DAC. Luego, para compensar esta ganancia y no afectar a la transferencia de la planta, se la afecta por un factor de de $\frac{5V}{3.3V}$ por medio del circuito de acondicionamiento.

\noindent 

\noindent De esta forma, se logra convertir correctamente la se\~{n}al digital en anal\'{o}gica.

\noindent \textbf{}

\noindent \textbf{\includegraphics*[width=4.60in, height=2.87in]{image52}}

\noindent \textbf{Figura 5.8. }Circuito acondicionador para la salida del DAC correspondiente al compensador. 

\noindent \textbf{\includegraphics*[width=6.30in, height=2.60in]{image53}}

\noindent \textbf{Figura 5.9.}Circuito acondicionador para la salida del DAC correspondiente al estimador digital.

\noindent 
\subsection{5.8. Transferencias de la planta y del controlador de corriente}

\noindent Para el an\'{a}lisis del compensador digital, se parte de las transferencias de la planta $G_P(s)$ y del controlador de corriente $G_{iL}(s)\ $ en dominio anal\'{o}gico para una masa de 30 Kg.

\noindent 
\[G_T(s)[30Kg]=G_P(s)*G_{iL}(s)=\frac{245*{10}^{-6}}{{1-(\frac{s}{70})}^2}*\frac{6}{\frac{S}{12.17\ }+1\ }=\frac{-87.7}{\ (s-70)\ (s+70)\ (s+12.17)}\] 


\noindent Al aplicar la transformada z por invarianza al impulso, considerando una $f_s=3.5\ KHz$, se obtiene:

\noindent 
\[G_T(Z)[m=30\ Kg]\ \ =\frac{-3.4*10^{-10}(z+3.7)(z+0.3)}{\ (z-0.9965)\ (z+0.9802)\ (z+0.2677)}\] 


\noindent Luego, usando la transformada bilineal para volver al dominio anal\'{o}gico:

\noindent 
\[G_T(w)[m=30\ Kg]\ \ =\frac{-8.5*10^{-11}(w-1.21*10^4)(w-7000)(w+1.21*10^4)}{\ (w-70)\ (w+70)\ (w+12.17)}\] 


\noindent Con las expresiones en [W], es posible dise\~{n}ar un controlador de manera anal\'{o}gica, para luego transformarlo al dominio digital.

\noindent 
\subsection{5.9. Dise\~{n}o de Compensador}

\noindent 
\paragraph{5.9.1. An\'{a}lisis de estabilidad con masa de 30 Kg}

\noindent Considerando que la ganancia de avance est\'{a} formada por la planta y el controlador de corriente y que el lazo de realimentaci\'{o}n es unitario, se procede a analizar la respuesta en frecuencia de ${GH}_T$ para una masa de 30 Kg y a dise\~{n}ar un compensador adecuado. Luego, se verifica la estabilidad para una masa de 1 Kg, que corresponde a la m\'{i}nima con la que trabaja el sistema.

\noindent 
\begin{equation} \label{GrindEQ__5_5_} 
G_T(w)*H(w)=\frac{-8.5*10^{-11}(w-1.21*10^4)(w-7000)(w+1.21*10^4)}{\ (w-70)\ (w+70)\ (w+12.17)} 
\end{equation} 


\noindent Con la transferencia de la \textbf{ecuaci\'{o}n} \textbf{ 5.5 }se  grafica el diagrama de Bode y el diagrama de Nyquist que se muestran en las \textbf{Figuras 5.10} y \textbf{5.11} respectivamente.

\noindent 

\noindent \includegraphics*[width=5.35in, height=4.31in]{image54}

\noindent \textbf{Figura 5.10}. Diagrama de Bode de lazo abierto ${GH}_T$ con M=30 Kg.

\noindent \includegraphics*[width=5.33in, height=4.17in]{image55}

\noindent \textbf{Figura 5.11}. Diagrama de Nyquist de ${GH}_T$ con M=30 Kg.

\noindent 

\noindent Considerando que ${GH}_T$ tiene un polo en el semiplano derecho, a partir del Nyquist se puede determinar:

\noindent 

\noindent Zona 1: Z=N+P=0+1=1 $\mathrm{\to}$ Inestable 

\noindent 

\noindent Zona 2: Z=N+P=1+1=2 $\mathrm{\to}$ Inestable

\noindent 

\noindent De esta forma, no es posible que el sistema sea estable. Para lograrlo se realimentar\'{a} positivamente y se generar\'{a} una zona en el diagrama de Nyquist donde N=-1. Para ello es necesario aumentar la fase para que pueda superar el valor de 0$\mathrm{{}^\circ}$.  Para que esto se cumpla, el diagrama de Nyquist deber\'{i}a tener una forma como la  mostrada en la \textbf{Figura 5.12}.

\noindent 

\noindent \includegraphics*[width=4.25in, height=3.40in]{image56}

\noindent \textbf{Figura 5.12}. Forma del diagrama de Nyquist deseado.

\noindent 

\noindent Para poder lograr el aumento de fase mencionado se utiliza una red de adelanto de fase. Se debe tener en cuenta que el m\'{o}dulo de la transferencia de lazo abierto en el primer cruce de la fase por 0$\mathrm{{}^\circ}$ debe ser mayor a 0 dB y, en el segundo cruce, menor. De esta forma, al observar la \textbf{Figura 5.10} se decide adelantar la fase 100º en aproximadamente 200 rad/s. Esto se logra usando dos redes de adelanto de fase de 65$\mathrm{{}^\circ}$ cada una.

\noindent 

\noindent Ecuaciones de dise\~{n}o:

\noindent 
\[W_0=200\ r/s\] 

\[{\varphi }_{max}=65\textrm{º}\] 

\[\alpha =\frac{1+sen{\varphi }_{max}}{1-sen{\varphi }_{max}}=20.346491\] 

\[W_c=\frac{W_0}{\sqrt{\alpha }}=\ 44.3\ r/s\] 

\[W_p=\sqrt{\alpha }*W_0=902.1\ r/s\] 


\noindent Finalmente se llega a la transferencia del controlador:

\noindent 
\[G_c(s)=K*{[20.346*\frac{(s+44.3)}{(s+902.1)}]}^2\] 


\noindent En la \textbf{Figura 5.10 }se muestra el diagrama de bode de ${GH}_T*G_C$ con $K=1$. Se puede observar que la ganancia $K$ puede adoptar valores desde 64 dB hasta 89.5 dB. Considerando que el sistema debe soportar una masa variable entre 1 kg y 30 kg, y que la ganancia de la transferencia de la planta para 1 kg es de 5.5 veces (14 dB) mayor que para 30 kg, se puede adoptar una ganancia del compensador que mantenga la estabilidad para estos dos casos. Es decir, la ganancia m\'{i}nima es de 64 dB y la m\'{a}xima es de 89.5 dB - 14 dB = 75.5 dB. Por lo tanto, se elige que el cruce por cero de la ganancia se encuentre ahora en 88 rad/s, lo que significa que $K=68.4dB\ \equiv \ 2630\ veces$.

\noindent 

\noindent \includegraphics*[width=5.45in, height=4.27in]{image57}

\noindent \textbf{Figura 13. }Diagrama de Bode de ${GH}_T*G_C$ para K=1 y M=30 Kg.

\noindent 

\noindent En la \textbf{Figura 5.13} se muestra el diagrama de Bode considerando la ganancia del compensador. En ella se puede observar que se  cumple con el criterio de estabilidad, puesto que en el primer cruce por 0º, la magnitud es mayor a 0 dB y en el segundo cruce, menor. Adem\'{a}s, en la \textbf{Figura 5.14} se puede ver que la forma del diagrama de Nyquist es como la deseada.

\noindent \includegraphics*[width=4.79in, height=3.71in]{image58}

\noindent \textbf{Figura 5.14. }Diagrama de Bode de ${GH}_T*G_C$ para K=2630 y M=30 Kg.

\noindent 

\noindent \includegraphics*[width=4.72in, height=3.82in]{image59}

\noindent \textbf{Figura 5.15}. Diagrama de Nyquist de ${GH}_T*G_C$ para K=2630 y M=30 Kg.

\noindent 

\noindent En la \textbf{Figura 5.16} se puede observar la respuesta al escal\'{o}n del sistema con masa de 30 Kg.

\noindent \includegraphics*[width=4.53in, height=3.68in]{image60}

\noindent \textbf{Figura 5.16}. Respuesta al escal\'{o}n para M=30 Kg.

\noindent 

\noindent 

\noindent 
\paragraph{5.9.2. An\'{a}lisis de estabilidad con masa de 1 Kg}

\noindent 

\noindent En esta secci\'{o}n se verifica la estabilidad del sistema  para el caso en que la masa sea de 1 Kg, utilizando el compensador dise\~{n}ado para el caso de masa m\'{a}xima. Para ello, se analizan los diagramas de Bode y Nyquist mostrados en las \textbf{Figuras 5.17 }y \textbf{5.18.} Adem\'{a}s, en la \textbf{Figura 5.19 }puede observarse la respuesta al escal\'{o}n. A partir de ellos, es posible verificar que efectivamente el sistema resulta estable para todo el rango de masas en el que opera el sistema. 

\noindent \includegraphics*[width=4.54in, height=3.61in]{image61}

\noindent \textbf{Figura 5.17. }Diagrama de Bode de ${GH}_T*G_C$ para M=1 Kg.\textbf{}

\noindent \textbf{}

\noindent \includegraphics*[width=4.51in, height=3.59in]{image62}

\noindent  \textbf{Figura 5.18. }Diagrama de Nyquist de ${GH}_T*G_C$ para M=1 Kg.\textbf{}

 

\noindent 

\noindent 

\noindent \includegraphics*[width=4.59in, height=3.76in]{image63}

\noindent \textbf{Figura 5.19.} Respuesta al escal\'{o}n para M=1 Kg.

\noindent 
\subsection{5.10.  Dise\~{n}o de lazo de realimentaci\'{o}n externo}

\noindent 

\noindent Se plantea un lazo de realimentaci\'{o}n externo como se muestra en la  \textbf{figura 5.20.}

\noindent \includegraphics*[width=6.21in, height=2.38in]{image64}

\noindent 

\noindent \textbf{Figura 5.20. }Diagrama del sistema completo.

\noindent 

\noindent En el lazo de realimentaci\'{o}n interno act\'{u}a el compensador por adelanto de fase dise\~{n}ado previamente y, en el externo, un controlador del tipo integral. De esta forma, se logra suavizar la respuesta al escal\'{o}n del sistema y eliminar el error en r\'{e}gimen permanente.

\noindent 

\noindent Para el an\'{a}lisis se considera como realimentaci\'{o}n: 

\noindent 
\[H(w)=1\] 

  

\noindent La cadena de avance con masa de 30 Kg es:

\noindent 
\[G[m=30]=Tlc(W)[m=30]*G_{Integrador}\] 


\noindent Se  plantea un compensador del tipo :

\noindent 
\[Ginteg\ =\ kint\ *\ \frac{1}{w}\] 


\noindent La ganancia del bloque de entrada (F) se establece igual a la ganancia del estimador (H) pero cambiada de signo, debido a que la transferencia de lazo cerrado tiene una inversi\'{o}n de fase. Por lo tanto, se toma $F=-H=-1$.

\noindent 

\noindent Inicialmente se adopta kint = 1 para poder evaluar, por medio de lugar de ra\'{i}ces mostrado en la \textbf{Figura 5.21}, la estabilidad del sistema. Para este lazo de realimentaci\'{o}n externo tambi\'{e}n debe utilizarse realimentaci\'{o}n positiva, puesto que los polos de la TLC interna est\'{a}n en el semiplano izquierdo pero presenta una inversi\'{o}n de signo.

\noindent 

\noindent \includegraphics*[width=5.43in, height=4.25in]{image65}

\noindent \textbf{Figura 5.21. }Lugar de ra\'{i}ces con el integrador.

\noindent 

\noindent En la \textbf{Figura 5.21 } se puede observar que, para que se mantenga la estabilidad del sistema, la ganancia del integrador ($K_{int\ }$) debe ser menor a 115. Teniendo esto en cuenta, en la \textbf{Figura 5.22} se muestra la respuesta al escal\'{o}n del sistema compensado con el integrador para una ganancia de $K_{int\ }=1$.  Es posible observar que, si bien no presenta oscilaciones, el tiempo de establecimiento es de aproximadamente 3 segundos. Por lo tanto, se decide aumentar el valor de ganancia hasta obtener una relaci\'{o}n aceptable entre el tiempo de respuesta y el sobrepico.

\noindent 

\noindent \includegraphics*[width=3.97in, height=3.30in]{image66}

\noindent \textbf{Figura 5.22. }Respuesta al escal\'{o}n con integrador con $K_{int\ }=1$ y M=30 Kg.

\noindent 

\noindent En la \textbf{Figura 5.23}, se observa la respuesta al escal\'{o}n para una ganancia del integrador de $K_{int\ }=20$ que resulta en un tiempo de establecimiento de 0.22 segundos y un overshoot de 4.41\%. Por lo tanto, se adopta este valor de ganancia para el dise\~{n}o del integrador.

\noindent 

\noindent \includegraphics*[width=4.33in, height=3.47in]{image67}

\noindent \textbf{Figura 5.23. }Respuesta al escal\'{o}n con integrador para $K_{int\ }=20$ y M = 30 Kg.

\noindent 

\noindent La respuesta al escal\'{o}n cuando la masa es de 1 Kg se muestra en la \textbf{Figura 5.22}. All\'{i} se puede observar que el tiempo de crecimiento es de 0.104s y el de establecimiento de 0.196s. Adem\'{a}s, es posible notar que no presenta sobrepicos.

\noindent 

\noindent \includegraphics*[width=4.40in, height=3.73in]{image68}

\noindent \textbf{Figura 5.22. }Respuesta al escal\'{o}n con integrador para $K_{int\ }=20$ y M = 1 Kg.

\noindent 
\subsection{5.11.  C\'{a}lculo de los coeficientes del controlador}

\noindent Para implementar el algoritmo de control en el microcontrolador se aplica la transformada bilineal inversa a las transferencias del compensador por adelanto de fase $C(W)$ y al integrador $G_{integ}(W)$.\textbf{}

\noindent 

\noindent Por lo tanto, se obtiene:

\noindent 
\begin{equation} \label{GrindEQ__5_6_} 
C(Z)=\ \ \frac{U(z)}{e(z)}=\frac{8.6896*10^5(z-0.9877)^2}{\ (z-0.7757)^2}\  
\end{equation} 

\begin{equation} \label{GrindEQ__5_7_} 
G_{integrador}(Z)=\frac{e_f(z)}{e_i(z)}\ =\frac{0.0028(\ z\ +\ 1)}{\ (z\ -\ 1)} 
\end{equation} 


\noindent Considerando $H(z)=1$ se obtiene que:

\noindent 
\begin{equation} \label{GrindEQ__5_39_} 
e(z)=e_f(Z)+Y(z) 
\end{equation} 

\begin{equation} \label{GrindEQ__5_40_} 
e_i(Z)=F*Vref+Y(z) 
\end{equation} 


\noindent Al aplicar la partir de las \textbf{ecuaciones 5.6 }y\textbf{ 5.7 }se obtiene las expresiones a implementar en el microcontrolador:

\noindent 
\[U[n]=8.651*10^5e[n]-\ 1.709*10^6e[n-1]+\ 0.843*10^6e[n-2]+1.5514\ U[n-1]-\ 0.60171U[n-2]\ \ \ (5.8)\] 

\begin{equation} \label{GrindEQ__5_9_} 
e_f[n]=0.0028{\ *e}_i[n]\ +\ {0.0028*e}_i[n-1]+e_f[n-1] 
\end{equation} 


\noindent Luego, para dejar el algoritmo de control en funci\'{o}n de las entradas del sistema, se debe reemplazar en las \textbf{ecuaciones 5.8} y\textbf{ 5.9} las expresiones mencionadas en las \textbf{ecuaciones 5.10} y \textbf{5.11}

\noindent 
\begin{equation} \label{GrindEQ__5_10_} 
e[n]=e_f[n]+Y[n] 
\end{equation} 

\begin{equation} \label{GrindEQ__5_11_} 
e_i[n]=F*Vref+Y[n] 
\end{equation} 

\subsection{5.12.  Conexi\'{o}n entre el PCB y el microcontrolador}

\noindent Se utiliza un conector tipo DB9 hembra como v\'{i}a de conexi\'{o}n para las distintas salidas y entradas digitales. Adem\'{a}s, en la placa se dispone de un led que se enciende cuando  se detecta una correcta conexi\'{o}n con el microcontrolador.

\noindent 

\noindent 
\section{\eject }

\noindent 
\section{6. Fuentes de Alimentaci\'{o}n}

\noindent 
\subsection{6.1. Fuente de alimentaci\'{o}n externa de 24V.}

\noindent La fuente externa se encarga de alimentar todo el circuito. Debe ser capaz de suministrar 24V y hasta 30A. Para ello se puede utilizar una fuente de laboratorio o bater\'{i}as.

\noindent 
\subsection{6.2 Fuente de alimentaci\'{o}n interna de 12V}

\noindent La fuente de 12V se encarga de alimentar al regulador de 5V y al driver del puente H. Debido a los bajos consumos de potencia y bajo costo, se utiliza una fuente lineal. Por lo tanto, se decide utilizar el integrado L78M12CDT-TR, el cual presenta las siguientes caracter\'{i}sticas:

\begin{enumerate}
\item  Corriente m\'{a}xima: 500mA.

\item  M\'{a}xima regulaci\'{o}n de carga: 4\%

\item  Protecci\'{o}n contra cortocircuito.
\end{enumerate}

\noindent 
\subsection{6.3 Fuente de alimentaci\'{o}n interna de 5V}

\noindent La fuente de 5V se encarga de alimentar los operacionales, el sensor de efecto hall, el inversor y el regulador de tensi\'{o}n de 2.5V. Debido a los bajos consumos de potencia y bajo costo, se utiliza una fuente lineal. Por lo tanto, se decide utilizar el integrado L78M05CDT, el cual presenta las siguientes caracter\'{i}sticas:

\begin{enumerate}
\item  Corriente m\'{a}xima: 500mA.

\item  M\'{a}xima regulaci\'{o}n de carga: 4\%

\item  Protecci\'{o}n contra cortocircuito.
\end{enumerate}

\noindent 
\section{\eject }

\noindent 
\section{7. Bibliograf\'{i}a}

\noindent [1] HIP4081A, 80V High Frequency H-Bridge Driver. AN9405. Rev 6.00. Renesas. Dic de 2014

\noindent 

\noindent [2]\textit{ HIP4081A 80V/2.5A Peak, High Frequency Full Bridge FET Driver. }FN3659. Rev 8.00. Renesas. Sep de 2015

\noindent 

\noindent [3] \textit{RSX205LAM30, Schottky Barrier Diode}. Rev 2.00. Rohm Semiconductor.May de 2019

\noindent 

\noindent [4] \textit{EKY-350ELL222MM25S, Miniature Aluminum Electrolytic Capacitor}. Ver 3.Chemi-Con. 2021

\noindent 

\noindent [5]\textit{ IPB160N04, OptiMOS{\circledR} -T2Power-Transistor, Rev 1.0, Infineon, Abr de 2010}

\noindent \textit{}

\noindent [6] MCP660/1/2/3/4/5/9, 60MHz, 32V/µs Rail-to-Rail Output (RRO) Op Amps, Rev E, Microchip, 2009.\textit{}

\noindent 
\section{8. Anexo}

\noindent 
\subsection{8.1  Esquem\'{a}ticos}

\noindent 
\paragraph{8.1.1 Principal}

\noindent \includegraphics*[width=5.60in, height=7.92in]{image69}

\noindent 

\noindent 
\paragraph{8.1.2. Controlador de corriente}

\noindent \includegraphics*[width=6.09in, height=8.70in]{image70}

\noindent 
\paragraph{8.1.3. Puente H}

\noindent 

\noindent \includegraphics*[width=5.97in, height=8.52in]{image71}

\noindent 

\noindent 
\paragraph{8.1.4 Compensador anal\'{o}gico}

\noindent 

\noindent \includegraphics*[width=6.25in, height=8.83in]{image72}

\noindent 
\paragraph{8.1.5 Estimador anal\'{o}gico}

\noindent \includegraphics*[width=6.21in, height=8.78in]{image73}\eject 

\noindent 
\paragraph{8.1.6 Interfaz con microcontrolador}

\noindent \includegraphics*[width=6.09in, height=8.70in]{image74}

\noindent 
\paragraph{8.1.7 Fuentes de alimentaci\'{o}n}

\noindent 
\paragraph{\includegraphics*[width=6.09in, height=8.61in]{image75}}

\noindent 
\subsection{8.2 PCB}

\noindent 
\paragraph{8.2.1 Modelo 2D}

\noindent 
\subparagraph{11.2.1.1 Vista superior}

\noindent \includegraphics*[width=5.60in, height=8.01in]{image76}

\noindent 

\noindent 
\subparagraph{8.2.1.2 Vista inferior\includegraphics*[width=6.09in, height=8.70in]{image77}\eject }

\noindent 
\paragraph{8.2.2 Modelo 3D}

\noindent 
\subparagraph{8.2.2.1 Vista superior}

\noindent \includegraphics*[width=6.28in, height=6.27in]{image78}

\noindent 
\subparagraph{8.2.2.2 Vista inferior}

\noindent \includegraphics*[width=6.28in, height=6.48in]{image79}\eject 

\noindent 
\subsection{8.3 Lista de Materiales}

\noindent 

\begin{tabular}{|p{1.5in}|p{0.3in}|p{0.9in}|p{0.2in}|p{0.6in}|p{0.8in}|} \hline 
Designator & Value & Description & Quantity & Mfr PN & Manufacturer \\ \hline 
C100, C102, C104, C110, C111, C112, C125, C126, C127, C128, C129, C206, C207, C208, C301, C302, C405, C410, C505, C506 & 0.1uF & CAP CER 0.1UF 50V X7R 1206 & 20 & C1206X7R500-104KNE & Venkel \\ \hline 
C101, C200, C303, C304, C502 & 1uF & CAP CER 1UF 25V X7R 1206 & 5 & GMC31X7R105M25NT & CAL-CHIP ELECTRONICS, INC. \\ \hline 
C103 & 900pF & CAP CER 910PF 50V NP0 1206 & 1 & C1206C911J5GAC7800 & KEMET \\ \hline 
C105, C106, C123, C124 & 5.6uF & CAP CER 4.7UF 50V X7R 1206 & 4 & CL31B475KBHNFNE & Samsung Electro-Mechanics \\ \hline 
C107, C108, C109 & 2uF & CAP1206 X5R 2.2UF 20\% 35V & 3 & GMC31X5R225M35NT & CAL-CHIP ELECTRONICS, INC. \\ \hline 
C113, C114, C115, C116, C117, C118 & 2200uF & CAP ALUM 2200UF 20\% 35V RADIAL & 6 & EKY-350ELL222MM25S & United Chemi-Con \\ \hline 
C119 & 4.7nF & CAP CER 4700PF 200V X7R 1206 & 1 & GMC31X7R472K200NT & CAL-CHIP ELECTRONICS, INC. \\ \hline 
C120, C122 & 47nF & CAP CER 0.047UF 50V C0G/NP0 1206 & 2 & GMC31CG473J50NT & CAL-CHIP ELECTRONICS, INC. \\ \hline 
C121 & 10uF & CAP CER 10UF 16V X7R 1206 & 1 & GMC31X7R106M16NT & CAL-CHIP ELECTRONICS, INC. \\ \hline 
C201, C202, C203, C300 & 22uF & CAP CER 22UF 16V X5R 1206 & 4 & CL31A226MOCLNNC & Samsung Electro-Mechanics \\ \hline 
C204 & 220nF & CAP CER 0.22UF 50V X7R 1206 & 1 & CL31B224KBFNNNE & Samsung Electro-Mechanics \\ \hline 
C205 & 47uF & CAP CER 47UF 10V X5R 1206 & 1 & LMK316BJ476ML-T & Taiyo Yuden \\ \hline 
C400, C403 & 18nF & CAP CER 0.018UF 100V X7R 1206 & 2 & C1206C183K1RAC7800 & KEMET \\ \hline 
C401 & 150nF & CAP CER 0.15UF 50V X7R 1206 & 1 & 12065C154KAT2A & KYOCERA AVX \\ \hline 
C402, C407 & 270pF & CAP CER 270PF 50V C0G/NP0 1206 & 2 & C1206C271J5GAC7800 & KEMET \\ \hline 
C404, C406 & 1.6uF & CAP CER 1.5UF 10V X7R 1206 & 2 & CC1206KKX7R6BB155 & YAGEO \\ \hline 
C408, C409 & 1.6nF & CAP CER 1600PF 50V NP0 1206 & 2 & C1206C162J5GAC7800 & KEMET \\ \hline 
C500, C504 & 10nF & CAP CER 10000PF 500V X7R 1206 & 2 & C1206C103KCRAC7800 & KEMET \\ \hline 
C501 & 1nF & CAP CER 1000PF 2KV X7R 1206 & 1 & 202R18W102KV4E & Johanson Dielectrics Inc. \\ \hline 
C503 & 470pF & CAP CER 470PF 1KV C0G/NP0 1206 & 1 & CL31C471JIHNNNE & Samsung Electro-Mechanics \\ \hline 
D100, D101, D104, D106, D108, D109, D110, D111, D200, D201, D202, D500, D501 &  & DIODE SCHOTTKY 30V 2A PMDTM & 13 & RSX205LAM30TR & [NoParam], ROHM Semiconductor \\ \hline 
D102, D103, D105, D107 &  & TVS DIODE 18VWM 29.2VC DO214AC & 4 & SMAJ18A & Littelfuse Inc. \\ \hline 
D203, D204, D205, D403 &  & 1206 RED SMD LED & 4 & L152L-LIC & American Opto Plus LED \\ \hline 
D400, D401, D402 & 3V6 & DIODE ZENER 3.6V 3W SMB & 3 & 1SMB5914BT3G & onsemi \\ \hline 
F200 &  & FUSE 500MA 125VAC FAST 1206 & 1 & C1F 500 & Bel Fuse Inc. \\ \hline 
J100, J200, P100, P101, P500, P501 &  & TERM BLK 2P SIDE ENT 2.54MM PCB & 6 & 282834-2 & TE Connectivity AMP Connectors \\ \hline 
J400 &  & CONN D-SUB RCPT 9POS R/A SLDR & 1 & A-DF 09 A/KG-T4S & Assmann WSW Components \\ \hline 
P1, P2, P102, P300 &  & TERM BLK 3P SIDE ENT 2.54MM PCB & 4 & 282834-3 & TE Connectivity AMP Connectors \\ \hline 
Q100, Q101, Q102, Q103 &  & MOSFET N-CH 40V 160A TO263-7 & 4 & IPB160N04S3H2ATMA1 & Infineon Technologies \\ \hline 
R1 & 316 & RES 316 OHM 1\% 1/4W 1206 & 1 & RMCF1206FT316R & Stackpole Electronics Inc \\ \hline 
R2 & 1k & POT 1K OHM 0.08W CARBON LINEAR & 1 & PDB12-H4251-102BF & Bourns Inc. \\ \hline 
R3 & 4k99 & RES 4.99K OHM 1\% 1/4W 1206 & 1 & RMCF1206FT4K99 & Stackpole Electronics Inc \\ \hline 
R100 & 3k2 & RES 3.2K OHM 1\% 1/4W 1206 & 1 & RN73H2BTTD3201F25 & KOA Speer Electronics, Inc. \\ \hline 
R101, R102, R107, R108, R112, R124, R319, R323, R400, R401, R406, R410, R411, R412, R413, R414 & 10k & RES 10K OHM 1\% 1/4W 1206 & 16 & RC1206FR-0710KL & YAGEO \\ \hline 
R103, R113, R114, R115, R116, R122, R204, R301, R302, R303, R304, R306, R308, R309, R310, R321, R322, R405, R409, R513, R515 & 1k & RES 1K OHM 1\% 1/4W 1206 & 21 & RMCF1206FT1K00 & Stackpole Electronics Inc \\ \hline 
R104, R109 & 500 & RES SMD 500 OHM 0.1\% 0.3W 1206 & 2 & Y1625500R000B9W & Vishay Foil Resistors (Division of Vishay Precision Group) \\ \hline 
R105, R117, R118, R125, R126, R202 & 4k7 & RES 4.7K OHM 1\% 1/4W 1206 & 6 & RMCF1206FT4K70 & Stackpole Electronics Inc \\ \hline 
R106, R509 & 15k & RES 15K OHM 1\% 1/4W 1206 & 2 & CRCW120615K0FKEAC & Vishay Dale \\ \hline 
R110, R119, R120, R312, R313, R318, R320 & 200k & RES 200K OHM 1\% 1/4W 1206 & 7 & RC1206FR-07200KL & YAGEO \\ \hline 
R111 & 2k & RES 2K OHM 1\% 1/4W 1206 & 1 & RMCF1206FT2K00 & Stackpole Electronics Inc \\ \hline 
R121, R502, R503, R506, R512 & 100K & RES 100K OHM 1\% 1/4W 1206 & 5 & RMCF1206FT100K & Stackpole Electronics Inc \\ \hline 
R123, R127, R128, R129, R130, R200, R201 & 0R & RES 0 OHM JUMPER 1/4W 1206 & 7 & RMCF1206ZT0R00 & Stackpole Electronics Inc \\ \hline 
R203 & 2.37K & RES 2.37K OHM 1\% 1/4W 1206 & 1 & RMCF1206FG2K37 & Stackpole Electronics Inc \\ \hline 
R205 & 300 & RES 300 OHM 1\% 1/4W 1206 & 1 & RC1206FR-10300RL & YAGEO \\ \hline 
R300, R311 & 459K & RES 459K OHM 1\% 1/4W 1206 & 2 & RN73H2BTTD4593F10 & KOA Speer Electronics, Inc. \\ \hline 
R305, R307 & 9.2k & RES 9.2K OHM 1\% 1/4W 1206 & 2 & RN73H2BTTD9201F10 & KOA Speer Electronics, Inc. \\ \hline 
R314, R315 & 1100 & RES 1.1K OHM 1\% 1/4W 1206 & 2 & RC1206FR-071K1L & YAGEO \\ \hline 
R316, R317 & 21.5k & RES 21.5K OHM 1\% 1/4W 1206 & 2 & RC1206FR-0721K5L & YAGEO \\ \hline 
R402, R407 & 5k & RES SMD 5K OHM 1\% 0.3W 1206 & 2 & Y16255K00000F9W & Vishay Foil Resistors (Division of Vishay Precision Group) \\ \hline 
R403 & 3.3k & RES SMD 3.3K OHM 1\% 3/4W 1206 & 1 & CRCW12063K30FKEAHP & Vishay Dale \\ \hline 
R404, R408 & 50k & RES SMD 50K OHM 0.1\% 1/4W 1206 & 2 & RT1206BRD0750KL & YAGEO \\ \hline 
R415 & 130 & RES 130 OHM 1\% 1/4W 1206 & 1 & RMCF1206FT130R & Stackpole Electronics Inc \\ \hline 
R500, R507 & 1.8M & RES 1.8M OHM 1\% 1/4W 1206 & 2 & RC1206FR-071M8L & YAGEO \\ \hline 
R501, R505, R511 & 3.56K & RES 3.57K OHM 1\% 1/4W 1206 & 3 & RC1206FR-073K57L & YAGEO \\ \hline 
R504 & 25K & RES SMD 25K OHM 1\% 1/4W 1206 & 1 & Y163025K0000F0W & Vishay Foil Resistors (Division of Vishay Precision Group) \\ \hline 
R508 & 8.9K & RES 8.87K OHM 1\% 1/4W 1206 & 1 & RC1206FR-078K87L & YAGEO \\ \hline 
R510 & 10 & RES 10 OHM 1\% 1/4W 1206 & 1 & RMCF1206FT10R0 & Stackpole Electronics Inc \\ \hline 
R514 & 293 & RES SMD 294 OHM 1\% 1/4W 1206 & 1 & CRCW1206294RFKEA & Vishay Dale \\ \hline 
TP100, TP101, TP102, TP103, TP104, TP105, TP106, TP107, TP108, TP109, TP110, TP111, TP112, TP113, TP114, TP115, TP200, TP300, TP301, TP302, TP303, TP400, TP401, TP402, TP403, TP404, TP500, TP501, TP502, TP503, TP504, TP505 &  & PC TEST POINT MINIATURE RED & 32 & 5000 & Keystone Electronics \\ \hline 
U100, U300, U301, U400, U401, U500, U501 &  & IC OPAMP GP 4 CIRCUIT 14TSSOP & 7 & MCP664-E/ST & Microchip Technology \\ \hline 
U101 &  & IC INVERTER OD 3CH 3-INP 8VSSOP & 1 & SN74LVC3G06DCUT & Texas Instruments \\ \hline 
U102 &  & IC GATE DRVR HALF-BRIDGE 20SOIC & 1 & HIP4081AIBZ & Renesas Electronics America Inc \\ \hline 
U103 &  & SENSOR CURRENT HALL 15A AC/DC & 1 & HO 15-NP-0000 & LEM USA Inc. \\ \hline 
U104 &  & IC OPAMP GP 4 CIRCUIT 14TSSOP & 1 &  & Microchip Technology \\ \hline 
VR100, VR101 & 1k & e, TRIMMER 1K OHM 0.5W PC PIN TOP & 2 & 3299Y-1-102LF & Bourns Inc. \\ \hline 
VR200 &  & Voltage Regulator & 1 & L78M12CDT-TR & STMicroelectronics \\ \hline 
VR201 &  & IC REG LINEAR 5V 500MA DPAK & 1 & L78M05CDT-TR & STMicroelectronics \\ \hline 
VR202 &  & IC VREF SERIES 0.08\% SOT23-6 & 1 & MAX6071BAUT41+T & Analog Devices Inc./Maxim Integrated \\ \hline 
VR500 & 309 & TRIMMER 500 OHM 0.5W PC PIN TOP & 1 & 3299Y-1-501LF & Bourns Inc. \\ \hline 
\end{tabular}



\noindent 


\end{document}

