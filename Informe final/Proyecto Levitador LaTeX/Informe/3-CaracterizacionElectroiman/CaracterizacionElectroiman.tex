\chapter{Caracterización del  Electroimán} \chapterlabel{Informe/3-CaracterizacionElectroiman} \label{cap:CaracterizacionElectroiman}

El objetivo de este apartado es poder obtener un modelo físico del electroimán que se dispone. A partir de sus características constructivas, se determinarán otros parámetros del sistema de control (como masa máxima, corriente nominal de trabajo, etc) \cite{Noticia1}


\section{Modelado físico}


%\section{Caracterización del Electroimán}

%\noindent El actuador de este sistema de control es un electroimán. Se eligió construirlo utilizando un núcleo de acero al silicio con un bobinado en su interior. 
%mejorar esto

\noindent Se puede modelar el sistema como un objeto de masa puntual que es sometido a dos fuerzas opuestas en el eje “Y” de la figura \ref{fig:img_modelado-fisico}: la de su propio peso hacia abajo, y una fuerza realizada por el electroimán en sentido contrario. \colorbox{red}{MARCAR Yo EN LA IMAGEN}

%capaz poner otra imagen
\begin{figure}[H]
	\centering
	\includegraphics[width=\textwidth]{modelo-fisico.png}
	\caption{Modelado físico.}
	\label{fig:img_modelado-fisico}
\end{figure}

\noindent La fuerza correspondiente al peso del objeto es $P=m*g$, donde $m$ es la masa en kg y $g$ es la aceleración de la gravedad en $\frac{m}{s^{2}}$. Para mantenerlo levitando en estado de equilibrio, el electroimán debe generar una fuerza de igual módulo y sentido contrario. Esto lo logra a partir de la circulación de un flujo magnético entre su núcleo y la pieza con forma de I. Entre estas dos piezas se conforma lo que se conoce como circuito magnético, en el que la energía fluye impulsada por una fuerza magnetomotriz.

%hacer una intro de como se genera la fuerza magnética???
\noindent La fuerza magnetomotriz generada en el núcleo del electroimán es proporcional a la corriente que circula por su bobinado, y su módulo está dado por la ecuación \ref{eq_fuerza-magnetomotriz}, que también la relaciona con la reluctancia del circuito magnético y la magnitud del flujo magnético.

\begin{equation} \label{eq_fuerza-magnetomotriz}
	F_{mm}=N*i=R_{m}*\phi	
\end{equation}

%mejorar este parrafo
\noindent $R_{m}$ corresponde a la reluctancia del circuito magnético, $\phi$ indica la magnitud del flujo, es decir, la cantidad de campo magnético que atraviesa una superficie y $F_{mm}$ es la fuerza magnetomotriz (que es distinta a la fuerza magnética $F_{m}$). La ley de Hopkinson relaciona estos parámetros con la corriente que circula por el bobinado ($i$) y la cantidad de vueltas de su núcleo ($N$).


\noindent Por otro lado, la inductancia del bobinado está dada por la ecuación \ref{eq_inductancia_flujo}

\begin{equation} \label{eq_inductancia_flujo}
	L*i=N*\phi
\end{equation}


\noindent El electroimán utilizado (figura \ref{fig:img_modelado-fisico}) está compuesto por una pieza en forma de E y otra en forma de I, que se encuentran separadas por un espacio o gap de aire. Este circuito magnético se puede modelar como a un toroide con un corte o separación  de longitud $lA=2*y_{o}$, como se muestra en la figura \ref{fig:img_toroide}

\begin{figure}[H]
	\centering
	\includegraphics[scale=0.75]{toroide.png}
	\caption{Toroide con gap de aire.}
	\label{fig:img_toroide}
\end{figure}

\noindent Partiendo del caso de un toroide sin gap de aire, se puede modelar su reluctancia con la ecuación \ref{eq_reluctancia}. En la que A es el área transversal, $\mu_{r}$ es la permeabilidad relativa del material, $\mu_{o}$ es la permeabilidad magnética en el vacio y $l_{h}$ es la longitud del circuito magnético. 

\begin{equation}\label{eq_reluctancia}
	R_{m}=\frac{l_{h}}{\mu_{o}*\mu_{r}*A}
\end{equation}

\noindent Combinando las ecuaciones \ref{eq_fuerza-magnetomotriz}, \ref{eq_inductancia_flujo}, y \ref{eq_reluctancia}, se llega a la expresión de la inductancia del toroide sin gap de aire:

\begin{equation}\label{eq_inductancia}
	L=\frac{N}{i}*\phi=\frac{N}{i}*\frac{N*i}{R_{m}}=\frac{N^	{2}*A*\mu_{o}*\mu_{r}}{l_{h}}
\end{equation}


\noindent Al agregar un gap de aire, se genera una discontinuidad en el medio magnético de distancia $l_{A}$ (longitud en el aire).

\noindent El flujo magnético atraviesa esta discontinuidad sin desviaciones (siempre que $l_{A}$ sea pequeña comparada con el área transversal A).

\noindent Al haber un gap de aire la reluctancia del sistema aumenta, por lo que se debe sumar, en serie a $R_{m}$, la reluctancia en el aire $R_{a}$. Por otro lado, $l_{h}$ prácticamente no se ve modificada ya que es una pequeña incisión en el circuito magnético. De esta forma, se obtiene la ecuación \ref{eq_inductancia_con_gap}.

\begin{equation}\label{eq_inductancia_con_gap}
L=\frac{N^{2}}{R_{m}*R_{a}}=\frac{N^{2}}{\frac{l_{h}}{A*\mu_{o}*\mu_{r}}+\frac{l_{a}}{A*\mu_{o}}}=\frac{N^{2}*A*\mu_{o}}{l_{a}+\frac{l_{h}}{\mu_{r}}}
\end{equation}

\noindent Debido a que $l_{a}$ es el gap de aire, se debe reemplazar por la distancia de separación entre las dos piezas magnéticas, que está representada por la variable $y_{o}$. En el caso de nuestro electroimán, las líneas de fuerza atraviesan dos veces y, por lo tanto $l_{a}=2*y_{o}$.

\noindent Además, en la ecuación \ref{eq_inductancia_con_gap} $\mu_{r}>>1$ entonces $2*y_{o}>>\frac{l_{h}}{\mu_{r}}$, se obtiene la inductancia en función de la distancia del gap de aire:

\begin{equation}\label{eq_inductancia_vs_y}
		L(y)=\frac{{N^{2}*A*\mu_{o}}}{2*y_{o}}
\end{equation}

\section{Cálculo de la fuerza magnética}

\noindent La fuerza magnética necesaria para mantener a un objeto en suspensión se puede encontrar a partir de la energía almacenada en un inductor y al considerar que esta es  igual al trabajo:

\begin{equation}\label{eq_energia}
	E(i,y)=W=\int{F_{m}*dy}=>F_{m}=\frac{\partial{E(i,y)}}{\partial{y}}
\end{equation}

\noindent Y sabiendo la expresión para obtener la energía que almacena un inductor en su campo magnético:

\begin{equation}\label{eq_energia_2}
	E(i,y)=\frac{L(i,y)*i^{2}}{2}
\end{equation}

\noindent Esto significa que la cantidad de energía que almacena el sistema es función del gap de aire. Combinando las ecuaciones \ref{eq_inductancia_vs_y}, \ref{eq_energia} y \ref{eq_energia_2}:

\begin{equation}\label{eq_fuerza_magnetica}
	\abs{F_{m}}=\frac{\partial{E(i,y)}}{\partial{y}}=\frac{i^{2}}{2}*\frac{\partial{\frac{{N^{2}*A*\mu_{o}}}{2*y}}}{\partial{y}}=\frac{i^{2}*N^{2}*\mu_{o}*A}{4*y^{2}}
\end{equation}

\noindent Como se puede apreciar en la ecuación \ref{eq_fuerza_magnetica} la fuerza es proporcional al cuadrado de la acción de control (es decir, de la corriente), e inversamente proporcional al cuadrado de la variable que se desea controlar (es decir, de la distancia). Por lo tanto, el problema adquiere un comportamiento sumamente alineal.

\section{Diseño del electroimán}

%¿sería conveniente hacer el diseño partiendo del electroimán que ya tenemos, y en base a eso determinar el peso que podemos levantar?
\noindent Se debe determinar qué dimensiones debe tener el electroimán a utilizar para que sea capaz de ejercer la fuerza magnética necesaria para mantener levitando el peso deseado.

\noindent Al analizar la ecuación \ref{eq_fuerza_magnetica} se puede observar que hay dos parámetros que son propios del electroimán: el área del núcleo A y la cantidad de vueltas del bobinado N. Para obtener una expresión de diseño, partimos de esta ecuación y la igualamos al fuerza ejercida por el peso que debe soportar ($\abs{F_{m}}=m*g$):

\begin{equation}\label{eq_fuerza_peso}
	m*g=\frac{i^{2}*N^{2}*\mu_{o}*A}{4*y^{2}}
\end{equation}

\noindent De la ecuación \ref{eq_fuerza_peso} podemos despejar el producto $N*i$ que es la fuerza magnetomotriz:

\begin{equation} \label{eq_n_por_i}
	N*i=y*\sqrt{\frac{4*m*g}{\mu_{o}*A}}
\end{equation}

\noindent Partiendo de la ecuación \ref{eq_inductancia_flujo}, reemplazando $\phi=B*A$ y considerando el caso de máxima inducción magnética ($i_{max}$):

\begin{equation}\label{eq_inductancia_densidad}
	L*i_{max}=N*B_{max}*A
\end{equation}

\noindent Luego, relacionando la ecuación \ref{eq_inductancia_vs_y} y \ref{eq_inductancia_densidad}:

\begin{equation} \label{eq_bmax}
	B_{max}=\mu_{o}*\frac{N*i_{max}}{2*y}
\end{equation}

\noindent Combinando la ecuación \ref{eq_bmax} y \ref{eq_n_por_i}:

\begin{equation}
	B_{max}=\sqrt{\frac{m*g*\mu_{o}}{A}}
\end{equation}

\noindent Finalmente:

\begin{equation} \label{eq_area}
	A>=\frac{m*g*\mu_{o}}{B_{max}^{2}}
\end{equation}

\noindent Con el objetivo de evitar la saturación del núcleo del electroimán en el caso que circule la máxima corriente, el área transversal debe cumplir con la expresión \ref{eq_area}. Para el cálculo se utiliza $B_{max}=0.7$ Tesla, que corresponde con la inducción máxima del acero de silicio. \colorbox{red}{CHEQUEAR}. Finalmente se obtiene:

\begin{equation}
	A>=20 cm^{2}
\end{equation}


\section{Diseño del electroimán}

\noindent El electroimán está construido por un núcleo de acero laminado con forma de E, que tiene un cable bobinado en su rama central, y otra pieza con forma de I.\colorbox{red}{poner imagen con dimensiones}. 
% 

Está construido por una laminación normalizada sin desperdicio 600. Estas son útiles ya que cada par de laminas E y I puede fabricarse a partir de una lámina de acero rectangular, de manera de que no se desperdicia material durante la fabricación. \colorbox{red}{poner figura}

 Está compuesto por dos piezas: una con forma de “E” y otra con forma de “I”. En su núcleo tiene un bobinado de 150 vueltas (N) de cobre esmaltado con un diámetro de $2.5$ mm. El núcleo es de sección cuadrada ya que esto maximiza el área mientras que disminuye el perímetro, reduciendo así la longitud media de las espiras y ahorrando material. 


\section{Diseño del bobinado}

\noindent Se desea dimensionar la cantidad de vueltas del bobinado y la corriente requerida para las condiciones del problema. Partiendo de la ecuación \ref{eq_n_por_i} y considerando un área transversal $A=25 cm^{2}$, y las condiciones de trabajo más exigentes ($y=5$ mm y $m=30$ kg):

\begin{equation}
	N*i_{max}=3060.69 
\end{equation}

\noindent Debido a que ya se dispone de un electroimán con $N=150$, según el resultado anterior, se impone una corriente máxima de $20.4$ A. Para poder tener cierto margen en la masa máxima soportada, se adopta una corriente máxima de $21$ A, lo que resulta en un $N*i_{max}=3150$.



\section{Expresión de inductancia linealizada}

\noindent Volviendo a la ecuación \ref{eq_inductancia_vs_y}, se realiza una expansión por serie de Taylor y se desprecian los términos de orden mayor o igual a 2, con el objetivo de llegar a una expresión lineal para la inductancia se obtiene:

\begin{equation} \label{eq_inductancia_lineal_teorica}
	L(y)[mH]=-2.2089*y[mm]+17.67 [mH]
\end{equation}

\section{Mediciones sobre el electroimán}

\colorbox{red}{esto iría acá??}

