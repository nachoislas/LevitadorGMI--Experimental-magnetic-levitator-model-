%esto es para poder dibujar flechitas en la histeresis usando ->-
\tikzset{->-/.style={decoration={
			markings,
			mark=at position .5 with {\arrow{>}}},postaction={decorate}}}

%aca se define el dibujito de la histeresis uniendo lineas
\def\hist{
	\tikz[remember picture,overlay]{
		%dibujamos los ejes. stealth es para las flechas
		\draw [stealth-stealth] (-1,0) -- (1,0) node[pos=1, yshift=0.5em]{${\scriptstyle e}$};  %eje x
		\draw [stealth-stealth] (0,-0.8)--(0,0.8) node[pos=0.9, xshift=0.8em]{${\scriptstyle V_L}$}; %eje y
		
		%esto es para dibujar los tramos de la histeresis
		\draw [->-] (-0.7,-0.4)-- (0.3,-0.4);
		\draw  [->-] (0.3,-0.4) -- (0.3,0.4); 
		\draw [->-] (0.7,0.4) --(-0.3,0.4);
		\draw [->-] (-0.3,0.4) -- (-0.3, -0.4);
}}

%aca hago lo mismo pero la salida es Vh en vez de VL
\def\histvh{
	\tikz[remember picture,overlay]{
		%dibujamos los ejes. stealth es para las flechas
		\draw [stealth-stealth] (-1,0) -- (1,0) node[pos=1, yshift=0.5em]{${\scriptstyle e}$};  %eje x
		\draw [stealth-stealth] (0,-0.8)--(0,0.8) node[pos=0.9, xshift=0.8em]{${\scriptstyle V_h}$}; %eje y
		
		%esto es para dibujar los tramos de la histeresis
		\draw [->-] (-0.7,-0.4)-- (0.3,-0.4);
		\draw  [->-] (0.3,-0.4) -- (0.3,0.4); 
		\draw [->-] (0.7,0.4) --(-0.3,0.4);
		\draw [->-] (-0.3,0.4) -- (-0.3, -0.4);
}}