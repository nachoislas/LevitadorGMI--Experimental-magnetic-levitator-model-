\chapter{Introduccion} \chapterlabel{Informe/1-Introduccion} \label{cap:Introducción}

\section{Introducción}

\noindent El desarrollo de este proyecto surge de la necesidad de contar con un dispositivo que permita a los alumnos de la asignatura Sistemas de Control poner en práctica los conceptos aprendidos durante las clases, puesto que al ser del tipo integrador,  abarca varios temas de la materia.

\noindent Durante la cursada del año 2019, los integrantes de este proyecto, realizaron una serie de pruebas y modificaciones sobre un sistema de levitación magnética basado en un electroimán de laminación normalizada con núcleo tipo “E” previamente desarrollado y construido por la cátedra de Sistemas de Control, el cual nunca había sido probado. Debido a que no fue posible obtener una versión funcional del prototipo al finalizar la cursada, se les propuso realizar un rediseño de todas las etapas que componen al sistema como proyecto final de grado. 


\section{Alcance del proyecto}

\noindent El objetivo de este proyecto es modelar un electroimán de laminación normalizada con núcleo tipo “E'' y diseñar un control analógico y digital para un sistema de levitación magnética. El documento consta del modelado teórico de todas las etapas pertenecientes al sistema, junto con un diseño circuital, simulaciones correspondientes y un diseño integral de PCB.

\noindent Se espera que en el futuro pueda ser construido para que sirva como herramienta para los alumnos de la asignatura  Sistemas de Control, de forma tal que les permita experimentar y afianzar los conceptos teóricos adquiridos durante el transcurso de la cursada.

\section{Contexto del proyecto}

\noindent El proyecto comenzó en Junio del 2020 con el objetivo de poder construir e implementar el prototipo. Sin embargo, debido a los retrasos ocasionados por la pandemia de COVID-19,  la escasez de componentes a nivel mundial y de presupuesto para la construcción del PCB, sumada a la  necesidad de no extender indefinidamente el proyecto, se optó por acotar el alcance sólo al diseño teórico de todas las etapas y del PCB. 
