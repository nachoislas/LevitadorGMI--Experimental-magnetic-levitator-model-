\tikzset{%
	buffer/.style={
		draw,
		shape border rotate=270,
		regular polygon,
		regular polygon sides=3,
		fill=blue!20,
		node distance=2cm,
		minimum height=4em
	}
}

\tikzstyle{block} = [draw, fill=blue!20, rectangle, 
minimum height=2.5em, minimum width=3em]

%Acá se define eñ diagrama en bloques completo
\begin{tikzpicture}[auto, node distance=1.5cm,>=latex']
	% We start by placing the blocks
	\node [input, name=input] {};
	\node [buffer, right=of input](F){F};
	\node [sum, right of=F, node distance=2cm] (suma_interna) {+};
	\node [block, right=of suma_interna] (interno) {$G_{ext}$};
	\node [block, right=of interno] (tlc_interna) {$TLC_{interna}$};
	\node [block, below=of interno] (realimentacion_interna) {$H_{estim}$};
	
	
%	\node [block, right of=suma] (amplificador) {$A(s)$};
	\node [output, right of=tlc_interna, node distance=3cm] (output) {};
%	\node [block, below of=amplificador] (realimentacion) {$H(s)$};
%	
%	
%	% Once the nodes are placed, connecting them is easy. 
	\draw [draw,->] (input) -- node[pos=0.2]{$V_{y_{ref}}$} (F);
	\draw [draw,->] (F) --  node[pos=0.9]{$+$}(suma_interna);
	\draw [draw,->] (suma_interna) -- node{$Ve_{ext}$} (interno);
	\draw [draw,->] (interno) -- node{$V_{ref_c}$} (tlc_interna);
	\draw [draw,->] (tlc_interna) -- node[name=y]{$Y_g$} (output);
%	\draw [draw,->] (amplificador) -- node[name=y]{$V_{deriv}$} (output);
	\draw [->] (y) |- (realimentacion_interna);
	\draw [->] (realimentacion_interna) -| node[pos=0.25]{$V_{estim}$}  node[pos=0.99]{$+$} (suma_interna);
%	\draw [->] (y) |- (realimentacion_externa);
%	\draw [->] (realimentacion_externa) -| node[pos=0.25]{$V_{estim}$} node[pos=0.99]{-} (suma_externa);
\end{tikzpicture}