\chapter{Estimador analógico} \chapterlabel{Informe/5-EstimadorAnalogico} \label{cap:Estimador Analogico}

Para que la placa de control pueda mantener la distancia de separación de entrehierro $Y_{g}$ es necesario conocer su valor para luego actuar en consecuencia. Si bien se podrían utilizar sensores  especializados para ello, para este proyecto se optó por medirla de manera indirecta a partir de la pendiente de la corriente que circula por el electroimán. De esta forma, se logran aplicar conceptos de estimación de variables, aprendidos durante la carrera. 

En este capítulo se detalla la estrategia utilizada para realizar la estimación de posición a partir de la corriente del electroimán, junto con el diseño circuital y sus respectivas simulaciones. Finalmente se obtiene una función transferencia del bloque estimador que será luego utilizada para el diseño del compensador analógico.

\section{Análisis de la estimación}

Como se analizó en el capítulo \ref{cap:ControladorCorriente}, para controlar el valor medio de la corriente se utiliza una fuente conmutada que alterna la polaridad de la tensión aplicada al electroimán. Esto genera una onda de corriente triangular superpuesta al valor medio deseado, cuyas pendientes de crecimiento y de decrecimiento contienen información de la distancia de separación de entrehierro. Por lo tanto, se decidió realizar una estimación de la distancia a partir de la medición de dichas pendientes.

En la sección \ref{sec_analisis_estimacion} se obtuvo una ecuación que relaciona la pendiente de la corriente con la distancia de entrehierro, en función de constantes del sistema. Si bien este resultado es correcto, se hace uso de las mediciones realizadas sobre el electroimán para calcular la ecuación \ref{eq_derivada_corriente_2} utilizando la expresión de inductancia linealizada \ref{eq_induct_practica}, que se repite a continuación. De esta forma, se obtiene una expresión mas aproximada al comportamiento real de la planta.

\begin{equation*}
	L(Y_g)=-2.56*Y_{g}+0.027\:Hy
\end{equation*}

Por lo tanto, utilizando esta inductancia en la ecuación \ref{eq_derivada_corriente_2} se obtiene:

\begin{equation} \label{eq_derivada_corriente_3}
	\frac{di_L(t)}{dt}= \frac{V_L}{L(Y_g)}=\frac{V_L}{-2.56*Y_{g}+0.027}
\end{equation}

Como se diseñó en el capítulo \ref{cap:ControladorCorriente}, la alimentación $V_L$ del electroimán tiene dos valores posibles $V_L=+V_{cc}$ o $V_L=-V{cc}$. Es por ello que se puede reescribir la expresión \ref{eq_derivada_corriente_3} en función de $V_{cc}$:

\begin{equation} \label{eq_derivada_corriente_4}
	\left|\frac{di_L(t)}{dt}\right|= \frac{V_{cc}}{L(Y_g)}=\frac{V_{cc}}{-2.56*Y_{g}+0.027}
\end{equation}

A partir de la expresión \ref{eq_derivada_corriente_4} se evalúa la derivada para distintos valores de distancia y se obtiene la tabla \ref{tab_derivada_linealizada}.

\begin{table}[H]
	\begin{center}
		\begin{tabular}{| c | c |}
			\hline
			$Y_g[\:mm]$ & $\frac{di_L(t)}{dt} [\:\frac{A}{s}]$\\ \hline
			2 & 1091.9 \\ \hline 
			3 & 1235.8 \\ \hline 
			4 & 1423.5 \\ \hline 
			5 & 1678.3 \\ \hline 
		\end{tabular}
		\caption{Pendiente de la corriente en función de la posición.}
		\label{tab_derivada_linealizada}
	\end{center}
\end{table}

Se realiza una regresión por mínimos cuadrados a los valores de la tabla \ref{tab_derivada_linealizada} y se obtiene una expresión linealizada de la pendiente, que se  muestra en la ecuación \ref{eq_di-dt_lineal}.


\begin{equation} \label{eq_di-dt_lineal}
	{\left|\frac{di_L}{dt}\right|}_{Lineal}[\:\frac{A}{s}]= 194690 * Y_g\:[m]+676
\end{equation}

De la expresión \ref{eq_di-dt_lineal} se despeja la distancia de separación y se obtiene:

\begin{equation} \label{eq_Yg_despejada}
	Y_g\:[m] =5.136*10^{-6}*{\left|\frac{di_L}{dt}\right|}_{Lineal} - 3.472*10^{-3}
\end{equation}


En la expresión \ref{eq_Yg_despejada} se puede observar que, para obtener una estimación de la distancia de separación, es necesario medir el módulo de la pendiente de la onda triangular de la corriente en el electroimán. Para realizar la estimación a partir de un circuito analógico se propone implementar un circuito derivador que permita obtener la pendiente de la corriente. Como la corriente tiene forma triangular, estará compuesta por un segmento creciente (pendiente positiva) y un segmento decreciente (pendiente negativa). Debido a que en la expresión \ref{eq_Yg_despejada} se necesita obtener su módulo, se agrega una etapa de rectificación a la salida del derivador.

La forma de onda correspondiente a la salida de cada etapa a implementar se muestra en la figura \ref{fig:img_forma-de-onda-estimador}. En ella se observa que se parte con una onda triangular, correspondiente a la corriente que circula por el electroimán. Luego, al pasar por el derivador, se obtiene una onda pulsada, cuyos valor superior e inferior se corresponden con la pendiente de crecimiento y de decrecimiento de la corriente. Luego, al entrar a la etapa de rectificación se obtiene un valor constante y proporcional a la pendiente y, por ende, a la distancia de separación de entrehierro.


\begin{figure}[H]
	\centering
	\includegraphics[scale=0.7]{forma-de-onda-estimador.png}
	\caption{Formas de onda de cada etapa del estimador.}
	\label{fig:img_forma-de-onda-estimador}
\end{figure}



Hasta ahora se planteó que la entrada al derivador es la corriente del electroimán. Sin embargo, esta no es medida de manera directa, sino que se mide a través del sensor de efecto Hall, cuya salida es $V_{IL_{feed}}$ (\ref{eq_salida_restador_hall_2}). Por lo tanto es necesario tener en cuenta su ganancia $H_0$ al momento de diseñar el estimador. La expresión a implementar circuitalmente resulta:

\begin{equation} \label{eq_Yg_salida_sensor}
	Y_g=\frac{5.136*10^{-6}}{H_0}*{\left|\frac{dV_{IL_{feed}}}{dt}\right|}_{Lineal} - 3.472*10^{-3}
\end{equation}

%Además se decidió agregar un LPF luego del rectificador para reducir las frecuencias no deseadas introducidas por el cambio de signo de la pendiente, las cuales generan saltos sobre un punto de operación en la tensión de salida del derivador.
El diagrama en bloques que describe cómo está conformado el estimador se muestra en la figura \ref{fig:img_diag-en-bloques_estimador}.

\begin{figure}[H]
	\centering
	%Acá se define eñ diagrama en bloques completo
\begin{tikzpicture}[auto, node distance=2.5cm,>=latex']
	% We start by placing the blocks
	\node [input, name=input] {};
	\node [block, right of=input] (derivador) {$DERIVADOR$};

	\node [block, right of=derivador, 
	node distance=4.5cm] (modulo) {$|.|$};
	\node [output, right of=modulo] (output) {};

	% Once the nodes are placed, connecting them is easy. 
	\draw [draw,->] (input) -- node[pos=0.1]{$V_{IL_{feed}}$} (derivador);
	\draw [draw,->] (derivador) -- node{$V_{deriv}$} (modulo);
	\draw [->] (modulo) -- node [name=y] {$V_{estim}$}(output);
\end{tikzpicture}


	
	\caption{Diagrama en bloques simplificado del estimador.}	\label{fig:img_diag-en-bloques_estimador}
\end{figure}
\colorbox{red}{Ver si están bien los nombres de las señales y los bloques en el diagrama}

A continuación se diseña la implementación circuital de cada una de las etapas que componen al diagrama en bloques \ref{fig:img_diag-en-bloques_estimador}.

{
	%Por lo tanto, en esta capítulo/sección.. se aborda el análisis de la implementación…
	
	%En el capítulo 3 se analizó la forma de onda resultante de la corriente del electroimán al ser excitado con una fuente conmutada. Esta resulta en una onda triangular con un valor medio deseado, cuya pendiente de crecimiento y de decrecimiento varía proporcionalmente con distancia de separación de entrehierro. A partir de esto, se decidió realizar una estimación de la misma a partir de la medición de la pendiente de la onda triangular de la corriente.
}


\section{Diseño circuital del estimador de posición}

\subsection{Circuito derivador}

Para poder obtener $\frac{dV_{IL_{feed}}}{dt}$ se utiliza un circuito derivador basado en un amplificador operacional como el que se muestra en la figura \ref{fig:img_Circuito-derivador}. Se decide utilizar amplificadores operacionales alimentados con una fuente de tensión simple de $5\:V$ con una excursión completa en su salida. 

Debido a que $V_{IL_{feed}}$ es una onda triangular, su derivada será una onda pulsada. Por lo tanto, para poder representar valores de pendientes tanto positivos como negativos es necesario polarizar la salida del derivador en un punto de operación ($V_{op}$) para no perder información. Como la salida varía entre $0\:V$ y $5\:V$ se utiliza $V_{op}=2.5\:V$ para permitir excursión en ambos sentidos.

\begin{figure}[H]
	\centering
	\includegraphics[scale=0.5]{Circuito-derivador.png}
	\caption{Circuito derivador.}
	\label{fig:img_Circuito-derivador}
\end{figure}


La salida del circuito $V_{deriv}(t)$ ante su entrada $V_{IL_{feed}}$ es:

\begin{equation} 
	V_{deriv}(t)\ = V_{op} - \frac{dV_{IL_{feed}}}{dt}*C_1*R_1
\end{equation}

Teniendo en cuenta que $V_{IL_{feed}}=I_L*H_0+0.1\:V$, al hacer su derivada se pierde el término constante y podemos expresar la ecuación en función de $V_{IL}=I_L*H_0$, entonces $ \frac{dV_{IL_{feed}}}{dt}= \frac{dV_{IL}}{dt}$.

\colorbox{red}{Para que se deja en función de ViL si en la expresión 4.7 lo volvemos a dejar en función de la corriente IL?}

\begin{equation} \label{eq_vyf1}
	V_{deriv}(t)\ = 2.5V - \frac{dV_{IL}}{dt}*C_1*R_1
\end{equation}

Al expresar la ecuación \ref{eq_vyf1} en función de la derivada de la corriente se obtiene:

\begin{equation} \label{eq_vyf2}
	V_{deriv}(t)\ =2.5\:V\ -\frac{di_L}{dt}*H_0*C_1*R_1
\end{equation}

Según la expresión \ref{eq_vyf2}, la salida del derivador $V_{deriv}$ excursionará alrededor de $V_{op}=2.5\:V$ con un valor máximo de $5\:V$ y un mínimo de $0\:V$. Cualquier valor que supere esos límites provocaría una saturación.

Para evitar una saturación de la salida del derivador y teniendo en cuenta que $V_{deriv}(t)$ presenta variaciones alrededor del punto de operación de $2.5\:V$ se debe cumplir, para todos los casos posibles de pendiente en el rango de distancia de trabajo, que:

\begin{equation} \label{eq_vyf3}
	\left|-\frac{dI_L}{dt}*H_0*C_1*R_1\right|\ \le 2.5\:V
\end{equation}

Por lo tanto, con la ecuación \ref{eq_derivada_corriente_2} y \ref{eq_vyf3}:

\begin{equation} \label{eq_condicionC1-R1}
	C_1*R_1\le\frac{2.5\ \:V\ *L_{min}}{V_{cc}*H_0}
\end{equation}

Con $L_{min}= L(5\: mm) = 14.9\: mHy$ se obtiene: 

\begin{equation} \label{eq_condicionC1-R1-2}
	C_1*R_1\le\ 29.1\ ms
\end{equation}

El derivador tiene como salida una onda pulsada, cuyo nivel superior es proporcional a la pendiente de bajada de la corriente en el electroimán, y el nivel inferior es proporcional a la pendiente de subida.

Para los cálculos se utilizó $\tau = C_1*R_1= 25\:ms$, para dar un margen y evitar la saturación del amplificador operacional.  

Con la ecuación \ref{eq_di-dt_lineal} y \ref{eq_vyf2}, y con una variación en torno a $2.5\:V$ se obtiene una expresión para la salida del derivador en función de la distancia de entrehierro:


\begin{equation} \label{eq_Vyf-lineal}
	V_{deriv}(Y_g)\:[V] =\ |H_0*C_1*R_1*\frac{di_L}{dt}| +2.5=0.2595*Y_g[mm]+3.4
\end{equation}

\colorbox{red}{Esta bien que vaya el modulo en la ecuación de arriba?}CREO QUE IMPLICITAMENTE ASUMIMOS QUE DESPUÉS SE RECTIFICA, ENTONCES CONSIDERAMOS EL CASO DE PENDIENTE DECRECIENTE, QUE SE SUMA A LOS 2.5V

-------------------------------
Para pendiente positiva, la salida del circuito derivador resulta:

\begin{equation} %\label{eq_Vyf-lineal}
	V_{deriv}(Y_g) = -0.2595*Y_g[mm] + 1.6\:V
\end{equation}


Para pendiente negativa, la salida del circuito derivador resulta:

\begin{equation} %\label{eq_Vyf-lineal}
	V_{deriv}(Y_g) = 0.2595*Y_g[mm] + 3.4\:V
\end{equation}

----------------------------------

A partir de la expresión \ref{eq_Vyf-lineal} se realizó la tabla \ref{tab_Vyf_vs_y} que muestra el valor de salida del derivador para cada distancia de entrehierro. Se puede observar  que para los posibles valores en los que el electroimán trabaja, el estimador posee un rango de salida ${\mathit{\Delta}{V_{deriv}}_{Lineal}}(5\:mm-2\:mm)= 0.78\:V$.

\begin{table}[H]
	\begin{center}
		\begin{tabular}{| c | c |}
			\hline
			$Y_g[\:mm]$ & ${V_{deriv}(Y_g)}_{Lineal} [\:V]$\\ \hline
			2 & 3.92 \\ \hline 
			3 & 4.18 \\ \hline 
			4 & 4.44 \\ \hline 
			5 & 4.7 \\ \hline 
		\end{tabular}
		\caption{Tensión de salida del derivador ($V_{deriv}$) en función de la posición ($Y_g$).}
		\label{tab_Vyf_vs_y}
	\end{center}
\end{table}

\colorbox{red}{Esta tabla solo sería para los niveles superiores de la onda cuadrad..  Faltaría otra columna con los niveles inferiores..}

\subsection{Elección de amplificador operacional}

Para el diseño del derivador es necesario escoger de forma adecuada un amplificador operacional que permita mantener características derivativas en toda la banda de interés.

Dado que la forma de onda triangular que se desea derivar posee una frecuencia fundamental de $2 \:kHz$ es necesario que el amplificador operacional seleccionado posea un ancho de banda y una ganancia suficiente que permita el diseño de un derivador con características derivativas hasta al menos la quinta armónica de dicha frecuencia que es de $2KHz*5=10 KHz$.

Se escogió el MCP660 el cual cumple con las características mencionadas anteriormente. Posee una ganancia continua de $130\:dB$, y un polo en $20\:Hz$. Además se destaca su slew rate de $32\:V/us$.

\subsection{Circuito del derivador compensado}

Puesto que los circuitos derivadores pueden presentar inestabilidad a alta frecuencia, es necesario compensarlos mediante el agregado de una resistencia en serie al capacitor, para que genere un cero en la transferencia de realimentación (ecuación \ref{eq_Aw_2}), como se observa en la figura  \ref{fig:img_Circuito_derivador_compensado}.

\begin{figure}[H]
	\centering
	\includegraphics[scale=0.5]{Circuito-derivador- compensado.png}
	\caption{Circuito derivador compensado.}
	\label{fig:img_Circuito_derivador_compensado}
\end{figure}

El amplificador operacional utilizado es internamente compensado, por lo que todos sus otros polos los tiene luego del cruce por $0\:dB$ de la ganancia. Para simplificar el análisis estos no se tienen en cuenta, ya que están fuera de la zona de interés.

\begin{equation} \label{eq_Aw_1}
	A(w)=\frac{1778279}{(\frac{s}{2\pi *20}+1)}
\end{equation} 

\begin{equation} \label{eq_Aw_2}
	\frac{1}{H(w)}=\frac{1+s*C_1*(R_1+R_2)}{1+s*C_1*R_2}\simeq \frac{1+s*C_1*R_1}{1+s*C_1*R_2}
\end{equation}

Para compensar el circuito se coloca un polo en $16 \:kHz$, que da como resultado $R_2=10\:\Omega$, $C_1=1\:uF$ y $R_1=25\: k\Omega$ y un margen de fase de $\phi =49.6{}^\circ $, como se puede observar en la figura \ref{fig:img_GH del derivador compensado}.

\begin{figure}[H]
	\centering
	\includegraphics[scale=0.5]{GH-del-derivador-compensado.png}
	\caption{Transferencia a lazo abierto del derivador compensado.}
	\label{fig:img_GH del derivador compensado}
\end{figure}

\begin{figure}[H]
	\centering
	\includegraphics[scale=0.7]{Transferencia-de-lazo-cerrado.png}
	\caption{Transferencia de lazo cerrado.}
	\label{fig:img_Transferencia-de-lazo-cerrado}
\end{figure}

Como se observa en la figura \ref{fig:img_Transferencia-de-lazo-cerrado}, la transferencia de lazo cerrado (TLC) tiene un comportamiento derivativo hasta la quinta armónica de la señal ($10 \:kHz$), como es deseado.

A continuación se muestra la TLC del circuito derivador:

\begin{equation} \label{eq_TLC_derivador}
	{TLC}_{derivador}=\frac{V_{deriv}}{V_{iL}}=\frac{-0.025*s}{1+(\frac{2*0.473}{94.5\: k})*s+(\frac{s}{94.5\:k})^2}
\end{equation} 

\section{Etapa de rectificación}

Como se mencionó en la \colorbox{red}{introducción}, es necesario obtener el valor absoluto de la salida del derivador para tener a la salida del estimador una señal aproximadamente continua. Es por ello que se implementa un circuito rectificador en la salida del derivador alrededor de $V_{op}=2.5\:V$.

Para la implementación circuital del rectificador se decidió utilizar la misma topología usada en la versión anterior del prototipo, que se muestra en la figura \ref{fig:img_Circuito_estimador_de_posición_completo}. Este circuito está compuesto por tres etapas: rectificación, resta y filtrado. Para facilitar su comprensión se realiza un análisis de cada una de estas etapas por separado.

\begin{figure}[H]
	\centering
	\includegraphics[scale=0.5]{Circuito-estimador-de-posición-completo.png}
	\caption{Circuito de rectificación, resta y filtrado.}
	\label{fig:img_Circuito_estimador_de_posición_completo}
\end{figure}

\subsection{Rectificador}

Para entender el funcionamiento del rectificador, se comienza con el análisis de la primer etapa del circuito de la figura \ref{fig:img_Circuito_estimador_de_posición_completo}. Por lo tanto, se simplifica el circuito al mostrado en la figura \ref{fig:img_Rectificador_y_restador}. Se parte de la suposición de que en un amplificador operacionideal, la tensión diferencial ($V_d$) es igual a cero. De esta forma, como la entrada no inversora está fijada en $2.5\:V$, la misma tensión se encuentra en la entrada inversora.


\begin{figure}[H]
	\centering
	\includegraphics[scale=0.7]{Rectificador-y-restador.png}
	\caption{Circuito rectificador.}
	\label{fig:img_Rectificador_y_restador}
\end{figure}

Al analizar la corriente en la resistencia $R_{25}$ (con sentido positivo hacia la izquierda) en función de $V_{deriv}$, resulta:

\begin{equation} \label{eq_corriente_r25}
	I_{R25}=\frac{2.5\:V\ -\ V_{deriv}}{R_{25}}
\end{equation}

En el caso de que $V_{deriv}$ $\mathrm{<}$ $2.5\:V$, la corriente será positiva. Esta misma corriente proviene desde la salida del operacional, a través del diodo $D_5$ y la resistencia $R_{26}$. Si se desprecia la tensión del diodo en directa se obtiene que la salida del operacional es igual a $V^+$:

\begin{equation} \label{eq_V+}
	V^+=I_{R25}*R_{26}+2.5\:V=\frac{2.5\:V-V_{deriv}\ }{R25}*R_{26}+2.5\:V\ 
\end{equation} 

Como $R_{25}=R_{26}$

\begin{equation} \label{eq_V+_2}
	V^+\ =\ 2.5\:V\ -\ V_{deriv}\ +2.5\:V\ =\ 5\:V\ -\ V_{deriv}\ 
\end{equation}

Además, dado que el diodo $D_6$ queda polarizado en inversa y la caída de tensión en $R_{27}$ es despreciable, se obtiene que:
 
 \begin{equation} 
 	V^- = 2.5\:V 
 \end{equation}

Análogamente, si $V_{deriv}$ $\mathrm{>}$ $2.5\:V$, se puede encontrar:

\begin{equation} \label{eq_V+_3}
	V^- =5\:V-V_{deriv} 
\end{equation}

\begin{equation} 
	V^+ = 2.5\:V
\end{equation}


\subsection{Restador}

Se utiliza un amplificador operacional en modo diferencial como restador. El circuito utilizado se observa en la figura \ref{fig:img_Restador} y se obtiene lo siguiente:

Cuando $V_{deriv}$ $\mathrm{<}$ $2.5\:V$:
\begin{equation*} 
	\begin{aligned}
		V_{estim}&=V^+-\ V^-\ +2.5\:V\\ 
		V_{estim}&=(5\:V -\ V_{deriv})-(2.5\: V)+2.5\:V\\
		V_{estim}&=5\: V\ -\ V_{deriv}\\ 
	\end{aligned}
\end{equation*}


Cuando $V_{deriv}$ $\mathrm{>}$ $2.5\:V$: 
\begin{equation*} 
	\begin{aligned}
		V_{estim}&=V^+-\ V^-+2.5V\\ 
		V_{estim}&=2.5V\ -\ (5V-\ V_{deriv})+\ 2.5V\\
		V_{estim}&=V_{deriv}\\
	\end{aligned}
\end{equation*}

Si se toma a $V_{deriv}$ como $V_{deriv}=\mathit{\Delta}V_{deriv}\ +\ 2.5\: V$, al reemplazar en los dos casos se obtiene:
\begin{equation} \label{eq_rest_3}
	V_{estim}= 2.5\:V + |\mathit{\Delta}V_{deriv}|
\end{equation}

\colorbox{red}{La salida siempre esta por encima de 2.5V? No convendría quitarle 2.5v?}CREO QUE DA IGUAL, LO IMPORTANTE ES QUE EL POTENCIOMETRO DE REFERENCIA DE POSICIÓN TENGA VALORES SIMILARES.

\begin{figure}[H]
	\centering
	\includegraphics[scale=0.4]{Restador.png}
	\caption{Circuito restador.}
	\label{fig:img_Restador}
\end{figure}

\subsection{Filtrado}

En el restador se implementa un filtrado adicional a la señal de salida como se observa en la figura  \ref{fig:img_Esquema-circuital-del-restador-con-una-etapa-de-filtrado-en-159}. De esta \'{u}ltima etapa, si se considera $C_5=C_6=C\ $y $R_{33}=R_{31}=R$, se obtiene:



\begin{equation}
	V_{estim}=\frac{1}{1+s*C*R}*{(V}^+-V^-\ +\ 2.5\:V)
\end{equation}

\begin{equation} \label{eq_Vestim_1}
	V_{estim}=\frac{1}{1+s*C*R}*(2.5\: V\ +\ |\mathit{\Delta}V_{deriv}|)
\end{equation}

\begin{equation} \label{eq_Vestim_2}
	V_{estim} \approx \frac{1}{1+s*C*R}*\ |\mathit{\Delta}V_{deriv}|\ +2.5\:V
\end{equation}



Puesto que la salida $V_{estim}$ debe ser una continua, es importante eliminar cualquier posible ripple. Por ello, se escogen los siguientes valores para los componentes:

\colorbox{red}{Tal vez habría que poner como conseguimos estos valores.. algún bode o simulación?}podriamos decir que elegimos un ancho de banda de 1krad/s y en base a eso salen los componentes.

\begin{enumerate}
	\item  $C=10\: nF$
	
	\item  $R=100\:\Omega$
	
	\item  $\frac{1}{2*\pi *C*R}=159.2\: Hz$
\end{enumerate}

\begin{figure}[H]
	\centering
	\includegraphics[scale=0.6]{Esquema-circuital-del-restador-con-una-etapa-de-filtrado-en-159.png}
	\caption{Esquema circuital del restador con una etapa de filtrado en $159.2\: Hz$.}
	\label{fig:img_Esquema-circuital-del-restador-con-una-etapa-de-filtrado-en-159}
\end{figure}


\section{Simulaciones de derivador y rectificador}

A continuación se realizan simulaciones del circuito estimador conformado por los bloques derivador, rectificador y filtrado. Para ello se utilizó el software NL5 y se anexó el circuito estimador dentro del controlador de corriente. En la figura \ref{fig:img_simulacion_derivador+rectificador} se muestra el circuito utilizado para la simulación.

\begin{figure}[H]
	\centering
	\includegraphics[scale=0.75]{simulacion_derivador+rectificador.png}
	\caption{Circuito utilizado para simulación.}
	\label{fig:img_simulacion_derivador+rectificador}
\end{figure}

\subsection{Simulación en régimen permanente}

Se realizó una simulación del controlador de corriente junto con el estimador, utilizando $V_{IL_{ref}}=5\:V$, que resulta en una corriente media del electroimán $\overline{I_L}=30\:A$. En la figura \ref{fig:img_simulacion_derivador+rectificador_ss} se muestran las formas de onda de la corriente del electroimán, la salida del circuito derivador, y la salida del rectificador. 

\begin{figure}[H]
	\centering
	\includegraphics[scale=0.3]{simulacion_derivador+rectificador_ss.png}
	\caption{Formas de onda de corriente, salida del derivador y salida rectificada.}
	\label{fig:img_simulacion_derivador+rectificador_ss}
\end{figure}

Al observar la figura \ref{fig:img_simulacion_derivador+rectificador_ss} es posible notar dos problemas. El primero es que la forma de onda de corriente presenta pendientes de crecimiento y de decrecimiento asimétricas. Esto es provocado por el efecto de la resistencia interna del electroimán, que se hace notorio cuando se trabaja con corrientes elevadas. Esta asimetría en las pendientes provoca que la salida del derivador no sea simétrica alrededor de $V_{op}$. Por lo tanto, como luego se rectifica esta señal pulsada alrededor de $V_{op}$, la señal resultante presentará variaciones. El segundo problema es que la salida del derivador, con forma de onda pulsada, presenta un comportamiento ruidoso tanto en el nivel superior como en el inferior. Esto proviene de la conmutación auxiliar, que también es derivada. 

Por lo tanto, a continuación se analiza en profundidad ambos inconvenientes y se plantean soluciones a los mismos: un filtro pasa bajos en la entrada del estimador y una compensación para los efectos de la resistencia interna.

\section{Diseño del filtro pasa bajos}

Debido a que el derivador amplifica las señales de alta frecuencia se decidió agregar un filtro pasa bajos en su entrada, que permita el paso de la onda triangular de corriente pero que filtre la frecuencia de conmutación auxiliar. 

La señal que ingresa al derivador es $V_{IL_{feed}}$, que es una onda triangular de frecuencia fundamental variable entre $1\:kHz$ y $2\:kHz$. Por lo tanto se permite el paso de sus componentes frecuencia hasta la quinta armónica. De esta manera se diseña que la frecuencia de corte del filtro sea de $20\:kHz$.

Para su implementación se utiliza un filtro activo Butterworth de segundo orden, con una frecuencia de corte en $20\:kHz$. En la figura  \ref{fig:img_Filtro-para-la-entrada-del-derivador} se puede ver el circuito utilizado y en la figura \ref{fig:img_Respuesta-en-frecuencia-del-filtro-activo}, su respuesta en frecuencia.

\begin{figure}[H]
	\centering
	\includegraphics[scale=0.45]{Filtro-para-la-entrada-del-derivador.png}
	\caption{Filtro para la entrada del derivador.}
	\label{fig:img_Filtro-para-la-entrada-del-derivador}
\end{figure}

\colorbox{red}{deberíamos cambiar la imagen poniendo los nombres de la señal que entra (vilfeed) y la que sale (vilfiltrada)}
\begin{figure}[H]
	\centering
	\includegraphics[scale=1]{Filtro-Activo.png}
	\caption{Respuesta en frecuencia del filtro activo.}
	\label{fig:img_Respuesta-en-frecuencia-del-filtro-activo}
\end{figure}

\colorbox{red}{hará falta aclarar qué software usamos para el diseño??}

A continuación, en la figura \ref{fig:img_simulacion_lpf_entrada} se muestra el resultado de una simulación agregando el filtro LPF en la entrada. Esta simulación fue realizada con las mismas condiciones de corriente e inductancia que la de la figura \ref{fig:img_simulacion_derivador+rectificador_ss}.

\begin{figure}[H]
	\centering
	\includegraphics[scale=0.3]{simulacion_con_lpf_entrada.png}
	\caption{Simulacion del estimador con filtro pasa 	bajos en la entrada.}
	\label{fig:img_simulacion_lpf_entrada}
\end{figure}

Se puede observar que, en comparación con la figura \ref{fig:img_simulacion_derivador+rectificador_ss}, la señal de salida del derivador es menos ruidosa. Si bien en la señal $V_{estim}$ no se observa gran diferencia, el agregado del filtro en la entrada mejora la estimación de posición.

\section{Compensación de resistencia interna}

El segundo inconveniente mencionado en la sección de simulaciones es la diferencia entre pendientes de crecimiento y de decrecimiento causada por la resistencia interna del electroimán. A continuación se analiza cómo esta corriente provoca este efecto, y qué solución se propone.

Al circular corriente siempre en el mismo sentido por el electroimán, se produce una caída de tensión casi constante en su resistencia interna. Esto provoca que no siempre estén aplicados $\pm 24\:V$ al electroimán sino que, durante el $T_{ON}$ se aplican $+24\:V-I_L*R_L$ y durante el $T_{OFF}$ se aplican $-24\:V-I_L*R_L$. Esto genera que las pendientes sean distintas. Se puede reescribir la ecuación \ref{eq_derivada_corriente_3} para tener en cuenta el efecto mencionado:

\begin{equation} \label{eq_Vbus-didt-RL}
	\left|\frac{di_L}{dt}\right|=\left|\frac{\pm V_{cc}-R_L*I_L}{L(Y_g)}\right|
\end{equation}

A modo de ejemplo se plantea el caso en que circula la corriente nominal de $21\:A$ y se muestran los valores de la tensión aplicada al electroimán en el semiciclo de crecimiento y del de decrecimiento:

\begin{equation} \label{eq_Vbus-didt-RL-2}
	\pm V_{cc}-R_L*I_L=\ \pm 24\:V-4.2\:V
\end{equation}

Para el caso en que $V_{cc}=24\:V$:

\begin{equation} \label{eq_Vbus-didt-RL-3}
	V_{cc}-R_L*I_L=\ +24\:V-4.2\:V=\ 19.8\:V
\end{equation}

Para el caso en que $V_{cc}=-24\:V$:

\begin{equation} \label{eq_Vbus-didt-RL-4}
	V_{cc}-R_L*I_L=\ -24\:V-4.2\:V=\ 28.2\:V
\end{equation}

Por lo tanto, sobre el electroimán se aplican dos tensiones distintas, en valor absoluto, durante la carga y descarga. Esto provoca que la rampa de corriente sea asimétrica.

Como se planteó en la etapa de simulaciones, esta diferencia de pendientes provoca que la salida del derivador no tenga una excursión simétrica al rededor del punto de operación $V_{op}$. Es por ello que luego, al rectificar esta señal, se obtienen saltos de tensión.

En la figura \ref{fig:img_simulacion_sin_compensacion_IR} se muestra la señal $V_{estim}$ sin su etapa de filtrado, para que se vea mas claro el efecto provocado al rectificar la salida del derivador. 
\colorbox{red}{le saqué el filtro a la salida para que se vean mejor los saltos de tensión}

%Debido a que luego se utilizará un rectificador de onda completa, se desea que la rectificación de cada una de estas pendientes resulte en el mismo valor. En la figura \ref{fig:img_Forma-de-onda-luego-de-rectificar-sin-compensación-IR} se muestra el efecto luego de la rectificación sin realizar ninguna compensación.

\begin{figure}[H]
	\centering
	\includegraphics[scale=0.3]{simulacion_sin_compensacion_IR.png}
	\caption{Forma de onda luego de rectificar sin compensación de resistencia interna.}
	\label{fig:img_simulacion_sin_compensacion_IR}
\end{figure}

La solución a este problema es aumentar el punto de operación de la salida del derivador, de manera de compensar el término $\frac{R_L*I_L}{L(Y_g)}$ que se agrega.

Para modificar el punto de operación se debe cambiar la tensión en la entrada no inversora del circuito derivador ($V_{bias}$) como se muestra en la figura \ref{fig:img_Esquema-circuital-del-derivador}. 

\begin{figure}[H]
	\centering
	\includegraphics[scale=0.5]{Esquema-circuital-del-derivador.png}
	\caption{Esquema circuital del derivador con $V_{bias}$.}
	\label{fig:img_Esquema-circuital-del-derivador}
\end{figure}

Se tiene que la pendiente de bajada de la onda triangular, en módulo, es mayor que la de subida. Por lo tanto, al derivarla (con la inversión de signo), queda por encima del punto de operación, y la pendiente de subida, por debajo. Por ello, se debe compensar el punto de operación para que la forma de onda sea simétrica alrededor de $2.5\:V$. 

Para la pendiente de bajada, la salida del derivador es:

\begin{equation} \label{eq_Vyf-Vbias}
	{Vyf}_{off}\ =\ V_{bias}+H_0\ *\ \tau *\frac{\abs{V_{cc}}\ +\ I_L*R_L}{L(Y_g)}\ 
\end{equation}

Para la pendiente de subida se tiene:

\begin{equation} \label{eq_Vyf-Vbias2}
	{Vyf}_{on}\ =\ V_{bias}\ -\ H_0\ *\ \tau *\frac{\abs{V_{cc}} -\ I_L*R_L}{L(Y_g)}
\end{equation}

Se desea que se cumpla:

\begin{equation} \label{eq_Vyf_Vbias3}
	{Vyf}_{off}\ -2.5\ \:V=\ 2.5\ V\ -{Vyf}_{on}
\end{equation}

Si se despeja $V_{bias}$ se llega a:

\begin{equation} \label{eq_Vyf-Vbias4}
	V_{bias}\ =2.5\ \:V -\ H_0\ *I_L*\ \tau *\frac{\ R_L}{L(Y_g)}
\end{equation}

Se tiene $H_0 = 53.3\:\frac{mV}{A},\; R_L = 0.2\:\Omega$ y $\tau=C_1*R_1=25 \:ms$. Se utiliza la inductancia para $Y_g=4\:mm$ $L(4\:mm) = 16.44\:mHy$.

$V_{IL_{feed}}$ es la tensión de salida del sensor de efecto Hall menos el punto de operación de $2.5\:V$. Sin embargo, debido al cambio de punto de operación agregado al sensor en la sección \ref{sec:ajuste_corrientes_negativas} para llevar su valor medio a $2.6\:V$, al restarle $2.5\:V$ no se produce una cancelación completa sino que quedan $0.1\:V$ de error. Por ello, para implementar la ecuación  \ref{eq_Vyf_Vbias3} se utiliza el circuito mostrado en la figura \ref{fig:img_Generación_de_Vbias}. Este compensa la diferencia de pendientes, el error de $0.1\:V$ y genera $V_{bias}$ para ingresar al derivador.

\begin{figure}[H]
	\centering
	\includegraphics[scale=0.6]{Generación-de-Vbias.png}
	\caption{Generación de $V_{bias}$.}
	\label{fig:img_Generación_de_Vbias}
\end{figure}

\colorbox{red}{acá la que entra es la  vilfeedfiltrada}

A partir del circuito de la figura \ref{fig:img_Generación_de_Vbias} se obtiene:

\begin{equation} \label{eq_Vyf-Vbias3}
	V_{bias} =-\frac{R_4}{R_3}(H_0I_L+ 0.1\:V)+V_{op}(1+\frac{ R_4}{R_3})(\frac{R_1}{R_1+R_2})
\end{equation}

Para poder llegar a la expresión de la ecuación \ref{eq_Vbus-didt-RL-2} se debe cumplir que:

\begin{enumerate}
	\item  $-\frac{R_4}{R_3}=- \tau *\frac{R_L}{L}= -0.304$  
	
	\item  $-\frac{R_4}{R_3}(0.1V)+V_{op}(1+\frac{ R_4}{R_3})(\frac{R_1}{R_1+R_2}) = 2.5V$     
\end{enumerate}

Por lo tanto, al resolver la condición 1) se elige $R_4 = 304\: \Omega$ y se obtiene $R_3=1\:k\Omega$. Luego, al resolver la condición 2) con $V_{op}=2.5\:V$ se elige $R_1=1\:k\Omega$ y se obtiene $R_{2}=291.8\:\Omega$.

Se volvió a simular el circuito estimador con el agregador del circuito de compensación de resistencia interna. En la figura \ref{fig:img_Formas_de_onda_obtenidas_en_la_simulación} se muestra como cambia la forma de onda a la salida del rectificador.

\begin{figure}[H]
	\centering
	\includegraphics[scale=0.3]{simulacion_compensacion_IR.png}
	\caption{Formas de onda obtenidas en la simulación.}
	\label{fig:img_Formas_de_onda_obtenidas_en_la_simulación}
\end{figure}

\colorbox{red}{Esta simulación tiene puesto el filtro pasa bajo a la entrada?} sip, con filtro a la entrada y la señal que entra al compensador IR es la salida del filtro.

La onda superior corresponde a la corriente en el electroimán, la que se encuentra al medio, a la salida del derivador y la inferior, a la onda rectificada con la corrección de la resistencia interna.

En la figura \ref{fig:img_simulacion_completo} se muestra una simulación del estimador completo (con filtro a la salida).

\begin{figure}[H]
	\centering
	\includegraphics[scale=0.3]{simulacion_completo.png}
	\caption{Simulación del circuito estimador completo.}
	\label{fig:img_simulacion_completo}
\end{figure}
\colorbox{red}{la salida roja queda 1v por encima con la compensación IR comparando con la que no tiene compensación (figura 4.16), no se si se notará}

Se puede observar que la salida del estimador presenta menos oscilaciones. Usando cursores se midió una amplitud pico a pico de $97\:mV$. Sin la compensación resulta en $233\:mV$.

\section{Simulación del estimador completo}

Finalmente el circuito estimador completo queda...

\begin{figure}[H]
\centering
\includegraphics[scale=0.3]{circuito_completo.png}
\caption{Circuito final del estimador.}
\label{fig:img_circuito_final_del_estimador}
\end{figure}


\colorbox{red}{la imagen de simulacion abajo sería vieja, pero se podría usar la 4.20 que acabo de hacer}
En la figura \ref{fig:img_Simulación_final_del_estimado} se pueden observar tres formas de onda. La superior corresponde a la corriente del electroimán, la del medio a la salida del derivador y la inferior a la salida $V_{estim}$. Con el uso de los cursores se midió un ripple de $52.66\:mV $ en $V_{estim}$.

\begin{figure}[H]
	\centering
	\includegraphics[scale=0.6]{Simulación-final-del-estimador.png}
	\caption{Simulación final del estimador.}
	\label{fig:img_Simulación_final_del_estimado}
\end{figure}

\colorbox{red}{A que se deben los picos en la onda pulsada? Tal ves podríamos aclararlo no?}


\colorbox{red}{Ponerle los nombres de las señales en la simulacion}


\colorbox{red}{Podríamos agregar la imagen del circuito completo?}



En la tabla \ref{tab_Resultados_de_simulación_del_estimador} se muestran valores medidos de $V_{estim}$ en función de la posición.


\begin{table}[H]
	\begin{center}
		\begin{tabular}{| c | c | c |}
			\hline
			$Y_g\:[mm]$ & $L(Y_g)\:[mHy]$ & $V_{estim}\:[V]$ \\ \hline 
			2 & 22.64 & 3.86 \\ \hline 
			3 & 18.8 & 4.13 \\ \hline 
			4 & 16.44 & 4.36 \\ \hline 
			5 & 14.9 & 4.55 \\ \hline 
		\end{tabular}
		\caption{Resultados de simulación del estimador.}
		\label{tab_Resultados_de_simulación_del_estimador}
	\end{center}
\end{table}

\colorbox{red}{Podríamos agregar los valores teóricos a la tabla, ademas de la simulación}

\section{Transferencia final del estimador de posición}

El comportamiento del circuito estimador no es lineal.  Por lo tanto, para poder modelar una función transferencia, se deben tomar ciertas consideraciones. La parte del derivador es lineal, por lo que se puede modelar su transferencia como:
\begin{equation}\label{eq_v-estim}
	V_{estim}=-0.025*\frac{dVi_L}{dt} 
\end{equation}

A partir de la expresión \ref{eq_v-estim} se puede determinar que, además de realizarse la derivada, se introduce una inversión de signo. De esta forma, una pendiente positiva a la entrada resulta en valores menores a $2.5\:V$ a la salida, mientras que una pendiente negativa produce una tensión mayor a $2.5\:V$.

Luego, el bloque rectificador y restador se encarga de calcular el valor absoluto de esta señal (en torno a los $2.5\:V$). Al considerar que la pendiente aumenta a medida que lo hace la distancia de separación, se puede concluir que el bloque estimador no produce inversión de signo. Por lo tanto, se debe considerar solamente la ganancia del derivador y el polo que introduce la etapa de restador. Finalmente, para poder obtener una estimación de la posición, se utiliza la expresión linealizada \ref{eq_di-dt_lineal} que relaciona la derivada de la corriente con el entrehierro:

\begin{equation} \label{eq_dil_yg}
	\frac{dI_{L}}{dt} = 194690 * Y_{g}
\end{equation}


Al considerar la ganancia que tiene el sensor de efecto Hall sobre la corriente:

\begin{equation} \label{eq_dil_dvil}
	\frac{dI_{L}}{dt} =\frac{dVi_{L}}{dt}*\frac{1}{0.0533}
\end{equation}

Por lo tanto al utilizar \ref{eq_dvil_yg} en \ref{eq_dil_yg} se obtiene \ref{eq_dvil_yg}.

\begin{equation} \label{eq_dvil_yg}
	\frac{dVi_{L}}{dt} = 0.0533*194690*Y_{g}
\end{equation}


En base al resultado anterior, se llega a:



\begin{equation}
	V_{estim}=0.025*194690*0.0533 * \frac{Y_{g}}{1 + \frac{s}{1\:k}}=259.6*\frac{Y_{g}}{1 + \frac{s}{1\:k}}	
\end{equation}

Finalmente, al considerar la etapa de filtrado de la entrada, que tiene dos polos en $2\pi *10\: \:{kHz}\ \simeq 60\: \:{krad/s}$ se obtiene:

\begin{equation} \label{eq_TLC_deriv_7}
	H_{estim}=\frac{V_{estim}}{Y_{g}[m]}=\frac{259.6}{(1+\frac{s}{1\:k})*{(1+\frac{s}{60\: k})}^2}
\end{equation}

