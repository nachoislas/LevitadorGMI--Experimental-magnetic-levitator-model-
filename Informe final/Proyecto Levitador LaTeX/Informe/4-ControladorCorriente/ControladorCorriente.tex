\chapter{Controlador de corriente} \chapterlabel{Informe/4-ControladorCorriente} \label{cap:ControladorCorriente}
\section{Diseño y modelado}

\noindent Para regular la fuerza ejercida por el electroimán es necesario controlar la corriente que circula por él. Para ello, se modela a la planta como la impedancia de un inductor con una resistencia serie, cuya inductancia varía con el gap de aire:

\begin{equation} \label{eq1}
\frac{1}{sL(y)\ +\ R_L}
\end{equation}

\noindent Para realizar este control se utiliza un sistema realimentado, como el que se muestra en la figura \ref{fig:img_diag-en-bloques}. Se puede ver que se ingresa con una tensión de referencia ($V_{in}$) proporcional a la corriente de salida deseada, que luego se multiplica por la ganancia de entrada ($K_{in}$). La corriente del electroimán se realimenta en forma de una tensión proporcional a ella ($V_{iF}$). Ambas tensiones son restadas y el resultado (e) ingresa al bloque de comparador con histéresis, que actúa en conmutación, por lo que su salida tiene dos estados posibles: $\pm$$V_L$.

\noindent Al ser aplicadas al inductor se producirá una rampa de corriente: si la tensión es positiva, la rampa crece, y si es negativa decrece. De esta forma, debido a la conmutación del comparador se obtiene, a la salida, una forma de onda triangular $I_L$, cuyo valor medio es la corriente deseada y se corresponde a la tensión de referencia.

\begin{figure}[H]
	\centering
	\includegraphics[width=\textwidth]{Diagrama-en-bloques.png}
	\caption{Diagrama en bloques simplificado del controlador de corriente.}
	\label{fig:img_diag-en-bloques}
\end{figure}

\subsection{Características del sistema}

\begin{itemize}
    \item Para sensar la corriente se utiliza un sensor de efecto Hall HO 15-NP, con una transconductancia de H(s) = 53.3 mV/A.
    \item Para la ganancia de entrada $K_{in}$ se utiliza un valor de 0.32 puesto que $V_{in}$ varía entre 0 V y 5 V y debe mapearse con una corriente variable entre 0 A y 30 A.
    \item Se adopta una variación de la corriente en torno a su valor medio (ripple) de 500 mA, por lo que resulta en un ancho de histéresis de 26.665 mV.
    \item Según mediciones realizadas sobre el electroimán, la inductancia en el punto de equilibrio y0=4mm es de 16.44 mHy (considerando la inductancia de dispersión de 8.89 mHy) y la resistencia serie es de 0.2 	$\Omega$.
    \item La tensión aplicada sobre el electroimán es +24 V para el estado ON y -24 V para el estado OFF.
    \item Se utiliza un driver de corriente que trabaja en conmutación mediante un puente H con 4 N-MOS. 
\end{itemize}

\subsection{Circuito del controlador de corriente}

\noindent Se comienza planteando la etapa de entrada que consiste en la ganancia de entrada y el restador con la realimentación. El objetivo es imponer una ganancia de entrada de 0.32, y que la salida de esta etapa tenga un punto de operación de 2.5V para poder utilizar una fuente de alimentación entre 0 y 5 V para los operacionales. Para lograr esto se utiliza un circuito como el que se muestra en la figura \ref{fig:img_etapa-de-entrada}.

\begin{figure}[H]
	\centering
	\includegraphics[scale=1]{Etapa-de-entrada.png}
	\caption{Etapa de entrada.}
	\label{fig:img_etapa-de-entrada}
\end{figure}

\noindent Para la implementación del comparador con histéresis se utiliza un amplificador operacional realimentado positivamente. Se implementa un ancho de histéresis de 26.665 mV,  alrededor de un punto de operación de 2.5 V, como se muestra en la figura \ref{fig:img_comp-con-hist}.

\begin{figure}[H]
	\centering
	\includegraphics[scale=1]{Comparador-con-histeresis.png}
	\caption{Comparador con histéresis.}
	\label{fig:img_comp-con-hist}
\end{figure}

\noindent Para controlar la corriente en el electroimán se utiliza una topología en puente H, que permite conmutar la polaridad de la tensión aplicada a la bobina. Para medir la corriente se utiliza un sensor de efecto Hall, que es modelado en la simulación como  una fuente de tensión controlada por corriente, con una ganancia de 53.3 mV/A correspondiente a su transconductancia. Esta implementación puede observarse en la figura \ref{fig:img_puenteH}. Luego, su salida es realimentada a la etapa de entrada luego de restarle la tensión de referencia  $V_{bias}$ de 2.5V, como se muestra en la figura \ref{fig:img_resta-Vbias}. 

\begin{figure}[H]
	\centering
	\includegraphics[scale=0.5]{PuenteH.png}
	\caption{Puente H y sensor de efecto Hall.}
	\label{fig:img_puenteH}
\end{figure}

\begin{figure}[H]
	\centering
	\includegraphics[scale=1]{Resta-Vbias.png}
	\caption{Resta del $V_{bias}$ al sensor de efecto Hall.}
	\label{fig:img_resta-Vbias}
\end{figure}

\subsubsection{Simulaciones de formas de onda}

\begin{figure}[H]
	\centering
	\includegraphics[scale=0.5]{Formas-de-onda-corriente.png}
	\caption{Formas de onda de corriente en el electroimán y salida del comparador.}
	\label{fig:img_formas-de-onda-corriente}
\end{figure}

\noindent En la figura \ref{fig:img_formas-de-onda-corriente} se pueden observar dos formas de onda. La inferior (violeta) se corresponde con la salida del comparador con histéresis, que conmuta. La onda triangular (verde) es la corriente en el electroimán. Para la simulación se utilizó una tensión de referencia de entrada de 1 V, por lo tanto el valor medio de la corriente en la salida es 6 A con un ripple de 500 mA. Esto fue verificado en la simulación mediante cursores.

\subsubsection{Simulación de un escalón en la referencia de corriente}

\noindent En la figura \ref{fig:img_respuesta-al-escalon} se muestra cómo cambia la corriente en el electroimán al aplicarle a la entrada del controlador un escalón de tensión entre 1 y 3 V. Se puede observar cómo la conmutación del comparador se detiene para ajustar la corriente con la referencia.

\begin{figure}[H]
	\centering
	\includegraphics[scale=0.5]{Respuesta-al-escalon.png}
	\caption{Respuesta al escalón del circuito.}
	\label{fig:img_respuesta-al-escalon}
\end{figure}


\subsubsection{Implementación circuital del puente H}

\noindent La corriente que se desea controlar es la que circula por el electroimán y, debido a que el sistema va a trabajar con corrientes elevadas, es importante que la implementación del controlador de corriente sea eficiente. Por lo tanto, para disminuir la disipación de potencia del circuito se utiliza un controlador que funciona en conmutación. 

\subsubsection{Descripción general de la topología}

\noindent Para lograr una corriente contínua en el electroimán utilizando una fuente conmutada se debe alternar la polaridad de la tensión aplicada en los bornes del inductor. Al hacer esto, la corriente crece y decrece (según la polaridad) con forma exponencial debido a la resistencia interna del electroimán. Sin embargo, como el intervalo de tiempo que se mantiene la fuente en positivo o negativo es pequeño comparado con la constante de tiempo de la planta, el incremento de corriente será pequeño y puede ser aproximado a una recta. Por lo tanto se obtiene una corriente contínua (valor medio) con un ripple superpuesto de forma triangular. 

\noindent Para lograr alternar la polaridad de la fuente sobre el inductor se utiliza una topología en puente H con 4 MOSFET que funcionan con un ciclo de trabajo determinado (manejado por el controlador por histéresis) como se observa en la figura \ref{fig:img_topologia-puenteH}. Pueden diferenciarse dos semiciclos de trabajo: uno de estado ON y otro de estado OFF. El estado ON se define como el semiciclo durante el cual la corriente en el inductor crece (pendiente positiva), mientras que el estado OFF se da cuando la corriente decrece.

\begin{figure}[H]
	\centering
	\includegraphics[scale=0.7]{Topologia-puenteH.png}
	\caption{Topología elemental del puente H.}
	\label{fig:img_topologia-puenteH}
\end{figure}

\noindent El electroimán se conecta entre los puntos medios de cada par de transistores. De esta manera se puede conmutar la polaridad de la tensión que se le aplica. Sólo se permite que dos transistores se enciendan a la vez, y esto se realiza de manera diagonal. Es decir, en la figura \ref{fig:img_topologia-puenteH}, Q1 y Q4 pueden estar encendidos, mientras que Q3 y Q2 están apagados, y viceversa. Es necesario evitar que se enciendan Q1 y Q2 a la vez, o Q3 y Q4, ya que ocurriría un cortocircuito entre la fuente de alimentación y GND, lo que produciría una circulación de corriente denominada shoot-through. 

\noindent Los 4 MOSFET utilizados para el puente H son de tipo N (pues es complicado conseguir un MOS tipo P de potencia adecuado). Para que estos puedan funcionar correctamente en conmutación es necesario que en el estado ON, la diferencia de tensión entre gate y source sea mayor o igual a 7 V. Esto no es un problema para los dos MOS inferiores del puente H (Q2 y Q4), ya que la tensión en source está fijada en GND y el driver puede aplicar 12 V al gate (superando los 7 V entre gate y source). El problema radica en los transistores superiores del puente H, ya que la tensión en source varía entre 0 V y 24 V, por lo que en el gate debería haber, por lo menos, 31 V con respecto a GND. Sin embargo, la tensión máxima disponible entregada por la fuente es de 24 V. Para resolver este problema se utiliza un driver flotante con bootstrap.

\noindent Para controlar la conmutación se utiliza un mosfet driver HIP4081A que se encarga de encender y apagar los transistores según las entradas de control. Además permite la configuración de un tiempo muerto para evitar que se enciendan dos transistores de un lado a la vez. También provee la circuitería necesaria para implementar la fuente flotante que enciende los mosfet del lado superior para lo cual solo se debe agregar un diodo y un capacitor de manera externa. Para la implementación circuital se van a utilizar los MOSFET IPB160N04.

\noindent En la figura \ref{fig:img_bootstrap} se observa solo una de las mitades del puente H (lado A)  junto con las señales de control provistas por el driver HIP4081A. El análisis para la otra mitad es análogo, por lo que se evita por simplicidad. La implementación del driver bootstrap permite obtener en el gate del MOS superior, una tensión de 36 V respecto a GND, logrando así una diferencia de tensión mayor a 7 V entre gate y source. 

\begin{figure}[H]
	\centering
	\includegraphics[scale=0.7]{Bootstrap.png}
	\caption{Configuración Bootstrap simplificada.}
	\label{fig:img_bootstrap}
\end{figure}

\noindent El driver bootstrap consiste en un capacitor ($C_{BS}$), un diodo, y la circuitería interna del HIP4081A. Para garantizar el correcto funcionamiento del bootstrap, al encender el sistema, la secuencia de inicio del HIP4081A enciende las dos salidas de la parte inferior del puente H: ALO y BLO con el fin de encender Q2 y Q4 durante un tiempo que se conoce como periodo de refresco de bootstrap. De esta forma, los capacitores de bootstrap de ambos lados quedan conectados a GND y se pueden cargar completamente. Durante este tiempo, las salidas a los gates AHO y BHO se mantienen en bajo continuamente lo que asegura que no se produzca corriente de shoot-through durante el período nominal de refresco del bootstrap. Al final de este período las salidas responden normalmente al estado de las señales de entrada de control.

\noindent Para comprender su funcionamiento se hará un breve análisis del sistema. Para ello, se parte suponiendo que el sistema se encuentra funcionando: con el transistor Q2 encendido (ALO = $V_{CC}$, Q1 apagado (AHO = AHS = 0 V) y la corriente circulando de izquierda a derecha como lo indica la figura \ref{fig:img_bootstrap}. En ese caso, el capacitor $C_{BS}$ se carga a 12 V, ya que en un terminal tiene la fuente de 12 V (a través del diodo $D_{BS}$) y el otro está conectado a GND por medio de Q2.

\noindent Una vez que se apaga el transistor inferior, empieza a transcurrir el tiempo muerto. Teniendo en cuenta que la carga es inductiva, el valor medio de la corriente mantiene su sentido circulando por los diodos antiparalelos del MOS inferior del lado A y el superior del lado B. Esto provoca que el source del MOS superior del lado A tenga una tensión negativa igual a la caída de tensión en directa del diodo antiparalelo de Q2. 

\noindent Una vez finalizado el tiempo muerto, se enciende el MOS Q1. Para encenderlo, la señal AHO se pone en nivel alto. Durante el tiempo que Q1 pasa de estar apagado a encendido, la tensión en el source cambia de $-V_d$ a $V_{bus}$ de manera gradual mientras se carga el gate, y AHO pasa a ser igual a AHB, que es igual a la tensión entregada por el capacitor de bootstrap sumada a la tensión en el source de Q1. De esta manera se logra una tensión de 36 V con respecto a GND en el gate y genera una diferencia entre gate y source de 12 V.

\noindent Para lograr un funcionamiento adecuado del Boostrap es necesario dimensionar correctamente al capacitor $C_{BS}$ con el fin de que pueda proveer la carga suficiente durante el tiempo en el que el MOS esté encendido.


\subsubsection{Dimensionamiento de capacitor de bootstrap}

\noindent Para el dimensionamiento se tuvieron en cuenta sugerencias y procedimientos descripto en [1] y [2].

\noindent Para encender un NMOS es necesario proveer corriente a su gate hasta cargar las capacidades parásitas entre gate-source y gate-drain. Una vez cargadas, el MOS queda en estado encendido y no consume más corriente en el gate. En el caso de los MOS del lado superior, esta corriente proviene del capacitor de bootstrap. 

\noindent En la implementación del puente H se decidió colocar resistencias entre gate y source. Estas aparecen como R1, R2, R3 y R4 en la figura \ref{fig:img_capacitores-puenteH}. Debido a la diferencia de tensión entre gate-source, se genera una corriente constante en estas resistencias durante el tiempo que el MOS esté encendido, que también debe ser provista por el  bootstrap.

\begin{figure}[H]
	\centering
	\includegraphics[scale=0.5]{Capacitores-puenteH.png}
	\caption{Puente H.}
	\label{fig:img_capacitores-puenteH}
\end{figure}

\noindent Por otro lado, el capacitor debe entregar corriente al diodo de bootstrap cuando este queda en inversa ($I_{DR}$ ), y también entregar una corriente de fuga al circuito integrado HIP ($I_{QBS}$ ). Esta última se desprecia ya que es compensada internamente por la bomba de carga del HIP.

\noindent Por lo tanto, para poder dimensionar correctamente el capacitor de bootstrap es necesario tener en cuenta todos estos efectos mencionados anteriormente. Para ello se parte planteando la carga que almacena el capacitor bootstrap. Esta se obtiene como:


























