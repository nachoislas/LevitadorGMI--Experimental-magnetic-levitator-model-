\chapter{COSAS QUE LUEGO DEBEN SACARSE DEL ESCRITO} \chapterlabel{Varios} \label{cap:varios}



\section{Cosas a CORREGIR del Escrito 22-02-2020}

\begin{itemize}
	\item \textbf{En cada capítulo poner al principio un resumen de lo que se va a tratar en el mismo. Poner algo similar a lo que se escribió sobre dicho capítulo al final de la Introducción general del trabajo.}
	\item Poner texto más explicativo en las figuras y las tablas. Que leyendo el epígrafe se pueda entender de qué se trata la foto. Por ejemplo, “imagen original” o “imagen binarizada” dicen poco o nada.
	\item Verificar que las figuras y tablas tengan todas nombres diferentes, no repetir.
	\item Inventar un nombre al sistema con una sigla asociada, para que sea más fácil referirse al mismo, sin tener que llamarlo siempre “Sistema de estacionamiento automatizado” (Sistema Automatizado de Estacionamiento -SAE-).
	\item Al esquema que hice en 3D del estacionamiento (está en el cap uno de estructura) agregar algún grafico similar al de la página web china (q muestra entrada y salida, en la Solution 1)
	\item Ver si usamos la palabra “Núcleo” para referirnos a las 2 partes más importantes de la tesis, o buscamos una mejor.
	\item Cambiar nombre a la placa (Placa I/O es muy cabeza).
	\item ¿AGREGAR?: El proyecto se escribió utilizando las herramientas TexStudio versión 2.12.14 (editor de Latex) y la 
	distribución de Latex MiKTeX 2.9.
\end{itemize}





\section{Aspectos aprendidos durante el desarrollo del Proyecto Final -no solo poner cosas técnicas}
\begin{itemize}
	\item Cómo documentar
	\item Interacción con personal, al realizar el relevamientos de los estacionamientos
	\item Manejo y configuración de un sistema tipo caja negra
	\item A trabajar en equipo
	\item Primeros pasos en C++
	\item Altium
	\item Latex
	\item Etc
\end{itemize}








