\chapter{Conclusión} \chapterlabel{Informe/9-Conclusion} \label{cap:Conclusión}

En el desarrollo del proyecto fue posible cumplir con los requerimientos planteados. Mediante simulaciones se verificó, tanto para la implementación analógica como para la digital, que el sistema presenta un comportamiento estable para todo el rango de separación entre piezas y para todo el rango de masas para el que fue diseñado. 

En el transcurso de la ejecución de las distintas etapas del proyecto fue posible ganar experiencia en la utilización de diversas herramientas de computadora. Por ejemplo, en el diseño del circuito impreso se utilizó el programa Altium Designer. Para la simulación de circuitos y sistemas se utilizaron los programas Matlab y NL5. Además, para poder trabajar de manera organizada se utilizó el software para el control de versiones Git y, para la escritura de este informe, se utilizó Latex. Por otra parte, fue posible desarrollar habilidades interpersonales, como trabajo en equipo, comunicación, gestión del tiempo, compromiso, dedicación, entre otras. Todas estas habilidades son necesarias e indispensables para el desarrollo personal y profesional.

En este informe fue posible documentar el diseño y funcionamiento de cada una de las etapas que componen al sistema y el criterio utilizado para la elección de cada topología y componente. Por otra parte, se dejan a disposición de la cátedra  los archivos de fabricación de la placa de control para que en un futuro sea enviada a prototipar y pueda ser utilizada para realizar prácticas.

Es importante destacar que el proyecto se realizó prácticamente en su totalidad bajo la situación de emergencia sanitaria debido al COVID-19. Esto ocasionó demoras en el tiempo de ejecución de cada una de las etapas con respecto a los plazos estimados. Los mayores retrasos se dieron en su comienzo, debido a que fue el período de mayor incertidumbre de la pandemia. Sin embargo, a pesar de no poder realizar encuentros presenciales entre los integrantes ni reuniones con los directores en los establecimientos educativos, se pudo avanzar hasta su finalización  mediante el uso de plataformas de comunicación virtuales.

Debido a la extensión en los plazos temporales del proyecto fue necesario acortar el alcance al modelado teórico de todas las etapas que componen el sistema, junto con el diseño del circuito impreso.

A pesar de que no se concretó la construcción del prototipo, fue posible adquirir conocimientos en distintos conceptos propios de la electrónica y la ingeniería en general. Se pudo modelar un problema físico real y, mediante un sistema de control, se modificó su comportamiento de la manera deseada. Para ello se utilizaron estrategias de compensación y estimación de variables tanto en el dominio analógico como digital, además de diseñar una etapa de control de corriente eficiente para trabajar con potencias elevadas. Finalmente se integraron todas las partes en un circuito impreso compuesto por etapas de electrónica analógica, etapas de potencia, y etapas de interfaz para la comunicación con un microcontrolador.






%
%
%
%
%
%%ALTERNATIVA 1
%Es importante destacar que el proyecto se realizó prácticamente en su totalidad bajo la situación de emergencia sanitaria debido al COVID-19. Esto ocasionó demoras en el tiempo de ejecución de cada una de las etapas con respecto a los plazos estimados. Los mayores retrasos se dieron en su comienzo, debido a que fue el período de mayor incertidumbre de la pandemia. Sin embargo, a pesar de no poder realizar encuentros presenciales entre los integrantes ni reuniones con los directores en los establecimientos educativos, se pudo avanzar hasta su finalización  mediante el uso de plataformas de comunicación virtuales.
%
%Debido a la extensión en los plazos temporales del proyecto fue necesario acortar el alcance al modelado teórico de todas las etapas que componen el sistema, junto con el diseño del circuito impreso.
%
%A pesar de que no se concretó la construcción del prototipo, fue posible adquirir conocimientos en distintos conceptos propios de la electrónica y la ingeniería en general. Se pudo modelar un problema físico real y, mediante un sistema de control, se modificó su comportamiento de la manera deseada. Para ello se utilizaron estrategias de compensación y estimación de variables tanto en el dominio analógico como digital, además de diseñar una etapa de control de corriente eficiente para trabajar con potencias elevadas. Finalmente se integraron todas las partes en un circuito impreso compuesto por etapas de electrónica analógica, etapas de potencia, y etapas de interfaz para la comunicación con un microcontrolador.
%
%Se pudo cumplir con los requerimientos planteados para este proyecto. Mediante simulaciones se verificó, tanto para la implementación analógica como para la digital, que el sistema presenta un comportamiento estable para todo el rango de separación entre piezas y para todo el rango de masas para el que fue diseñado. 
%
%En el transcurso de la ejecución de las distintas etapas del proyecto fue posible ganar experiencia en la utilización de diversas herramientas de computadora. Por ejemplo, en el diseño del circuito impreso se utilizó el programa Altium Designer. Para la simulación de circuitos y sistemas se utilizaron los programas Matlab y NL5. Además, para poder trabajar de manera ordenada se utilizó el software para el control de versiones Git y, para la escritura de este informe, se utilizó Latex. Por otra parte, fue posible desarrollar habilidades interpersonales, como trabajo en equipo, comunicación, gestión del tiempo, compromiso, dedicación, entre otras. Todas estas habilidades son necesarias e indispensables para el desarrollo personal y profesional.
%
%En este informe fue posible documentar el diseño y funcionamiento de cada una de las etapas que componen al sistema y el criterio utilizado para la elección de cada topología y componente. Por otra parte, se dejan a disposición de la cátedra  los archivos de fabricación de la placa de control para que en un futuro sea enviada  a prototipar y pueda ser utilizada para realizar prácticas.
%
